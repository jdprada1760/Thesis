\chapter{Quantum analysis of the three body problem in the plane}
In this chapter we continue the analysis implemented in \cite{alonso}, explaining in detail the quantum treatment of the three body problem in the plane. To achieve this goal, the quantum problem will be defined at the same time as the quantum analogous of the classical reduction of problem is going to be briefly discussed. The Schwinger oscillator is going to be explained to get some insights about the quantum analogue analysis, and finally it is going to be applied to the problem.\\

\section{The quantum reduction of the problem}
The quantum treatment of the three body problem on the plane is not very different from the classical approach, given that we thoroughly worked the classical system out in the Hamiltonian formalism. In fact, all the transformations and reductions from the classical approach work; however, we have to be careful when treating topics referring equations of movement.  The only thing we have to do is to apply the principles of canonical quantization \cite{Cq}. To make the problem of parsing to the quantum formalism less complicated, let us carry the analysis in fundamental units $\hbar =1$. \\

Including the last considerations, the quantum Hamiltonian is going to be the same as the one expressed in \eqref{eq:ham2d}, and as canonical transformations still work in quantum mechanics, a similar reduction of the problem can be performed.\\

%In this case, the classical guiding centres motion decoupling from the linear momenta $\vec{\pi}_i$ will produce a separable quantum Hamiltonian in the same fashion. However, the decoupling cannot be applied to the states, at least not completely. What we can do is consider pure states that can be totally separable, bearing in mind that at the end the final general solution  may consist of a linear combination of those pure states which may not be separable.Once left this clear, we can continue with the reduction of the problem in the same fashion we did in the classical formalism.%\\

First, given the big magnetic field decoupling, the problem can be reduced to the analysis of the guiding centres motion given by the Hamiltonian \eqref{eq:newham}. The decoupled system of linear momenta is equally identified as a quantum harmonic oscillator, which is also very well known amongst the physical sciences community.\\

%Puzzling comment on the Heissenbergs uncertainty principle consequences of the canonical transformation.\\%

Therefore, we are left with the reduced Hamiltonian for the guiding centres \eqref{eq:newham} with the same definition of the canonical coordinates. The integrability analysis carried out in Chapter 2 where there were found two integrals of motion in involution with the Hamiltonian, is not meaningful on the quantum formalism. Actually, speaking of integrability or chaos on quantum systems is somewhat complicated since the Schr\"odinger equation is lineal. The analysis of the consequences of integrability of classical systems on their quantum counterparts is a topic that goes beyond the scope of this monograph.\\


\section{The spinorial transformation}
There are some remarkable aspects referring the canonical variables $(\bar{x},\bar{y})$ associated with \eqref{eq:newham}. With these one can form creation-annihilation operators given by:\\
 then 
\begin{align*}
a_j &= \frac{1}{\sqrt{2}}(x_j+iy_j)\\
a_j^\dagger &= \frac{1}{\sqrt{2}}(x_j-iy_j)\\
\left[a_j,a_k^\dagger\right] &= \delta_{kj}.
\end{align*}

We then note that the variables called $z_i$ in the classical formalism will correspond to annihilation operators in the quantum formalism.\\

Following the reduction of the problem analogously to the classical approach, we can proceed to decouple the center of mass coordinates form the relative coordinates. The spinor $\Psi$ is known to follow relations of creation-annihilation operators (taking into account the canonical quantisation), however, the center of mass was not formalised as a canonical variable in the classical approach. To take this into account, we choose the constants to make it also an annihilation operator. If we encode the center of the triangle in $b$, the decoupling transformation can be expressed in the following way:\\

\begin{align*}
\begin{pmatrix} b \\ \Psi_1 \\ \Psi_2 \end{pmatrix} &= 
\begin{pmatrix}\frac{1}{\sqrt{3}} &\frac{1}{\sqrt{3}}&\frac{1}{\sqrt{3}}\\
				-\frac{1}{\sqrt{2}}&\frac{1}{\sqrt{2}}&0\\
				\frac{1}{\sqrt{6}}&\frac{1}{\sqrt{6}}&\frac{1}{\sqrt{6}}\end{pmatrix},
\end{align*}

where the new variables satisfy the creation-annihilation operators relation:

\begin{align*}
\left[ \Psi_\alpha,\Psi_\beta^\dagger\right] &= \delta_{\alpha\beta}\\
\left[ b,b^\dagger\right] &= 1\\
\left[ b,\Psi_\alpha^\dagger\right] &=\left[ b,\Psi_\alpha\right] = 0\\.
\end{align*}

From this transformation it is worth noticing:

\begin{align*}
b^\dagger b &= \frac{1}{3}\roundP{\sum_i a_i^\dagger}\roundP{\sum_j a_j} =  \frac{1}{6}\roundP{\sum_i x_i-iy_i}\roundP{\sum_j x_j+iy_j}\\
&= \frac{1}{6}\sum_i\sum_j(x_i-iy_i)(x_j-iy_j) = \frac{1}{6}\sum_i\sum_jx_ix_j+y_iy_j +i\cancelto{i\delta_{ij}}{[xi,y_j]}\\
&= L - \frac{1}{2},
\end{align*}
\small
\begin{align*}
\Psi^\dagger\Psi &= \Psi_1^\dagger\Psi_1 +\Psi_2^\dagger\Psi_2\\
&= \frac{1}{2} (a_2^\dagger-a_1^\dagger)(a_2-a_1) + \frac{1}{6}(a_2^\dagger+a_1^\dagger-2a_3^\dagger)(a_2+a_1-2a_3)\\
&= \frac{1}{2} (x_2-x_1 -i(y_2-y_1))(x_2-x_1 +i(y_2-y_1))\\
&+\frac{1}{6}(x_2+x_1-2x_3-i(y_2+y_1-2y_3))(x_2+x_1-2x_3 +i(y_2+y_1-2y_3))\\
&= \frac{1}{2} (x_2-x_1)^2+(y_2-y_1)^2+i\cancelto{2i}{[x_2-x_1,y_2-y_1]} \\
&+ \frac{1}{6} (x_2+x_1-2x_3)^2+(y_2+y_1-2y_3)^2+i\cancelto{3i}{[x_2+x_1-2x_3,y_2+y_1-2y_3]}\\
&= S-1.
\end{align*}
\normalsize

%This tells us that the eigenvalues from operator $b$ are semi-integers, while those of $\Psi$ are quantized integers.\\

So far we have reduced the problem to the study of the center of mass and relative coordinates spinor. In this terms, the Hamiltonian for these variables will be given by:\\

\begin{align*}
H_{gc} &= V\roundP{\Psi,\Psi^\dagger}+ \omega (S+L)\\
&= V\roundP{\Psi,\Psi^\dagger} + \omega(\Psi^\dagger\Psi + b^\dagger b +\frac{3}{2})\\
&= V\roundP{\Psi,\Psi^\dagger} + \omega(\Psi_1^\dagger\Psi_1 +\Psi_2^\dagger\Psi_2+ 1) +\omega(b^\dagger b +\frac{1}{2}).
\end{align*}

Ignoring the effects of the center of mass $b$, what we have here is the Hamiltonian of two oscillators $(\Psi_1,\Psi_2)$ coupled by the potential $V$ which does not affect the center of the triangle. What we will see next is that in fact, the two oscillators from $\Psi$ induce a Schwinger angular momentum \cite{Schwinger}.\\

In quantum mechanics the symmetries of a system play a substantial role in the solution of the associated problem. Actually, the role of the symmetries in quantum mechanics is more important than in classical mechanics for it is more sensitive to the internal structure of the symmetry groups \cite{qfhebook}. In quantum mechanics they determine the algebra of operators, and consequently, the eigenvectors and eigenvalues that generate the states of the Hilbert space. In this case, we know that the potential $V$ follows the symmetries associated with the special Euclidean group in 2D ($SE(2)$), that, given the simplification of the problem and the canonical spinor variables, is transformed into $SU(2)$.\\

The role of this symmetry can be understood as the operators $\Psi_\alpha$ implement the mentioned Schwinger oscillator \cite{Schwinger} associated with this symmetry.\\

\section{The Schwinger oscillator and the angular momentum representation}
To understand better the role of the symmetry of $\Psi$ in the problem, let us explain the theory associated with the Schwinger oscillators. In a nutshell, the Schwinger formalism demonstrates that the annihilation-creation operator algebra associated to two decoupled quantum oscillators induce a $SU(2)$ algebra of angular momentum.\\

To understand this, let us first consider two decoupled harmonic oscillators with the same frequency $\omega$. The Hamiltonian of this system in terms of the annihilation-creation operators $(a_1,a_2)$ of each oscillator, will be given by:

\begin{equation*}
H_{SO} = \omega\roundP{a_1^\dagger a_1 + a_2^\dagger a_2 + 1}.
\end{equation*}

As the oscillators are decoupled, we can define the excitation number hermitian operators $n_i = a_i^\dagger a_i$. Some important commutation relations between these operators are then given by:

\begin{align*}
\left[ a_i,a_j^\dagger\right] &= \delta_{ij}\\
\left[ a_i,a_j\right] &= 0\\
\left[ n_i, a_j \right] &= \left[ n_i, a_j^\dagger \right] = \left[ n_i, n_j \right] =0,
\end{align*}
 
Now we are going to show that this algebra of annihilation operators induces an angular momentum algebra with symmetry $SU(2)$. To do that, let us define an angular momentum in terms of the $\{a_i\}$ and see its commutation relations:

\begin{equation}
\begin{aligned}
J_i &= \frac{1}{2}\sigma^i_{\alpha\beta}a^\dagger_\alpha a_\beta \\
J_1 &= \frac{1}{2} \roundP{a_2^\dagger a_1 + a_1^\dagger a_2}\\
J_2 &= \frac{i}{2} \roundP{a_2^\dagger a_1 - a_1^\dagger a_2}\\
J_3 &= \frac{1}{2} \roundP{n_1-n_2}
\end{aligned},
\label{eq:schwingerAM}
\end{equation}

\begin{align*}
[J_i,J_j] &= \frac{1}{4}\left[ \sigma^i_{\alpha\beta}a^\dagger_\alpha a_\beta,\sigma^j_{\gamma\delta}a^\dagger_\gamma a_\delta \right] = \frac{1}{4}\sigma^i_{\alpha\beta}\sigma^j_{\gamma\delta}\left[ a^\dagger_\alpha a_\beta,a^\dagger_\gamma a_\delta \right]\\
&= \frac{1}{4}\sigma^i_{\alpha\beta}\sigma^j_{\gamma\delta}\roundP{a^\dagger_\alpha [a_\beta,a^\dagger_\gamma ]a_\delta +  a^\dagger_\gamma[a^\dagger_\alpha, a_\delta]a_\beta }\\
&= \frac{1}{4}\roundP{\sigma^i_{\alpha\beta}\sigma^j_{\beta\delta}a^\dagger_\alpha a_\delta - \sigma^i_{\alpha\beta}\sigma^j_{\gamma\alpha}a^\dagger_\gamma a_\beta } = \frac{1}{4}a^\dagger_\alpha a_\beta \left[ \sigma^i,\sigma_j \right]_{\alpha\beta}\\
&= i\epsilon_{ijk} \frac{1}{4} \sigma^k_{\alpha\beta}a^\dagger_\alpha a_\beta \\
\\
[J_i,J_j] &= i\epsilon_{ijk}J_k.
\end{align*}

Then as well as with any other angular momentum, one can get the operators $J^\pm = J_1\pm J_2$ and get, based on these, the rules for the quantised numbers $(j,m)$ that define the eigenvalues of $(J^2,J_3)$. Then, let us calculate $J^2$ in terms of the hermitian excitation number operators $(n_1,n_2)$:

\begin{align*}
J^2 &= J_1J_1+J_2J_2+J_3J_3\\
&= \frac{1}{4}( (n_1-n_2)^2 +(-\cancel{a_2^\dagger a_1a_2^\dagger a_1}- \cancel{a_1^\dagger a_2a_1^\dagger a_2}+n_1a_2a_2^\dagger+n_2a_1a_1^\dagger)\\
&+(\cancel{a_2^\dagger a_1a_2^\dagger a_1}+ \cancel{a_1^\dagger a_2a_1^\dagger a_2}+n_1a_2a_2^\dagger+n_2a_1a_1^\dagger))\\
&= \frac{1}{4}\roundP{(n_1-n_2)^2 + 2n_1(n_2+1)+2n_2(n_1+1)} = \frac{1}{4}\roundP{(n_1+n_2)(n_1+n_2+4)}\\
&= \frac{n_1+n_2}{2}\roundP{\frac{n_1+n_2}{2}+1}.
\end{align*}

Now, comparing this result with the eigenvalues $j(j+1)$ of $J^2$ it is clear that $j = \frac{n_1+n_2}{2}$. Furthermore, the eigenvalues $m$ of $J_3$ are given by  $m = \frac{n_1-n_2}{2}$. As a consequence, given that $n_1+n_2 = 1,2,...$, we conclude that the quantum number $j$ is going to be some semi-integer $j= \frac{1}{2},1,\frac{3}{2},...$. These results lead us to conclude that the symmetry group associated with this angular momentum algebra is SU(2).\\

To sum up, given the proceedings in this section, the algebra of creation-annihilation operators that characterise a set of 2 decoupled harmonic oscillators induces an angular momentum algebra with symmetry $SU(2)$.

\section{Quantum diagonalization of the Hamiltonian}
Now that we introduced some important aspects about Schwinger oscillators and its relation to $SU(2)$ symmetries, we are now able to apply this theory to the analysis of our reduced problem. The only thing we have to do is take the operators $\Psi_\alpha$ as the annihilation-creation operators of the decoupled harmonic oscillator. Given this, we obtain an angular momentum in accordance to equation \eqref{eq:schwingerAM} given by:

\begin{align*}
F_i  &= \frac{1}{2}\Psi_\alpha^\dagger \sigma^i_{\alpha,\beta}\Psi_\beta\\
\vec{F} &= \frac{1}{2}\Psi^\dagger \vec{\sigma}\Psi.
\end{align*}

Certainly, as seen before, this angular momentum satisfies $[F_i,F_j] = i\epsilon_{ijk}F_k$. Moreover, this angular momentum has the same definition as the Bloch sphere shape vector $\vec{\zeta}$ with the exception of the constants, namely $\frac{2}{S}$. We can indeed relate its components to the shape of the triangle and its area in a similar way as we did in \eqref{eq:compressed}:

\begin{align*}
\rho_i &= 2S+4\vec{m}_i\cdot\vec{F}\\
A &= \sqrt{3}F_2.
\end{align*}

On the other hand, the operator $F^2$, in analogy with $J^2$ studied before, will be given by: 

\begin{align*}
F^2 &= \frac{n_1+n_2}{2}\roundP{\frac{n_1+n_2}{2}+1} = \frac{\Psi_1^\dagger \Psi_1 + Psi_2^\dagger \Psi_2}{2}\roundP{\frac{\Psi_1^\dagger \Psi_1 + \Psi_2^\dagger \Psi_2}{2}+1}\\
&= \frac{S-1}{2}\roundP{\frac{S-1}{2}+1}\\
&= \frac{1}{4}(S^2-1).
\end{align*}

The $SU(2)$ algebra relation then becomes clear. First, because it is already related to the symmetry of the shape vector $\vec{\zeta}$ and then because we deduced it in the Schwinger oscillator implementation.\\

The quantum number associated with this angular momentum $\vec{F}$ are clearly given by $(s= S,m)$ where $m$ is some eigenvalue of some component of the shape vector $\vec{F}$. This, together with the quantum number $l=L$ characterises an eigenstate of our system $H_{gc}$. The eigen-energies associated with an eigenstate $(l,m,s)$ are then given by:

\begin{equation*}
E_{l,m,s} = V_{s,m} + \omega(s+l).
\end{equation*}

With this, we consider the quantum problem fully analysed, at least for the purpose of this document.
















