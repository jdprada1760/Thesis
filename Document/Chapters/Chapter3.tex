\chapter{Quantum analysis of the three body problem in the plane}
In this chapter we continue the analysis implemented in \cite{alonso}, explaining in detail the quantum treatment of the three body problem in the plane. To achieve this goal, the quantum problem will be defined at the same time as the quantum analogous of the classical reduction of problem is going to be briefly discussed; then, the remaining problem is going to be analysed; and finally, some intriguing and rare aspects of this problem are going to be clarified.\\

\section{The quantum three body problem on the plane}
The quantum treatment of the three body problem on the plane is not very different from the classical approach, given that we thoroughly worked the classical system out in the Hamiltonian formalism. In fact, all the transformations and reductions from the classical approach work, however, we have to be careful when treating topics referring equations of movement.  The only thing we have to do is to apply the principles of canonical quantization \cite{Canonical quantization}. To make the problem of parsing to the quantum formalism less complicated, let us carry the analysis in fundamental units $\hbar =1$. \\

Including the last considerations, the quantum Hamiltonian is going to be the same as the one expressed in \eqref{eq:ham2d}, and as canonical transformations still work in quantum mechanics, a similar reduction of the problem can be performed.\\

%In this case, the classical guiding centres motion decoupling from the linear momenta $\vec{\pi}_i$ will produce a separable quantum Hamiltonian in the same fashion. However, the decoupling cannot be applied to the states, at least not completely. What we can do is consider pure states that can be totally separable, bearing in mind that at the end the final general solution  may consist of a linear combination of those pure states which may not be separable.Once left this clear, we can continue with the reduction of the problem in the same fashion we did in the classical formalism.%\\

First, given the big magnetic field decoupling, the problem can be reduced to the analysis of the guiding centres motion given by the Hamiltonian \eqref{eq:newham}. The decoupled system of linear momenta is equally identified as a quantum harmonic oscillator, which is also very well known amongst the physical sciences community.\\

%Puzzling comment on the Heissenbergs uncertainty principle consequences of the canonical transformation.\\%

Therefore, we are left with the reduced Hamiltonian for the guiding centres \eqref{eq:newham} with the same definition of the canonical coordinates. The integrability analysis carried out in Chapter 2 where there were found two integrals of motion in involution with the Hamiltonian, may be interpreted as the finding of a complete set of commutative observables (CSCO) \cite{csco}. This means that the Hamiltonian of the guiding centres can be simultaneously diagonalised with the operators $L$ and $J$ that can be interpreted as orbital and total angular momentum respectively. However, as the problem can be refined to a better extent, this occurrence is currently of no interest for us.\\

Now, there are some remarkable aspects referring the canonical variables $(\bar{x},\bar{y})$ associated with \eqref{eq:newham}. With these one can form creation-annihilation operators given by:\\

\begin{align*}
a_j &= \frac{1}{\sqrt{2}}(x_j+iy_j)\\
a_j^\dagger &= \frac{1}{\sqrt{2}}(x_j-iy_j)\\
\left[a_j,a_k^\dagger\right] &= \delta_{kj}
\end{align*}

We note then that the variables called $z_i$ in the classical formalism will correspond to annihilation operators in the quantum formalism.\\

Following the reduction of the problem analogously to the classical approach, we can proceed to decouple the center of mass coordinates form the relative coordinates. The spinor $\Psi$ is known to follow relations of creation-annihilation operators (taking into account the canonical quantisation), however, the center of mass was not formalised as a canonical variable in the classical approach. To take this into account, we choose the constants to make it also an annihilation operator. If we encode the center of the triangle in $b$, the decoupling transformation can be expressed in the following way:\\

\begin{align*}
\begin{pmatrix} b \\ \Psi_1 \\ \Psi_2 \end{pmatrix} &= 
\begin{pmatrix}\frac{1}{\sqrt{3}} &\frac{1}{\sqrt{3}}&\frac{1}{\sqrt{3}}\\
				-\frac{1}{\sqrt{2}}&\frac{1}{\sqrt{2}}&0\\
				\frac{1}{\sqrt{6}}&\frac{1}{\sqrt{6}}&\frac{1}{\sqrt{6}}\end{pmatrix}
\end{align*}

Where the new variables satisfy the creation-annihilation operators relation:

\begin{align*}
\left[ \Psi_\alpha,\Psi_\beta^\dagger\right] &= \delta_{\alpha\beta}\\
\left[ b,b^\dagger\right] &= 1\\
\left[ b,\Psi_\alpha^\dagger\right] &=\left[ b,\Psi_\alpha\right] = 0\\
\end{align*}

From this transformation it is worth noticing:

\begin{align*}
b^\dagger b &= \frac{1}{3}\roundP{\sum_i a_i^\dagger}\roundP{\sum_j a_j} =  \frac{1}{6}\roundP{\sum_i x_i-iy_i}\roundP{\sum_j x_j+iy_j}\\
&= \frac{1}{6}\sum_i\sum_j(x_i-iy_i)(x_j-iy_j) = \frac{1}{6}\sum_i\sum_jx_ix_j+y_iy_j -i\cancelto{i\delta_{ij}}{[xi,y_j]}\\
&= L + \frac{1}{2}\\
\end{align*}
\small
\begin{align*}
\Psi^\dagger\Psi &= \Psi_1^\dagger\Psi_1 +\Psi_2^\dagger\Psi_2\\
&= \frac{1}{2} (a_2^\dagger-a_1^\dagger)(a_2-a_1) + \frac{1}{6}(a_2^\dagger+a_1^\dagger-2a_3^\dagger)(a_2+a_1-2a_3)\\
&= \frac{1}{2} (x_2-x_1 -i(y_2-y_1))(x_2-x_1 +i(y_2-y_1))\\
&+\frac{1}{6}(x_2+x_1-2x_3-i(y_2+y_1-2y_3))(x_2+x_1-2x_3 +i(y_2+y_1-2y_3))\\
&= \frac{1}{2} (x_2-x_1)^2+(y_2-y_1)^2-i\cancelto{2}{[x_2-x_1,y_2-y_1]} \\
&+ \frac{1}{6} (x_2+x_1-2x_3)^2+(y_2+y_1-2y_3)^2-i\cancelto{3}{[x_2+x_1-2x_3,y_2+y_1-2y_3]}\\
&= S+1
\end{align*}
\normalsize

This tells us that the eigenvalues from operator $b$ are semi-integers, while those of $\Psi$ are quantized integers.\\

So far we have reduced the problem to the study of the center of mass and relative coordinates spinor. In this terms, the Hamiltonian for these variables will be given by:\\

\begin{align*}
H_{gc} &= V\roundP{\Psi,\Psi^\dagger}+ \omega (S+L)\\
&= V\roundP{\Psi,\Psi^\dagger} + \omega{\Psi^\dagger\Psi + b^\dagger b -\frac{3}{2}}\\
&= V\roundP{\Psi,\Psi^\dagger} + \omega{\Psi_1^\dagger\Psi_1 +\Psi_2^\dagger\Psi_2+ b^\dagger b -\frac{3}{2}}
\end{align*}

Ignoring the effects of the center of mass $b$, what we have here is the Hamiltonian of two oscillators $(\Psi_1,\Psi_2)$ coupled by the potential $V$ which does not affect the center of the triangle. What we will see next is that in fact, the two oscillators from $\Psi$ induce a Schwinger angular momentum \cite{Schwinger}.\\

In quantum mechanics the symmetries of a system play a substantial role in the solution of the associated problem. In fact, the role of the symmetries in quantum mechanics is more important than in classical mechanics \cite{qfhebook}. In quantum mechanics they determine the algebra of operators, and consequently, the eigenvectors and eigenvalues that generate the states of the Hilbert space. In this case, we know that the potential $V$ follows the symmetries associated with the special Euclidean group in 2D ($SE(2)$), that, given the simplification of the problem and the canonical spinor variables, is transformed into $SU(2)$.\\

The role of this symmetry can be understood as the operators $\Psi_\alpha$ implement a Schwinger oscillator \cite{Schwinger} associated with this symmetry.\\

\section{The Schwinger oscillator and the angular momentum representation}
To understand better the role of the symmetry of $\Psi$ in the problem, let us explain the theory associated with the Schwinger oscillators. First, let us consider two decoupled harmonic oscillators with the same frequency $\omega$. The Hamiltonian of this system in terms of the annihilation-creation operators $(a_1,a_2)$ of each oscillator, will be given by:

\begin{equation*}
H_{SO} = \omega\roundP{a_1^\dagger a_1 + a_2^\dagger a_2 + 1}
\end{equation*}

As the oscillators are decoupled the commutation relations between the operators $a_i$ are given by:

\begin{align*}
\left[ a_i,a_j^\dagger\right] &= \delta_{ij}\\
\left[ a_i,a_j\right] &= 0\\
\end{align*}
 
Now we are going to show that this algebra of annihilation operators induces an angular momentum algebra with symmetry $SU(2)$. To do that, let us define an angular momentum in terms of the $\{a_i\}$ and see its commutation relations:

\begin{align*}
J_i &= \frac{1}{2}\sigma^i_{\alpha\beta}a^\dagger_\alpha a_\beta \\
\\
[J_i,J_j] &= \frac{1}{4}\left[ \sigma^i_{\alpha\beta}a^\dagger_\alpha a_\beta,\sigma^j_{\gamma\delta}a^\dagger_\gamma a_\delta \right] = \frac{1}{4}\sigma^i_{\alpha\beta}\sigma^j_{\gamma\delta}\left[ a^\dagger_\alpha a_\beta,a^\dagger_\gamma a_\delta \right]\\
&= \frac{1}{4}\sigma^i_{\alpha\beta}\sigma^j_{\gamma\delta}\roundP{a^\dagger_\alpha [a_\beta,a^\dagger_\gamma a_\delta]+  [a^\dagger_\alpha,a^\dagger_\gamma a_\delta]a_\beta}\\
&= \frac{1}{4}\sigma^i_{\alpha\beta}\sigma^j_{\gamma\delta}\roundP{a^\dagger_\alpha [a_\beta,a^\dagger_\gamma ]a_\delta +  a^\dagger_\gamma[a^\dagger_\alpha, a_\delta]a_\beta }\\
&= \frac{1}{4}\roundP{\sigma^i_{\alpha\beta}\sigma^j_{\beta\delta}a^\dagger_\alpha a_\delta - \sigma^i_{\alpha\beta}\sigma^j_{\gamma\alpha}a^\dagger_\gamma a_\beta }\\
&= \frac{1}{4}a^\dagger_\alpha a_\beta \left[ \sigma^i,\sigma_j \right]_{\alpha\beta}\\
&= i\epsilon_{ijk} \frac{1}{4} \sigma^k_{\alpha\beta}a^\dagger_\alpha a_\beta \\
\\
[J_i,J_j] &= i\epsilon_{ijk}J_k
\end{align*}

This representation of angular momentum is no different from the usual one. In fact, one can take annihilation-creation operators $J^\pm = J_1\pmiJ_2$ as well as the total angular momentum $J^2$ to deduce the rules for the quantum numbers $(j,m)$. Let us calculate $J^2$ to completely characterise this angular momentum algebra:





