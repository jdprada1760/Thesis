\chapter{Introduction}

In this chapter a classic approach of a somehow general case of the three body problem in 2 dimensions is going to be presented. This will give some necessary intuition to develop the analogous problem in the spherical geometry. To begin with, the problem is going to be described in great detail; then its integrability is going to be proven; and finally, a formalism to describe the movement of the particles is going to be presented.\\

\section{The three body problem in the plane}

The three body problem presented here is that of three particles of electrical charge $-e$ and mass $m$ confined to a plane, under the influence of a strong magnetic field perpendicular to it, and forces whose potentials satisfy translational and rotational symmetries in the plane.\\

Given this information, the Hamitonian associated with this system has the form:

\begin{equation}
H = \sum_{i=1}^{3} \frac{1}{2m_i} \norm{ \vec{p_i} + 
e\vec{ A } \left( \vec{q_i} \right)}^2
+ V \roundP{ \vec{q_1},\vec{q_2},\vec{q_3} }
+\frac{\omega_c^2}{2}\sum_{i=1}^{3} \norm{\vec{q_i}}^2
\label{eq:ham2d}
\end{equation}

Where $\vec{q_i} = x_i \hat{\imath} + y_i \hat{\jmath}$, $\vec{p_i} = {p_x}_i\hat{\imath} + {p_y}_i\hat{\jmath}$ and $\vec{A}\roundP{\vec{q}}$ is the magnetic vector potential, which satisfies $\nabla \times \vec{A} = B\hat{k}$.\\

Besides, the potential $V \roundP{ \vec{q_1},\vec{q_2},\vec{q_3} }$ satisfies the symmetries:

\begin{equation}
V \roundP{ R\vec{q_1}+\vec{a},R\vec{q_2}+\vec{a},R\vec{q_3}+\vec{a}  }= V \roundP{ \vec{q_1},\vec{q_2},\vec{q_3} }
\end{equation}

For any rotation $R$ and translation $\vec{a}$ in the plane.\\

\subsection{Integrability of the system}

For the proof of integrability for this system, and for further analysis of the trajectories of the particles, let us perform the well known canonical transformation of the guiding centres.\\

This transformation is defined by the following two equations:

\begin{equation}
\vec{\pi_i} = \vec{p_i} + e\vec{A}\roundP{\vec{q_i}}
\label{eq:ct2d1}
\end{equation}

\begin{equation}
\vec{R_i} = \vec{q_i} + \frac{\hat{k}\times\vec{\pi_i}}{eB}
\label{eq:ct2d2}
\end{equation}

The equation \ref{eq:ct2d1} passes from the canonical momenta $\vec{p_i}$ to the linear momenta $\vec{\pi_i}$, which is much more intuitive and understandable; while the equation \ref{eq:ct2d2} passes the general position  $\vec{q_i}$ to the position of the instantaneous guiding centre $\vec{R_i}$.\\
% Sera carreta?

In a system without the interaction potentials, the electrically charged particles are known to perform circular motion (cyclotron) with radius depending on the initial linear momentum. In this case, the guiding centres would be constant in time as would be the linear momenta. However, with the introduction of an interacting potential, the momenta of each particle may vary, making the guiding centres change too, making it necessary the interpretation of the guiding centres as the centres of the cyclotron that the particles would describe given no interactions, and the instantaneous linear momenta.\\



\section{Another Section}

Phasellus nisi quam, volutpat non ullamcorper eget, congue fringilla leo. Cras et erat et nibh placerat commodo id ornare est. Nulla facilisi. Aenean pulvinar scelerisque eros eget interdum. Nunc pulvinar magna ut felis varius in hendrerit dolor accumsan. Nunc pellentesque magna quis magna bibendum non laoreet erat tincidunt. Nulla facilisi.

Duis eget massa sem, gravida interdum ipsum. Nulla nunc nisl, hendrerit sit amet commodo vel, varius id tellus. Lorem ipsum dolor sit amet, consectetur adipiscing elit. Nunc ac dolor est. Suspendisse ultrices tincidunt metus eget accumsan. Nullam facilisis, justo vitae convallis sollicitudin, eros augue malesuada metus, nec sagittis diam nibh ut sapien. Duis blandit lectus vitae lorem aliquam nec euismod nisi volutpat. Vestibulum ornare dictum tortor, at faucibus justo tempor non. Nulla facilisi. Cras non massa nunc, eget euismod purus. Nunc metus ipsum, euismod a consectetur vel, hendrerit nec nunc.