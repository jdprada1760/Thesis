\chapter{Introduction}
The $N$ body problem is a highly known and studied issue in Physics. It consists in researching the trajectories that $N$ point masses would follow when interacting with external and internal forces with certain defined characteristics, given all the information of the initial conditions.\\

At first, the principal interest in the study of this problem was the exact prediction of the path of celestial bodies. However, as the problem was known more, it was understood that its study is of great importance not only for astrophysics but for the theoretical comprehension of classical mechanics. Great minds of physics and mathematics have worked in the restricted problem of three bodies, as Poincar\'e \cite{introPoincare} and Jacobi \cite{introJacobi}. It was this way that Poincar\'e, in an attempt of solving the three body problem, discovered that it is not integrable in general, and formulated the bases of what is known nowadays as chaos theory \cite{introPoincare}.\\

Independently of the formalism chosen to define the system, the N body problem is reduced to the integration of the equations of motion for the N particles. As this problem has been known to be non-integrable for the $N\geq 3$ cases, with the exception of few occurrences that involve forces with strange features as explicit dependence of the position an velocities \cite{strangeCases}, the study of realistic 3 body problems is a very interesting question in physics.\\

\section{Motivation}
One of the particular realistic cases of the three body problem that is known to be integrable, is the system of three particles on the plane with mass $m$ and charge $e$ under the influence of a huge constant magnetic field perpendicular to the surface and forces whose potentials are invariant under rotations and translations in the plane \cite{alonso}.\\

This problem is integrable by virtue of the action of the big magnetic potential, which decouples the movement of  the particle into two degrees of freedom known as the guiding centres and the linear momenta. 

This problem includes the general case of central potentials and can be taken as a model for non-relativistic electrons

\section{State of the art}
