\chapter{Introduction}
The $N$ body problem is a highly known and studied issue in Physics. It consists in researching the trajectories that $N$ point masses would follow when interacting with external and internal forces with certain defined characteristics, given all the information of the initial conditions.\\

At first, the principal interest in the study of this problem was the exact prediction of the path of celestial bodies. However, as the problem was known more, it was understood that its study is of great importance not only for astrophysics but for the theoretical comprehension of classical mechanics. Great minds of physics and mathematics have worked in the restricted problem of three bodies, as Poincar\'e \cite{introPoincare} and Jacobi \cite{introJacobi}. It was this way that Poincar\'e, in an attempt of solving the three body problem, discovered that it is not integrable in general, and formulated the bases of what is known nowadays as chaos theory \cite{introPoincare}.\\

Independently of the formalism chosen to define the system, the N body problem is reduced to the integration of the equations of motion for the N particles. As this problem has been known to be non-integrable for the $N\geq 3$ cases, with the exception of few occurrences that involve forces with strange features as explicit dependence of the position an velocities \cite{strangeCases}, the study of realistic 3 body problems is a very interesting question in physics.\\

\section{Motivation}
One of the particular realistic cases of the three body problem that is known to be integrable, is the system of three particles on the plane with mass $m$ and charge $e$ under the influence of a huge constant magnetic field perpendicular to the surface and forces whose potentials are invariant under rotations and translations in the plane \cite{alonso}.\\

This problem is integrable by virtue of the action of the big magnetic potential, which decouples the movement of  the particle into two degrees of freedom known as the guiding centres and the linear momenta. The analysis carried out in \cite{alonso} can be extended to the quantum formalism and it explains the quantum Hall effect (QHE) fact that particles at low temperatures are confined to the base state to energy due to Landau level gap modulated by the magnetic field.\\

The study of this problem may be applied to classical and quantum Hall effect models. These models are principally centred in the analysis of the bulk properties, and are not interested in the effect of border effects. Consequently, these models are usually defined over geometries with no borders as the infinite plane, the torus and the sphere \cite{schlief}. Having this into account, the principal objective of this work is to try to expand the analysis of the three body problem on the plane carried out in \cite{alonso} to the spherical geometry.\\

To do so, as a first approach to the problem, we are going reproduce the calculations and analysis from \cite{alonso} of the classical system in great detail. Then, we are going to study some important aspects of the classical one body problem on the sphere under the influence of a magnetic monopole to obtain some intuition about the important quantities and the analogies of the movement of the particle in this case and the one on the plane. After that, we are going to proceed with the analysis of the quantum three body problem in the plane. Then some intuition is going to be presented about the quantum one body problem in the sphere, known as Haldane's formalism \cite{haldane}.\\
