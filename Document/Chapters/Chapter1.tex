\chapter{Introduction}
The $N$-body problem is a highly-known and studied issue in Physics. It consists on researching the trajectories that $N$ point masses would follow when interacting with external and internal forces with certain defined characteristics, given all the information of the initial conditions.\\

At first, the principal interest in the study of this problem was the exact prediction of the path of celestial bodies. However, as the problem was known more, it was understood that its study is of great importance not only for astrophysics but for the theoretical comprehension of classical mechanics. Great minds of physics and mathematics have worked in the restricted problem of three bodies, as Poincar\'e \cite{introPoincare} and Jacobi \cite{introJacobi}. It was this way that Poincar\'e, in an attempt of solving the three body problem, discovered that it is not integrable in general, and formulated the bases of what is known nowadays as chaos theory \cite{introPoincare}.\\

Independently of the formalism chosen to define the system, the N-body problem is reduced to the integration of the equations of motion for the $N$ particles. As this problem has been known to be non-integrable for the $N\geq 3$ cases, with the exception of few occurrences that involve forces with strange features as explicit dependence of the position an velocities \cite{strangeCases}, the study of realistic 3 body problems is a very interesting question in physics.\\

Regarding quantum mechanics, there are many puzzling aspects about its fundamentals, as the role that symmetries play in the quantization of states. In this case, the analysis of N-body problems may be useful to get some intuition about the meaning and cause of those perplexing facts, given the strong correlation between classical and quantum mechanics via the Hamiltonian formalism.

\section{Motivation}
One of the particular realistic cases of the three body problem that is known to be integrable, is the system of three particles on the plane with mass $m$ and charge $e$ under the influence of a large constant magnetic field perpendicular to the surface and forces whose potentials are invariant under rotations and translations in the plane \cite{alonso}.\\

This problem is integrable by virtue of the action of the big magnetic potential, which decouples the movement of  the particle into two degrees of freedom known as guiding centres and linear momenta. Further analysis of the movement of the particles can be performed in terms of a canonical transformation that encodes the shape of the generated triangle into a Bloch sphere.\\

The analysis carried out in \cite{alonso} can be extended to the quantum formalism via the canonical quantization rules \cite{Cq}, where the role of the $SU(2)$ symmetry becomes clear with the implementation of a Schwinger angular momentum associated with the Bloch sphere variables. The classical decoupling of the guiding centres degrees of freedom is associated with the fact that, in the quantum formalism, electrons are confined to the ground state of energy due to the Landau level gap modulated by the magnetic field, an effect that is at the basis of the Quantum Fractional Hall Effect (QFHE). The clear relation between both quantum and classical formalisms, together with the fact that the spherical QFHE shows similar behaviour than the planar case on the large magnetic field regimen, tempts us to intend a generalization of the procedure for the spherical geometry.  \\

To do so, as a first approach to the problem, we are going reproduce the calculations and analysis from \cite{alonso} for the classical and quantum system in great detail. Then, we are going to study some important aspects of the classical one-body problem on the sphere under the influence of a magnetic monopole to obtain some intuition about the analogies of the movement of the particle in this case. Then, an analogue formalism to study the motion of the three-body problem in the sphere is going to be intended, giving some insights as why the translation of the analysis from the plane to the sphere is not trivial.
