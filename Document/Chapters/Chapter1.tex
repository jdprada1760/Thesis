\chapter{Introduction}

In this chapter a classic approach of a somehow general case of the three body problem in 2 dimensions is going to be presented. This will give some necessary intuition to develop the analogous problem in the spherical geometry. To begin with, the problem is going to be described in great detail; then its integrability is going to be proven; and finally, a formalism to describe the movement of the particles is going to be presented.\\

\section{The three body problem in the plane}

The three body problem presented here is that of three particles of electrical charge $e$ and mass $m$ confined to a plane, under the influence of a strong magnetic field perpendicular to it, and forces whose potentials satisfy translational and rotational symmetries in the plane.\\

Given this information, the Hamitonian associated with this system has the form:

\begin{equation}
H = \sum_{i=1}^{3} \frac{1}{2m_i} \norm{ \vec{p_i} - 
e\vec{ A } \left( \vec{q_i} \right)}^2
+ V \roundP{ \vec{q_1},\vec{q_2},\vec{q_3} }
+\frac{\omega_c^2}{2}\sum_{i=1}^{3} \norm{\vec{q_i}}^2
\label{eq:ham2d}
\end{equation}

Where $\vec{q_i} = x_i \hat{\imath} + y_i \hat{\jmath}$, $\vec{p_i} = {p_x}_i\hat{\imath} + {p_y}_i\hat{\jmath}$ and $\vec{A}\roundP{\vec{q}}$ is the magnetic vector potential, which satisfies $\nabla \times \vec{A} = B\hat{k}$.\\

Besides, the potential $V \roundP{ \vec{q_1},\vec{q_2},\vec{q_3} }$ satisfies the symmetries:

\begin{equation}
V \roundP{ R\vec{q_1}+\vec{a},R\vec{q_2}+\vec{a},R\vec{q_3}+\vec{a}  }= V \roundP{ \vec{q_1},\vec{q_2},\vec{q_3} }
\end{equation}

For any rotation $R$ and translation $\vec{a}$ in the plane.\\

\subsection{The canonical transformation of the guiding centres}

For the proof of integrability for this system, and for further analysis of the trajectories of the particles, let us perform the well known transformation of the guiding centres.\\

This transformation is defined by the following two equations:

\begin{equation}
\vec{\pi_i} = \vec{p_i} - e\vec{A}\roundP{\vec{q_i}}
\label{eq:ct2d1}
\end{equation}

\begin{equation}
\vec{R_i} = \vec{q_i} - \frac{\hat{k}\times\vec{\pi_i}}{eB}
\label{eq:ct2d2}
\end{equation}

The equation \ref{eq:ct2d1} passes from the canonical momentum $\vec{p_i}$ to the linear momentum $\vec{\pi_i}$, which is much more intuitive and understandable; while the equation \ref{eq:ct2d2} transforms the general position  $\vec{q_i}$ to the position of the instantaneous guiding centre $\vec{R_i}$.\\
% Sera carreta?

In a system without the interaction potentials, the electrically charged particles are known to perform the circular motion of the cyclotron with radii that depends on the initial linear momenta. In this case, the guiding centres would be constant in time as would be the linear momenta. However, with the introduction of an interacting potential, the momentum of each particle may vary making the guiding centre change too, which is why the instantaneous interpretation of the guiding centres is necessary.\\

Now, let us calculate the Poisson brackets for this new set of coordinates in a specific particle.

\begin{align*}
\poisson{\pi_1}{\pi_2} &= \frac{\partial \pi_1}{\partial q_{\alpha}}\frac{\partial \pi_2}{\partial p_{\alpha}} - \frac{\partial \pi_2}{\partial q_{\alpha}}\frac{\partial \pi_1}{\partial p_{\alpha}}\\
&= -e\delta_{\alpha 2}\frac{\partial A_1}{\partial q_{\alpha}} +e\delta_{\alpha 1}\frac{\partial A_2}{\partial q_{\alpha}}\\
&= -e\frac{\partial A_1}{\partial q_{2}}  + e\frac{\partial A_2}{\partial q_{1}}\\
&= e(\nabla \times \vec{A})_3 = eB
\end{align*}
\begin{align*}
\poisson{R_1}{R_2} &= \poisson{q_1}{q_2} + \poisson{q_1}{-\frac{\pi_1}{eB}} +\poisson{\frac{\pi_2}{eB}}{q_2} + \poisson{\frac{\pi_2}{eB}}{-\frac{\pi_1}{eB}}\\
&= \frac{1}{eB}\roundP{\cancelto{-1}{\poisson{p_1}{q_1}}- \cancelto{0}{e\poisson{A_1}{q_1}}+\cancelto{-1}{\poisson{p_2}{q_2}} \cancelto{0}{e\poisson{A_2}{q_2}}} +\frac{eB}{(eB)^2}\\
&= \frac{-2}{eB}+ \frac{1}{eB} = -(eB)^{-1}
\end{align*}
\begin{align*}
\poisson{R_{1}}{\pi_{2}} &= \poisson{q_1}{\pi_2}+\cancelto{0}{\poisson{\frac{\pi_2}{eB}}{\pi_2}}\\
&= \cancelto{0}{\poisson{q_1}{p_2}}-e\cancelto{0}{\poisson{q_1}{A_2}}\\
&= \poisson{R_2}{\pi_1} = 0
\end{align*}

This Poisson brackets can be generalised to the transformation for the three particles. Taking $i,j = \poisson{1,2}{3}$ and $\alpha,\beta=\poisson{1}{2}$:

\begin{equation}
\poisson{\pi_{i,\alpha}}{\pi_{j,\beta}}=\roundP{eB}\delta_{ij}\epsilon_{\alpha \beta}  
\label{eq:pb1}
\end{equation}

\begin{equation}
\poisson{R_{i,\alpha}}{R_{j,\beta}}= -\roundP{eB}^{-1} \delta_{ij}\epsilon_{\alpha \beta}  
\label{eq:pb2}
\end{equation}

\begin{equation}
\poisson{R_{i,\alpha}}{\pi_{j,\beta}}=0 
\label{eq:pb3} 
\end{equation}

Equations \ref{eq:pb1,eq:pb2,eq:pb3} allow us to identify the proposed transformation as canonical. However, this is not the usual canonical transformation where the position coordinates and the momentum coordinates are canonical conjugates. In this special case, one component of the momentum is canonical conjugate with the other momentum coordinate, and equally for the position coordinates.\\

Now, with a huge magnetic field, if the potential of the interaction forces does not vary abruptly in space, we can use the approximation $\vec{R}_i \approx \vec{q}_i$ to average the potentials over the guiding centres, that is, we can replace $\vec{q}_i$ for $\vec{R}_i$ in  $V \roundP{ \vec{q_1},\vec{q_2},\vec{q_3} }$.\\

We can interpret the last approximation as follows: In the cyclotron problem, the radius of the circular motion described is proportional to the linear momentum and inversely proportional to the magnetic field. Then, in the presence of a big $B$, the radius of the cyclotron would shrink to a very small size. In the case we are working, the radia of the instantaneous cyclotron motion would be proportional to $\norm{(\hat{k}\times \vec{\pi_i})(eB)^{-1}}$, then, in a presence of a huge magnetic field, the instantaneous radii of the particles would be small enough that the potential would not sense the instantaneous circular motion and would sense the particle as if it where always in the position of its guiding centre.\\


\section{Another Section}

Phasellus nisi quam, volutpat non ullamcorper eget, congue fringilla leo. Cras et erat et nibh placerat commodo id ornare est. Nulla facilisi. Aenean pulvinar scelerisque eros eget interdum. Nunc pulvinar magna ut felis varius in hendrerit dolor accumsan. Nunc pellentesque magna quis magna bibendum non laoreet erat tincidunt. Nulla facilisi.

Duis eget massa sem, gravida interdum ipsum. Nulla nunc nisl, hendrerit sit amet commodo vel, varius id tellus. Lorem ipsum dolor sit amet, consectetur adipiscing elit. Nunc ac dolor est. Suspendisse ultrices tincidunt metus eget accumsan. Nullam facilisis, justo vitae convallis sollicitudin, eros augue malesuada metus, nec sagittis diam nibh ut sapien. Duis blandit lectus vitae lorem aliquam nec euismod nisi volutpat. Vestibulum ornare dictum tortor, at faucibus justo tempor non. Nulla facilisi. Cras non massa nunc, eget euismod purus. Nunc metus ipsum, euismod a consectetur vel, hendrerit nec nunc.