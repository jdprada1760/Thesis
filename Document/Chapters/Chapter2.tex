\chapter{The three body problem in the plane}

In this chapter a classic approach of a somehow general case of the three body problem in 2 dimensions is going to be presented. This problem was developed by Alonso Botero et. al in \cite{alonso}. It will give some necessary intuition to develop the analogous problem in the spherical geometry. To begin with, the problem is going to be described in great detail; then its integrability is going to be proven; and finally, a brief description of the movement of the particles is going to be presented.\\

\section{The definition of the problem}

The three body problem presented here is that of three particles of electrical charge $e$ and mass $m$ confined to a plane, under the influence of a strong magnetic field perpendicular to it and forces whose potentials satisfy translational and rotational symmetries in the plane. To make the problem more interesting, we consider a harmonic potential with frequency $\omega$\\

Given this information, the Hamiltonian associated with this system has the form:

\begin{equation}
H = \sum_{i=1}^{3} \frac{1}{2m} \norm{ \vec{p_i} - 
e\vec{ A } \left( \vec{r_i} \right)}^2
+ V \roundP{ \vec{r_1},\vec{r_2},\vec{r_3} }
+\frac{m\omega}{2}\sum_{i=1}^{3} \norm{\vec{r_i}}^2
\label{eq:ham2d}
\end{equation}

Where $\vec{r_i} = x_i \hat{\imath} + y_i \hat{\jmath}$, $\vec{p_i} = {p_x}_i\hat{\imath} + {p_y}_i\hat{\jmath}$ and $\vec{A}\roundP{\vec{r}}$ is the magnetic vector potential on its symmetric gauge, which satisfies $\nabla \times \vec{A} = B\hat{k}$.\\

Besides, the potential $V \roundP{ \vec{r_1},\vec{r_2},\vec{r_3} }$ satisfies the symmetries:

\begin{equation}
V \roundP{ R\vec{r_1}+\vec{a},R\vec{r_2}+\vec{a},R\vec{r_3}+\vec{a}  }= V \roundP{ \vec{r_1},\vec{r_2},\vec{r_3} }
\label{eq:vsym}
\end{equation}

For any rotation $R$ about $\hat{k}$ and any translation $\vec{a}$ in the plane.

Here it is important to note that a magnetic field that is perpendicular to the plane will have no effect on the dynamics of the system. In fact, one can choose a convenient gauge in which the vector potential has the only non-null component along the $z$ axis. This way, given the restriction $\dot{z} = 0$, there will be no change in the Lagrangian function.\\

\section{The canonical transformation of the guiding centres}

For the proof of integrability for this system, and for further analysis of the trajectories of the particles, let us perform the well known transformation of the guiding centres.\\

This transformation is defined by the following two equations:

\begin{equation}
\vec{\pi_i} = \vec{p_i} - e\vec{A}\roundP{\vec{q_i}}
\label{eq:ct2d1}
\end{equation}

\begin{equation}
\vec{R_i} = \vec{r_i} - \frac{\hat{k}\times\vec{\pi_i}}{eB}
\label{eq:ct2d2}
\end{equation}

The equation \eqref{eq:ct2d1} passes from the canonical momentum $\vec{p_i}$ to the linear momentum $\vec{\pi_i}$, which is much more intuitive and understandable; while the equation \eqref{eq:ct2d2} transforms the general position  $\vec{r_i}$ to the position of the instantaneous guiding centre $\vec{R_i}$.\\

In a system without the interaction potentials, the electrically charged particles are known to perform the circular motion of the cyclotron with radii that depends on the initial linear momenta. In this case, the guiding centres would be constant in time as would be the linear momenta. However, with the introduction of an interacting potential, the momentum of each particle may vary making the guiding centre change too, which is why the instantaneous interpretation of the guiding centres is necessary.\\

Now, let us calculate the Poisson brackets for this new set of coordinates for a specific particle.

\begin{align*}
\poisson{\pi_1}{\pi_2} &= \frac{\partial \pi_1}{\partial r_{\alpha}}\frac{\partial \pi_2}{\partial p_{\alpha}} - \frac{\partial \pi_2}{\partial r_{\alpha}}\frac{\partial \pi_1}{\partial p_{\alpha}}\\
&= -e\delta_{\alpha 2}\frac{\partial A_1}{\partial r_{\alpha}} +e\delta_{\alpha 1}\frac{\partial A_2}{\partial r_{\alpha}}\\
&= -e\frac{\partial A_1}{\partial y}  + e\frac{\partial A_2}{\partial x}\\
&= e(\nabla \times \vec{A})_3 = eB
\end{align*}
\begin{align*}
\poisson{R_1}{R_2} &= \poisson{r_1}{r_2} + \poisson{r_1}{-\frac{\pi_1}{eB}} +\poisson{\frac{\pi_2}{eB}}{r_2} + \poisson{\frac{\pi_2}{eB}}{-\frac{\pi_1}{eB}}\\
&= \frac{1}{eB}\roundP{\cancelto{-1}{\poisson{p_1}{r_1}}- \cancelto{0}{e\poisson{A_1}{r_1}}+\cancelto{-1}{\poisson{p_2}{r_2}} \cancelto{0}{e\poisson{A_2}{r_2}}} +\frac{eB}{(eB)^2}\\
&= \frac{-2}{eB}+ \frac{1}{eB} = -(eB)^{-1}
\end{align*}
\begin{align*}
\poisson{R_{1}}{\pi_{2}} &= \poisson{r_1}{\pi_2}+\cancelto{0}{\poisson{\frac{\pi_2}{eB}}{\pi_2}}\\
&= \cancelto{0}{\poisson{r_1}{p_2}}-e\cancelto{0}{\poisson{r_1}{A_2}}\\
&= \poisson{R_2}{\pi_1} = 0
\end{align*}

This Poisson brackets can be generalised to the transformation for the three particles. Taking $i,j = \poisson{1,2}{3}$ and $\alpha,\beta=\poisson{1}{2}$:

\begin{equation}
\poisson{\pi_{i,\alpha}}{\pi_{j,\beta}}=\roundP{eB}\delta_{ij}\epsilon_{\alpha \beta}  
\label{eq:pb1}
\end{equation}

\begin{equation}
\poisson{R_{i,\alpha}}{R_{j,\beta}}= -\roundP{eB}^{-1} \delta_{ij}\epsilon_{\alpha \beta}  
\label{eq:pb2}
\end{equation}

\begin{equation}
\poisson{R_{i,\alpha}}{\pi_{j,\beta}}=0 
\label{eq:pb3} 
\end{equation}

Equations \eqref{eq:pb1}-\eqref{eq:pb2} allow us to identify the proposed transformation as canonical. However, this is not the usual canonical transformation where the position coordinates and the momentum coordinates are canonical conjugates. In this special case, one component of the momentum is canonical conjugate with the other momentum coordinate, and similarly for the guiding center position components.\\

Now, with a huge magnetic field, if the potential of the interaction forces does not vary abruptly in space, we can use the approximation $\vec{R}_i \approx \vec{q}_i$ to average the potentials over the guiding centres, that is, we can replace $\vec{q}_i$ for $\vec{R}_i$ in  $V \roundP{ \vec{q_1},\vec{q_2},\vec{q_3} }$.\\

We can support the last approximation as follows: In the cyclotron problem, the radius of the circular motion described is proportional to the linear momentum and inversely proportional to the magnetic field. Then, in the presence of a big magnetic field $B$, the radius of the cyclotron would shrink to a very small size. Regarding the case we are working with, the radii of the instantaneous cyclotron motion would be proportional to $\norm{(\hat{k} \times \vec{\pi_i})(eB)^{-1}} = \norm{\vec{\pi_i}(eB)^{-1}}$ and its frequency to $\sqrt{B}$. As the potential $V$ does not vary abruptly in the radii scale, the averaging of this motion over the guiding centres means that this potential does not sense that circular motion. Moreover, given the big frequency of the cyclotrons and the scale of variance of the potential, the scale of time of the local circular motions is far smaller than that of the motion of the guiding centres. Therefore, we can ignore the instantaneous quality of the circular motion, and take it as constant in a scale of time small enough for the motion of the guiding centres. In this sense we say that the coordinates for the guiding centres decouple from that of the linear momenta of the particles.\\

Before replacing the new set of coordinates in the Hamiltonian, it is necessary to do a scale transformation to obtain the proper Poisson brackets for the formal definition of canonical transformation, that is:

\begin{align*}
\vec{\pi}_i & \rightarrow (eB)^{-1/2} \vec{ \pi}_i\\
\vec{R}_i & \rightarrow \sqrt{eB} \vec{R}_i
\end{align*}

With this consideration, the Hamiltonian of the system in the new set of rescaled coordinates is given by:

\begin{equation}
H = \sum_{i=1}^{3} \frac{eB}{2m} \norm{ \vec{\pi}_i}^2
+ V\roundP{ (eB)^{-1/2}\vec{R_1},(eB)^{-1/2}\vec{R_2}, (eB)^{-1/2}\vec{R_3} }
+\frac{m\omega}{2eB}\sum_{i=1}^{3} \norm{\vec{R}_i}^2
\label{eq:newham}
\end{equation}

This Hamiltonian, given equation \eqref{eq:pb3} can be decomposed in a Hamiltonian that describes the movement of the guiding centres, and other that describes de movement of the linear momenta. In one hand, the Hamiltonian for the linear momenta is easily identified with the harmonic oscillator, whereas the one that characterises the movement of the guiding centres needs a deeper analysis.

\section{Integrability of the system}
As the Hamiltonian describing the trajectories of the linear momenta of the particles is that of an harmonic oscillator, this part of the problem is integrable and its solutions are widely known. The guiding centre Hamiltonian, in turn, needs to be analysed more deeply. For this purpose, let us take the following convention:\\


\begin{equation}
H_{gc} = \frac{\omega}{2} \sum_{i=1}^{3} \norm{\bar{x}^2} + \norm{\bar{y}^2}
+ V\roundP{\bar{x},\bar{y}}
\label{eq:hamgc}
\end{equation}

Where $\bar{x} = \roundP{x_1,x_2,x_3}$ and $\bar{y} = \roundP{y_1,y_2,y_3}$, being $x_i,y_i$ the rescaled coordinates of the guiding centres of the particles. For simplicity, the potential $V$ and the frequency $\omega$ have been rescaled to take into account the scale transform of the coordinates and maintain the original form of the Hamiltonian:

\begin{align*}
\frac{eB\omega}{m} &\rightarrow \omega\\
 V\roundP{\frac{\bar{x}}{\sqrt{eB}},\frac{\bar{y}}{\sqrt{eB}}} &\rightarrow V\roundP{\bar{x},\bar{y}}
\end{align*}

Clearly, the scaled potential $V$ still has the symmetries expressed in the equation \eqref{eq:vsym}. Furthermore, with this scale transformation the Poisson brackets take the form:

\begin{equation}
\poisson{y_i}{x_j} = \delta_{ij}
\label{lastpb}
\end{equation}

Now that the guiding center Hamiltonian has been expressed in terms of the proper canonical set of coordinates, the fastest way to prove the integrability of the system is via the Liouville-Arnol'd theorem \cite[Sect. 49]{arnold}. For this theorem, it is only necessary to find 2 more independent integrals in involution (besides the Hamiltonian).\\

To get this 2 integrals, let us exploit the symmetries of the guiding centres Hamiltonian. We then define the generators of translations and rotation in the plane, which are symmetries of the potential:

\begin{equation}
\begin{aligned}
T_x &= \sum_{i=1}^{3} x_i \\
T_y &= \sum_{i=1}^{3} y_i 
\end{aligned}
\label{eq:gentranslation}
\end{equation}


\begin{equation}
R_z = \frac{1}{2} \sum_{i=1}^{3} \roundP{x_i^2 + y_i^2}
\label{eq:genrotation}
\end{equation}

It is easily verifiable that these are indeed the symmetries generators. To see that, take the first order infinitesimal transformations of translations and rotations on the plane:

\begin{equation*}
x_i  \rightarrow x_i + \epsilon 
\end{equation*}

\begin{equation*}
y_i \rightarrow y_i + \epsilon 
\end{equation*}

\begin{equation*}
\roundP{x_i,y_i} \rightarrow \roundP{x_i +\epsilon y_i, y_i - \epsilon x_i}
\end{equation*}

Now note that for the infinitesimal translations, the potential of the transformed coordinates is related to the potential of the normal coordinates by a directional derivative, which can be identified with the Poisson bracket of the potential $V$ and each generator:

\begin{align*}
0={V\roundP{\bar{x}+\epsilon,\bar{y}} - V\roundP{\bar{x},\bar{y}}} &= \epsilon \sum_{i = 1}^3 \frac{\partial V\roundP{\bar{x},\bar{y}}}{\partial x_i}  = \epsilon \poisson{V}{T_x} = 0\\
0={V\roundP{\bar{x},\bar{y}+\epsilon} - V\roundP{\bar{x},\bar{y}}} &= \epsilon \sum_{i = 1}^3 \frac{\partial V\roundP{\bar{x},\bar{y}}}{\partial y_i} = \epsilon \poisson{T_y}{V} = 0
\end{align*}

For the infinitesimal rotation, the relation is analogous:

\begin{equation*}
0={V\roundP{\bar{x} +\epsilon \bar{y}, \bar{y} - \epsilon \bar{y}}-V\roundP{\bar{x},\bar{y}}}
= \frac{\partial V\roundP{\bar{x},\bar{y}}}{\partial x_i} \roundP{\epsilon y_i} -                                \frac{\partial V\roundP{\bar{x},\bar{y}}}{\partial y_i} \roundP{\epsilon x_i}                                        = \epsilon \poisson{V}{R_z}
\end{equation*}

Therefore, we conclude that the generators of translations and rotations in the plane commute with the potential $V$ due to its symmetries. Besides, the generator of rotations is exactly equal to the harmonic-like part of the guiding center Hamiltonian which validates that $R_z$ is other integral in involution. The generators of translations are not integrals in involution for they do not commute with the harmonic potential. However, we can calculate a quantity in terms of these generators, which already commute with the potential $V$, to make it commute with the remaining part of $H_{gc}$: 

\begin{equation}
L \coloneqq \frac{1}{6}\roundP{T_x^2 + T_y^2}
\label{eq:gccircle}
\end{equation}

This new quantity $L$ clearly commutes with the potential $V$ because the Poisson bracket is  a linear differential operator in one component and it obeys the Leibniz rule. Moreover, it also commutes with the rotation generator $J \coloneqq R_z$:

\begin{align*}
\poisson{T_x^2+ T_y^2}{ J} &= \sum_{i,j,k} \poisson{x_ix_j + y_iy_j}{x_k^2 + y_k^2}\\
&= \sum_{i,j,k}  \poisson{x_ix_j}{y_k^2} + \poisson{y_iy_j}{x_k^2} = \sum_{i,j,k} y_k \poisson{x_ix_j}{y_k} + x_k \poisson{y_iy_j}{x_k}\\
&= \sum_{i,j,k} y_kx_i\delta_{jk}+y_kx_j\delta_{ik} - x_ky_i\delta_{jk}- x_ky_j\delta_{ik}\\
&= 2\sum_{i,j}x_iy_j-x_iy_j = 0
\end{align*}

As we found $L$ as the last integral in involution, we conclude, by the Liouville-Arnol'd theorem, that the subsystem of guiding centres is integrable by quadratures.\\

\section{The spinorial representation and the Bloch sphere mapping}
Taking the motion integrals obtained in the previous section for the guiding center Hamiltonian \eqref{eq:hamgc} the motion of the particles can be broken down in that of the center of mass, and the relative trajectories of the particles. To see that, note that as $T_x^2 + T_y^2$ in \eqref{eq:gccircle} is an integral in involution with the Hamiltonian, it is a conserved quantity. The conservation of this term is clearly interpreted as a circular motion of the center of mass of the system $\roundP{\frac{T_x}{3},\frac{T_y}{3}}$.\\

The latter analysis can be clarified in terms of the decoupling of the guiding center trajectories from the general motion of the particles. In that sense one can deduce that the guiding centre movement is only affected by the interaction potential $V$ and the external harmonic potential. This can be confirmed by the lack of magnetic terms in the guiding centre Hamiltonian in equation \eqref{eq:newham}.\\

Taking this into account, as the potential $V$ is a central potential that produces only internal forces that do not affect the movement of the centre of mass, one can assure that its motion is determined by the only term of external interaction in the Hamiltonian, the harmonic potential.\\

With this, the equations of motion for the center of mass of the system will be given by $\dot{T}_i = \poisson{T_i}{{\omega}J}$. Now, from equations above about the commutation of $L$ and $J$, it is simple to deduce:

\begin{align*}
\poisson{T_x^2}{\omega J} &= 2T_x\poisson{T_x}{J} = 2\omega\sum_{i,j}x_iy_j = -\poisson{T_y^2}{\omega J}\\
&= 2\omega T_xT_y \\
\poisson{T_x}{\omega J} &= \omega T_y\\
\poisson{T_y}{\omega J} &= -\omega T_x
\end{align*}

Which tells us that the movement of $(T_x,T_y)$ mass is besides circular, uniform. The problem is then reduced to the analysis of the coordinates relative to the center of mass.\\

To carry on with this procedure, let us take the spinor of relative position defined as follows:

\begin{align}
\Psi = \frac{1}{2\sqrt{3}}\begin{pmatrix}\sqrt{3}\roundP{z_{2}-z_{1}}\\
z_{2}+z_{1}-2z_{2}\end{pmatrix}
\label{eq:spinor}
\end{align}

With $z_{i} = x_i + iy_i$. If we call define the center of mass as $T = \frac{1}{3}\roundP{T_x + iT_y} = \frac{1}{3}\roundP{z_1 +z_2+z_3}$ we can retrieve the general position of each particle:

\small
\begin{equation}
\begin{aligned}
z_1 &= T+\frac{1}{\sqrt{3}}\Psi_2 - \frac{1}{3}\Psi_1 \\
z_2 &= T+\frac{1}{\sqrt{3}}\Psi_2 + \frac{1}{3}\Psi_1 \\
z_3 &= T-\frac{2}{\sqrt{3}}\Psi_2
\end{aligned}
\label{eq:singlevecs}
\end{equation}
\normalsize

The spinor components were chosen to satisfy the canonical commutation relations:
\begin{align*}
\poisson{\Psi_\alpha}{\Psi_\alpha} &= 0\\
\poisson{\Psi_1}{\Psi_2} &= \poisson{\Psi_1^*}{\Psi_2^*}^* = \poisson{\frac{1}{2}\roundP{z_{2}-z_{1}}}{\frac{1}{2\sqrt{3}}\roundP{z_{2}+z_{1}-2z_{3}}} \\
&= i\poisson{\frac{1}{2}\roundP{x_{2}-x_{1}}}{\frac{1}{2\sqrt{3}}\roundP{y_{2}+y_{1}-2y_{3}}}\\
&+ i\poisson{\frac{1}{2}\roundP{y_{2}-y_{1}}}{\frac{1}{2\sqrt{3}}\roundP{x_{2}+x_{1}-2x_{3}}} = 0\\
\end{align*}
\small
\begin{align*}
\poisson{\Psi_1}{\Psi_2^*} &= \poisson{\Psi_1^*}{\Psi_2}^* = -\frac{i}{2\sqrt{3}}\poisson{x_{2}-x_{1}}{y_{2}+y_{1}-2y_{3}} = 0\\
\poisson{\Psi_1}{\Psi_1^*} &= \frac{-2i}{4}\poisson{x_{2}-x_{1}}{y_2-y_1} = -i\\
\poisson{\Psi_2}{\Psi_2^*} &= \frac{-i}{6}\poisson{x_{1}+x_{2}-2x_3}{y_1+y_2-2y_3} = -i
\end{align*}
\normalsize

Which can be resumed:

\begin{align}
\poisson{\Psi_\alpha}{\Psi_\beta} &= \poisson{\Psi_\alpha^*}{\Psi_\beta^*} = 0\\
\poisson{\Psi_\alpha^*}{\Psi_\beta} &= i\delta_{\alpha,\beta}
\label{eq:canonicalSpinor}
\end{align}

This spinorial representation of the relative coordinates of the particles with respect to the center of mass is analogous to the transformation performed to analyse the two body problem, only adapted to take into account one more particle and one less dimension. This spinor, besides representing a canonical transformation of the guiding centres, has norm equal to the spin quantity $S = J-L$ which is also an integral of motion:

\small
\begin{align*}
\Psi^\dagger\Psi &= \norm{\Psi_1}^2 + \norm{\Psi_2}^2 \\
&= \frac{1}{4}\roundP{\norm{z_1}^2 + \norm{z_2}^2 - 2\operatorname{Re}\roundP{z_1z_2^*}} + \frac{1}{12}\roundP{\norm{z_1}^2 + \norm{z_2}^2+ \norm{z_3}^2 + \operatorname{Re}\roundP{2z_1z_2^*- 4z_1z_3^*- 4z_2z_3^*}}\\
&= \frac{1}{3}\sum_{i=1}^3 x_i^2 + y_i^2 -\frac{1}{3}\sum_{i > j} x_ix_j + y_iy_j\\
&= \frac{1}{2}\sum_{i=1}^3 x_i^2 + y_i^2 - \frac{1}{6}\roundP{T_x^2+T_y^2}\\
&= J-L = S
\end{align*}
\normalsize

This quantity $S$ can be easily interpreted as proportional to the moment of inertia of the triangle rotating about its center of mass, which gives more intuition to the interpretation of $S$ as spin momentum:

\begin{align*}
I &= m\sum_{i=1}^3 \norm{\vec{r}_i - \vec{r}_{cm}}^2 = m\sum_{i=1}^3 \norm{\vec{r}_i}^2 + 3\norm{\vec{r}_{cm}} - 2\vec{r}_{cm}\cdot\sum_{i=1}^3\vec{r}_i =m\roundP{2J -3\norm{\vec{r}_{cm}}}\\
&= m\roundP{2J - \frac{1}{3}\roundP{T_x^2+T_y^2}} = m\roundP{2J-2L} = 2mS
\end{align*}

On the other hand, it is observed that any phase multiplication to the spinor $\Psi$ is equivalent to any rotation about the center of mass or about the origin, as they have the same effect on relative vectors $\vec{r_i}-\vec{r_j}$:

\begin{align*}
e^{i\phi}z &= \roundP{\cos{\phi}+i\sin{\phi}}\roundP{x+iy} = \roundP{x\cos{\phi}-y\sin{\phi}} + i\roundP{y\cos{\phi}+x\sin{\phi}}\\
\Psi &= \frac{1}{2\sqrt{3}}\begin{pmatrix}\sqrt{3}\roundP{z_{2}-z_{1}}\\
z_{2}+z_{1}-2z_{3}\end{pmatrix}=\frac{1}{2\sqrt{3}}\begin{pmatrix}\sqrt{3}\roundP{z_{2}-z_{1}}\\
z_{2}-z_3+ z_{1}-z_{3}\end{pmatrix}\\
e^{i\phi}\Psi &= \frac{1}{2\sqrt{3}}\begin{pmatrix}\sqrt{3}e^{i\phi}\roundP{z_{2}-z_{1}}\\
e^{i\phi}\roundP{z_{2}-z_3}+ e^{i\phi}\roundP{z_{1}-z_{3}}\end{pmatrix}
\end{align*}

Moreover, scale transformations will be clearly given by a scalar multiplication. Consequently, a general transformation performed to the triangle relative coordinates can be represented by a scalar and a phase, or which is the same, a complex number. Now, taking this into account, we can take the equivalence class of shapes associated with a spinor $\Psi$ as $\left[ \Psi \right] = \left\{ z\Psi | z \in \mathbb{C} / \left\{ 0 \right\} \right\}$. Here the equivalence class $\left[ \Psi \right]$ represents a shape of the triangle formed by the particles, and together with the norm of the spinor, it codifies the actual information needed of the triangle for the Hamiltonian of guiding centres \eqref{eq:newham}. 

In fact, if we take the space of normalised representatives $\Psi/\sqrt{S}$ we can note that the Hilbert spaces associated to the shapes of the triangles and to the wave vectors of a Qubit are equivalent, and can be mapped to a Bloch sphere. To comprehend better this mapping, let us define the vector consisting of the expected value of the Pauli matrices $\vec{\sigma}$ in terms of the equivalence class representatives:

\begin{equation}
\vec{\zeta} = \frac{1}{S} \Psi^{\dagger}\vec{\sigma}\Psi
\end{equation}

The interpretation of each component of this vector becomes clear with some heavy algebra. Here we present the result in terms of the position vectors $ \vec{r}_i $ of each particle:

\begin{align*}
\zeta_1 &= \frac{1}{2\sqrt{3}S}\roundP{\norm{\vec{r}_2 -\vec{r}_3}^2-\norm{\vec{r}_1 -\vec{r}_3}^2}\\
\zeta_2 &= \frac{1}{\sqrt{3}S}\roundP{\vec{r}_1\times\vec{r}_2 + \vec{r}_2\times\vec{r}_3 +\vec{r}_3\times\vec{r}_1}\cdot\hat{z} = \frac{2A}{\sqrt{3}S} \\
\zeta_3 &= \frac{1}{6S}\roundP{2\norm{\vec{r}_2 -\vec{r}_1}^2-\norm{\vec{r}_3 -\vec{r}_1}^2-\norm{\vec{r}_3 -\vec{r}_2}^2}\\
\end{align*}

From this, the second component of the vector can be easily interpreted as proportional to the signed area $A$ of the triangle which sign encodes the chirality of the system.\\

The first and third components, in turn, share a similar composition with the components of the vector $\Psi$, mapping the complex quantities $z_i$ that encode the vertices coordinates, to the squared norm of the opposite side of the triangle. In other words, we can obtain expressions for the squared lengths of the sides of the triangle as we did in equations \eqref{eq:singlevecs}:

\begin{align*}
S &= \frac{1}{3}\sum_{i=1}^3 x_i^2 + y_i^2 -\frac{1}{3}\sum_{i > j} x_ix_j + y_iy_j  = \frac{1}{6}\roundP{\norm{\vec{r}_1 -\vec{r}_2}^2+\norm{\vec{r}_2 -\vec{r}_3}^2+\norm{\vec{r}_3 -\vec{r}_1}^2}\\
\end{align*}

\begin{equation}
\begin{aligned}
\rho_1 &\coloneqq \norm{\vec{r}_2 -\vec{r}_3}^2 = 2S\roundP{1+\frac{\sqrt{3}}{2}\zeta_1-\frac{1}{2}\zeta_3} \\
\rho_2 &\coloneqq \norm{\vec{r}_3 -\vec{r}_1}^2 = 2S\roundP{1-\frac{\sqrt{3}}{2}\zeta_1-\frac{1}{2}\zeta_3} \\
\rho_3 &\coloneqq \norm{\vec{r}_2 -\vec{r}_1}^2 = 2S\roundP{1+\zeta_3}
\end{aligned}
\label{eq:lados}
\end{equation}

The information carried by the three components of the vector $\vec{\zeta}$ and the scalar $S$ presented before, can then be compressed elegantly in this fashion:

\begin{equation}
\begin{aligned}
\rho_k &= 2S\roundP{1+\vec{m}_k\cdot\vec{\zeta}}\\
\vec{m}_k &= \roundP{\sin{\frac{2\pi k}{3}},0,\cos{\frac{2\pi k}{3}}}, k \in \poisson{1,2}{3}\\
A &= \frac{\sqrt{3}S}{2} \zeta_2
\end{aligned}
\label{eq:compressed}
\end{equation}

Now, to know better the distribution of shapes in the Bloch sphere, let us take the coordinate $\zeta_2$ as the vertical axis which defines the poles and the rest as the ones defining the remaining perpendicular plane. In this terms, the pole vectors represent triangles of maximal area and the equatorial line, triangles of minimal null area. The triangles from the north hemisphere differ from their specular image with respect to the equatorial plane in the south hemisphere by only a sign in the area, which means that they have the same shape but different chirality.\\

Moreover, isosceles triangles require the vector $\vec{\zeta}$ to be perpendicular to $\vec{m}_i-\vec{m}_j$ for some $i \neq j$. It can be easily demonstrated that $\vec{m}_k \perp \vec{m}_i-\vec{m}_j$ for all $i \neq j \neq k$ and as a consequence, any isosceles triangle must follow that $\vec{\zeta}\parallel\vec{m}_i$ for any $i$. Besides, given that the vectors $\vec{m}$ live in the equatorial plane, it can be deduced that any isosceles triangle can be found on the lines on the sphere connecting the poles with the vectors $\vec{m}$. From this, it is clear that the polar triangles are equilateral.\\

Taking the analysis a little bit further, it can be seen from the form that takes the sides of the vertices $\rho_k$ in equation \eqref{eq:compressed}, that in the equatorial plane if we look the directions $\vec{\zeta} = \vec{m}_k$, no side of the triangle will be null. Hence, those directions will determine isosceles triangles with null area, that is, triangles whose two sides sum up to the other. On the other hand, if we look directions $\vec{\zeta} = -\vec{m}_k \coloneqq \vec{n}_k$, we find that one of the sides will be null. We find then on those directions, isosceles triangles with one of the sides equal to zero, which accounts for its null area.\\

One can be curious for the direction of equilateral triangles with null area, that is, all the points in the same position.  The answer is that we cannot find them in the Bloch sphere because, as we stated earlier, it is a representation for normalised vectors $\Psi/\sqrt{S}$ and this triangle would require $S=0$. However, if we take the normalisation factor $S$ as the radius of the sphere, and consider the family of Bloch spheres for different radii filling $\mathbb{R}^3/\left\{0\right\}$, we surely would find these triangles in any direction on the limit $S\to 0$.\\

\section{Analysis of the motion}
Taking into account the decoupling of the center of mass movement with respect to the relative coordinates which are encoded in the canonical spinor $\Psi$, the addition of terms that depend only on the center of mass components to the Hamiltonian will not affect the equations of motion of the relative coordinates. We then can add a term $-\omega L$ from the Hamiltonian in equation \eqref{eq:hamgc} with no effect on $\Psi$:

\begin{align*}
H_{\Psi} &= H_{gc} - \omega L =  V\roundP{\Psi,\Psi^*}+ \omega (J-L) = V\roundP{\Psi,\Psi^*}+ \omega S\\
&= V\roundP{\Psi_\alpha,\Psi_\alpha^*}+ \omega\Psi_\alpha\Psi^*_\alpha
\end{align*}

With this new Hamiltonian for relative coordinates, one can obtain the temporal evolution of the spinor $\Psi$, taking the Poisson bracket with it, the time evolution generator. However, to express the Poisson bracket in terms of the new spinorial coordinates, we can reformulate the canonical relations in \eqref{eq:canonicalSpinor} to resemble more the classical relations of $x$ and $p_x$:

\begin{align*}
\poisson{\Psi_1}{i\Psi^*_1} = \poisson{\Psi_2}{i\Psi^*_2} = 1
\end{align*}

With this clarification, we can take the usual Poisson bracket expressions with the canonical variables $(\Psi_\alpha,i\Psi^*_\alpha)$:

\begin{align*}
\poisson{f}{g} &= \sum_{\alpha = 1}^2 \roundP{\frac{\partial f}{\partial \Psi_\alpha }\frac{\partial g}{\partial i\Psi_\alpha^*}-\frac{\partial f}{\partial i\Psi_\alpha^* }\frac{\partial g}{\partial \Psi_\alpha}} \\
i\poisson{f}{g} &= \sum_{\alpha = 1}^2 \roundP{\frac{\partial f}{\partial \Psi_\alpha }\frac{\partial g}{\partial \Psi_\alpha^*}-\frac{\partial f}{\partial \Psi_\alpha^* }\frac{\partial g}{\partial \Psi_\alpha}} \\
\end{align*}

And the temporal evolution of $\Psi_\alpha$ takes the simple form:

\begin{align*}
i\dot{\Psi}_\alpha = i\poisson{\Psi_\alpha}{H_{rc}} = \frac{\partial V}{\partial \Psi_\alpha^*} + \omega \Psi_\alpha
\end{align*}

We can go further in this analysis if we take into account the Bloch sphere mapping. We then can pass from the spinorial representation given by $(\Psi,\Psi^*)$ to the shape space given by $(\vec{\zeta},S)$. In this fashion, the time evolution of the spinor component $\Psi_\alpha$ will be given by:

\begin{align*}
i\dot{\Psi}_\alpha &= \frac{\partial V}{\partial \Psi_\alpha^*} + \omega\Psi_\alpha\\
\\
&=  \frac{\partial V}{\partial S}\frac{\partial S}{\partial \Psi_\alpha^*}+ \frac{\partial V}{\partial \zeta_j}\frac{\partial \zeta_j}{\partial \Psi_\alpha^*} + \omega\Psi_\alpha  \\
\\
&= \omega \Psi_\alpha + \frac{\partial V}{\partial S}\frac{\partial}{\partial \Psi_\alpha^*}\roundP{\Psi^*_\beta\Psi_\beta} + \frac{\partial V}{\partial \zeta_j}\frac{\partial}{\partial \Psi_\alpha^*}\roundP{\frac{1}{S}\Psi^*_{\delta}\sigma^j_{\delta\beta}\Psi_\beta}\\
\\
&= \omega \Psi_\alpha + \frac{\partial V}{\partial S}\Psi_\alpha +\frac{\partial V}{\partial \zeta_j}\roundP{ \frac{-1}{S^2}\Psi^*_{\delta}\sigma^j_{\delta\beta}\Psi_\beta\frac{\partial S}{\partial \Psi_\alpha^*} +\frac{1}{S}\sigma^j_{\alpha\beta}\Psi_\beta}\\
\\
&= \roundP{\omega \Psi +\frac{\partial V}{\partial S}\Psi+ \frac{1}{S}\frac{\partial V}{\partial \zeta_j}\roundP{-\zeta_j \Psi + \sigma^j \Psi}}_\alpha
\end{align*}

This result can be compressed taking into account all components of $\Psi$:

\begin{equation}
i\dot{\Psi} = \roundP{\omega +\frac{\partial V}{\partial S}+ \frac{1}{S}\frac{\partial V}{\partial \zeta_j}\roundP{-\zeta_j \mathbb{I}  + \sigma^j }}\Psi
\label{eq:Psi_time_ev}
\end{equation}

Now, as we know that $S$ is a constant of movement, the time evolution of the components $\zeta_j$  of the vector of shapes will be given by:

\begin{align*}
\dot{\zeta}_j &= \frac{1}{S}\roundP{ \dot{\Psi^{\dagger}}\sigma^j\Psi + \Psi^{\dagger}\sigma^j\dot{\Psi}} =
\frac{1}{S}\roundP{ \dot{\Psi^{\dagger}}{\sigma^j}^{\dagger}\Psi + \Psi^{\dagger}\sigma^j\dot{\Psi}}\\
\\
&= \frac{2}{S}\Re{\left\{\Psi^{\dagger}\sigma^j\dot{\Psi}\right\}}
\end{align*}

From equation \eqref{eq:Psi_time_ev} this expression can be simplified:

\begin{align*}
\frac{1}{S}\Psi^{\dagger}\sigma^k\dot{\Psi} &= -i\roundP{\omega +\frac{\partial V}{\partial S}- \frac{1}{S}\frac{\partial V}{\partial \zeta_j}\zeta_j }\frac{1}{S}\Psi^{\dagger}\sigma^k\dot{\Psi} -i\frac{1}{S}\Psi^{\dagger}\sigma^k\roundP{\frac{1}{S}\frac{\partial V}{\partial \zeta_j}\sigma^j}\Psi\\
\\
\frac{2}{S}\Re{\left\{\Psi^{\dagger}\sigma^k\dot{\Psi}\right\}} &= 2\Re{\left\{-i\frac{1}{S}\frac{\partial V}{\partial \zeta_j}\frac{1}{S}\Psi^{\dagger}\sigma^k\sigma^j\Psi\right\}}\\
\\
\dot{\zeta}_k &= -i\frac{1}{S}\frac{\partial V}{\partial \zeta_j}\frac{1}{S}\Psi^{\dagger}\sigma^k\sigma^j\Psi +i\frac{1}{S}\frac{\partial V}{\partial \zeta_j}\frac{1}{S}\Psi^{\dagger}\sigma^j\sigma^k\Psi\\
\\
&= -i\roundP{\frac{1}{S}\frac{\partial V}{\partial \zeta_j}}\frac{1}{S}\Psi^{\dagger}\left[\sigma^k,\sigma^j\right]\Psi
\end{align*}

Where $[a,b] = ab-ba$ is the commutator between $a$ and $b$. It is widely known that the Pauli matrices $\vec{\sigma}$ satisfy the commutation relations $[\sigma^i,\sigma^j] = 2i\epsilon_{ijk}\sigma_k$, that are in fact  the angular momentum relations. Taking this into account, we finally obtain an expression for the time evolution of the shape vector $\vec{\zeta}$:

\begin{align}
\dot{\zeta}_k &= 2\epsilon_{kjl}\roundP{\frac{1}{S}\frac{\partial V}{\partial \zeta_j}}\frac{1}{S}\Psi^{\dagger}\sigma_l\Psi =  2\epsilon_{kjl}\roundP{\frac{1}{S}\frac{\partial V}{\partial \zeta_j}}\zeta_l \nonumber \\ 
\nonumber \\ 
\dot{\vec{\zeta}} &= \frac{2}{S}\roundP{\nabla_{\vec{\zeta}}V}\times\vec{\zeta} 
\end{align}

Having the time evolution for the vector $\vec{\zeta}$ and knowing the potential $V$, the shape of the triangle can be easily retrieved from the relations in \eqref{eq:lados}. Regarding the orientation of the triangle, the analysis requires more work because when passing from the spinor representation $(\Psi,\Psi^*)$ to the shape space $(\vec{\zeta},S)$ we loose the information of the phase of the spinor. This phase, as we deduced before, encodes the rotation (orientation) of the triangle and is not important when analysing shapes.\\

To retrieve the orientation information of the triangle, we would like to obtain a definition of phase for infinitesimally separated spinors that would correspond to infinitesimal evolution in time of the trajectories. To achieve this, let us take an infinitesimal rotation in the spinorial representation and compare with the infinitesimal separation of the spinor:

\begin{align*}
e^{-id\chi}\Psi &\approx \Psi -id\chi\Psi = \Psi + d\Psi\\
d\chi\Psi &= id\Psi \\
d\chi\Psi^\dagger\Psi &= i\Psi^\dagger d\Psi\\
d\chi &= i\frac{\Psi^\dagger d\Psi}{\Psi^\dagger\Psi}
\end{align*}

Now, if we identify $\Psi$ with the state of the triangle at a time $t$ and $\Psi+d\Psi$ with a time $t+dt$, we can obtain an expression for the dynamical angular velocity of the triangle:\\

\begin{align}
\omega_r^{(dyn)} &\coloneqq \frac{d\chi}{dt} = i\frac{\Psi^\dagger\dot{\Psi}}{\Psi^\dagger\Psi}\nonumber\\
\nonumber\\
\omega_r^{(dyn)}&= \roundP{\omega +\frac{\partial V}{\partial S}-\cancel{\frac{1}{S}\frac{\partial V}{\partial \zeta_j}\zeta_j}} + \cancel{\roundP{\frac{1}{S}\frac{\partial V}{\partial \zeta_j}}\frac{ \Psi^\dagger \sigma^j\Psi}{\Psi^\dagger \Psi}}\nonumber\\
\nonumber\\
\omega_r^{(dyn)} &= \omega + \frac{\partial V}{\partial S}
\label{eq:wdyn}
\end{align}

From the equation \eqref{eq:wdyn} for the dynamical angular velocity of the triangle, one may define a phase between two finite-time separated spinors in the trajectory by simple integration. However, if the initial and final shapes are not the same, the interpretation of this phase as an angle of rotation of the triangle is not possible.\\

If we then define the phase $\Delta \chi = \int_0^{T_s}\omega_r^{(dyn)}dt$ over a period $T_s$ of the shape motion, as the initial and final shapes must be equal over this period, one would expect that $\Delta \chi$ may be interpreted as the angle of rotation of the shape. This is wrong as long as a geometrical phase must be taken into account by parallel transport due to the change of shape of the triangle in the period. To see this more clearly, let us parametrise the spinor $\Psi$:obtained

\begin{equation*}
\Psi = \sqrt{S}e^{-i\gamma} \begin{pmatrix}\cos{\frac{\theta}{2}}\\
e^{-i\phi}\sin{\frac{\theta}{2}}\end{pmatrix}
\end{equation*}

Then apply the definition of $\omega_r^{(dyn)}$:

\begin{align*}
\omega_r^{(dyn)} &= i\frac{1}{S}\Psi^\dagger\dot{\Psi} \\
\\
&= \roundP{e^{i\gamma}\begin{pmatrix}\cos{\frac{\theta}{2}} \\e^{i\phi}\sin{\frac{\theta}{2}}\end{pmatrix}}\cdot e^{-i\gamma}\roundP{\dot{\gamma}\begin{pmatrix}\cos{\frac{\theta}{2}} \\e^{-i\phi}\sin{\frac{\theta}{2}}\end{pmatrix}+ \begin{pmatrix}0 \\ \dot{\phi}e^{-i\phi}\sin{\frac{\theta}{2}}\end{pmatrix} + \frac{\dot{\theta}}{2}\begin{pmatrix}-\sin{\frac{\theta}{2}} \\e^{-i\phi}\cos{\frac{\theta}{2}}\end{pmatrix}}\\
\\
&= \dot{\gamma}+\dot{\phi}\sin^2{\frac{\theta}{2}}
\end{align*}

Being $\gamma$ the rotation phase of the spinor, over a period $T_s$ one can truthfully identify $\Delta \gamma$ with the angle of rotation of that shape, obtaining then the overall angular velocity $\omega_r = \Delta \gamma /T_s$. With this clarification, the expression for $\omega_r$ in terms of the dynamical and geometrical angular velocities is easily deduced:

\begin{align*}
\Delta \gamma &= \Delta \chi - \int_0^{T_s}\dot{\phi}\sin^2{\frac{\theta}{2}}dt\\
\\
\omega_r &= \frac{\Delta \gamma}{T_s} = \frac{1}{T_s}\int_0^{T_s}\omega_r^{(dyn)}dt -\frac{1}{T_s}\oint\sin^2{\frac{\theta}{2}}d\phi\\
\\
\omega_r &= \left\langle \omega_r^{(dyn)} \right\rangle + \omega_r^{(geo)}\\
\end{align*}

With $\omega_r^{(geo)}= -\frac{1}{T_s}\oint\sin^2{\frac{\theta}{2}}d\phi$ being the solid angle on the Bloch sphere enclosed by the level curve of the shape:




