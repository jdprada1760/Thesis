\chapter{Conclusions}

In this thesis we reproduced a very insightful approach developed by Botero and Leyvraz \cite{alonso}, to analyse the three-body problem in the plane. We showed that in the classical case, this problem is integrable due to the decoupling of the guiding centres and the linear momenta of the particles, in the regimen of big magnetic fields. Moreover, the canonical transformation of relative coordinates can be manipulated to codify the shape of the triangle in a Bloch sphere. With this analysis, the study of the orientation and shape of the triangle can be studied naturally in terms of the dynamical and geometrical angular velocity of the system, which are in fact related to the angular variables in the angle-action formalism.\\

In the quantum counterpart of the problem, a similar analysis was shown, where the symmetry of the system has a more noticeable role. The mentioned relative coordinate transformation implements a Schwinger quantum angular momentum $SU(2)$ connected to the Bloch sphere mapping of the triangle shapes. Moreover, the reduction of the degrees of freedom of the problem due to the big magnetic field can be compared to the quantum Hall effect on the plane, where the electron states are restricted to the ground state in this regimen.\\

With the study carried by Botero and Leyvraz \cite{alonso}, the correlation of the classic and quantum formalisms becomes more noticeable through the mapping of the quantum and classic symmetries $SU(2)$ of the relative triangular coordinates, and the connection between the classic decoupling of the guiding centres degrees of freedom and the quantum confinement of the particles to the ground state on the big magnetic field regimen. This, together with the fact that Haldane's formalism \cite{haldane} of the quantum base states for Hall effect on the sphere is very similar to that of the planar quantum Hall effect, leaded us to believe that a formalism similar to Botero's and Leyvraz's on the regime of big magnetic fields could be easily developed for the sphere.\\

Moreover, as we analysed the one-body problem on the sphere as an introduction to the three-body formalism, we discovered that in terms of trajectories, there were many similarities that indicated the finding of the wanted analogue analysis was actually possible.\\

When we tried to develop the mentioned formalism on the sphere, we found that although there are some analogues for some quantities and some coincidences on the trajectories, the structure of the algebra of the guiding center transformation on the sphere is very different from that of the plane, impeding further simplification of the problem trough the crucial relative coordinate transformation.\\

We studied the three-body problem on the sphere in Cartesian, spherical and stereographic coordinates, all of which have their advantages and flaws. We found that if we want a very similar analysis from that of the plane, the stereographic projection, due to its nature and its results, is the most suitable candidate. Otherwise, the mapping of the plane formalism to the sphere is not trivial, nor even similar.  This is supported by the fact that on the Haldane formalism, some spinor representation, introduced to describe the ground states, is deeply related to the stereographic projection.\\

Given all these hints, we believe that a proper development of a spherical formalism similar to the planar from Botero et al. \cite{alonso}, can be performed through the use of stereographic coordinates. However, as this mapping demonstrated to be not trivial, we leave this study for future work.\\

