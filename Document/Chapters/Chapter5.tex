\chapter{Quantum analysis of the three body problem in the plane}
In this chapter we continue the analysis implemented in \cite{alonso}, explaining in detail the quantum treatment of the three body problem in the plane. To achieve this goal, the quantum problem will be defined at the same time as the quantum analogous of the classical reduction of problem is going to be briefly discussed; then, the remaining problem is going to be analysed; and finally, some intriguing and rare aspects of this problem are going to be clarified.\\

MODIFY LAST PART
\section{The quantum three body problem on the plane}
The quantum treatment of the three body problem on the plane is not very different from the classical approach, given that we thoroughly worked the classical system out in the Hamiltonian formalism. The only thing we have to do is to apply the principles of canonical quantization \cite{Canonical quantization}. \\

Including the last considerations, the quantum Hamiltonian is going to be the same as the one expressed in \eqref{eq:ham2d}, and as canonical transformations still work in quantum mechanics, a similar reduction of the problem can be performed.\\

%In this case, the classical guiding centres motion decoupling from the linear momenta $\vec{\pi}_i$ will produce a separable quantum Hamiltonian in the same fashion. However, the decoupling cannot be applied to the states, at least not completely. What we can do is consider pure states that can be totally separable, bearing in mind that at the end the final general solution  may consist of a linear combination of those pure states which may not be separable.Once left this clear, we can continue with the reduction of the problem in the same fashion we did in the classical formalism.%\\

First, given the big magnetic field decoupling, the problem can be reduced to the analysis of the guiding centres motion given by the Hamiltonian \eqref{eq:newham}. The decoupled system of linear momenta is equally identified as a quantum harmonic oscillator, which is also very well known amongst the physical sciences community.\\

%Puzzling comment on the Heissenbergs uncertainty principle consequences of the canonical transformation.\\%

Now, we are left with the reduced Hamiltonian for the guiding centres \eqref{eq:newham} with the same definition of the canonical coordinates. The integrability analysis carried out in Chapter 2 where there were found two integrals of motion in involution with the Hamiltonian, may be interpreted as the finding of a complete set of commutative observables (CSCO) \cite{csco}. This means that the Hamiltonian of the guiding centres can be simultaneously diagonalised with the operators $L$ and $J$ that can be interpreted as orbital and total angular momentum. However, as the problem can refined to a better extent, this occurrence is currently of no interest for us.\\

Proceeding with the reduction of the problem, we can decouple here the motion of the center of mass, which describes quantum harmonic motion, from the guiding centre  Hamiltonian and then take the same canonical transformation given by the spinor in \eqref{eq:spinor}:

\begin{align}
\Psi = \frac{1}{2\sqrt{3}}\begin{pmatrix}\sqrt{3}\roundP{z_{2}-z_{1}}\\
z_{2}+z_{1}-2z_{2}\end{pmatrix}
\label{eq:spinor2}
\end{align}

The Hamiltonian associated with the relative coordinates will be the same as \eqref{eq:LastHam}:

\begin{align}
H_{rc}\roundP{\Psi,\Psi^*} = V^*\roundP{\Psi,\Psi^*}
\end{align}

In quantum mechanics the symmetries of a system play a substantial role in the solution of the associated problem. They determine the algebra of operators, and consequently, the eigenvectors and eigenvalues that generate the states of the Hilbert space. In this case, we know that the potential $V^*$ follows the symmetries associated with the special Euclidean group in 2D ($SE(2)$), but given the simplification of the problem, this symmetry slightly vary for $V^*\roundP{\Psi,\Psi^*}$.\\

From here, the following step in analysing the quantum 2-body problem restricted to the plane is to look for the symmetries in the most simplified Hamiltonian \eqref{eq:LastHam}.\\


\section{The Schwinger oscillator and the angular momentum representation}
