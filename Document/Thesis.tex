%% ----------------------------------------------------------------
%% Thesis.tex -- MAIN FILE (the one that you compile with LaTeX)
%% ---------------------------------------------------------------- 

% Set up the document
\documentclass[a4paper, 11pt, oneside]{Thesis}  % Use the "Thesis" style, based on the ECS Thesis style by Steve Gunn
\graphicspath{Figures/}  % Location of the graphics files (set up for graphics to be in PDF format)

% Include any extra LaTeX packages required
%\usepackage[square, numbers, comma, sort&compress]{natbib}  % Use the "Natbib" style for the references in the Bibliography
\usepackage{verbatim}  % Needed for the "comment" environment to make LaTeX comments
\usepackage{vector}  % Allows "\bvec{}" and "\buvec{}" for "blackboard" style bold vectors in maths
\hypersetup{urlcolor=black, colorlinks=false}  % Colours hyperlinks in blue, but this can be distracting if there are many links.

% Defines the command for the norm
\usepackage{cancel}
\usepackage{amsmath} 
\usepackage{mathtools}
\newcommand{\norm}[1]{\left\lVert #1 \right\rVert}
% Command for round parenthesis
\newcommand{\roundP}[1]{\left( #1 \right)}
% Command for poisson brackets
\newcommand{\poisson}[2]{\left\lbrace #1, #2 \right\rbrace}

%\pagestyle{fancy}
%\renewcommand{\chaptermark}[1]{\markboth{#1}{}}
%\fancyhf{}
%\fancyhead[RE]{\chaptername~\thechapter}
%\fancyhead[LO]{\leftmark}
%\fancyhead[LE,RO]{\thepage}
%% ----------------------------------------------------------------
\begin{document}
\frontmatter      % Begin Roman style (i, ii, iii, iv...) page numbering

% Set up the Title Page
\title  {Three body problem in the spherical geometry}
\authors  {\texorpdfstring
            {\href{jd.prada1760@uniandes.edu.co}{Jesus David Prada Gonzalez}}
            {Jesus David Prada Gonzalez}
            }           
\addresses  {\groupname\\\deptname\\\univname}  % Do not change this here, instead these must be set in the "Thesis.cls" file, please look through it instead
\date       {\today}
\subject    {}
\keywords   {}

\maketitle
%% ----------------------------------------------------------------

\setstretch{1.3}  % It is better to have smaller font and larger line spacing than the other way round

% Define the page headers using the FancyHdr package and set up for one-sided printing
\fancyhead{}  % Clears all page headers and footers
\rhead{\thepage}  % Sets the right side header to show the page number
\lhead{}  % Clears the left side page header

\pagestyle{fancy}  % Finally, use the "fancy" page style to implement the FancyHdr headers

%% ----------------------------------------------------------------
% The Abstract Page
\addtotoc{Abstract}  % Add the "Abstract" page entry to the Contents
\abstract{
\addtocontents{toc}{\vspace{1em}}  % Add a gap in the Contents, for aesthetics
English: In this document an elegant formalism for the study of the planar three-body problem by Botero et al. \cite{alonso} is reproduced with the objective to be mapped to its spherical counterpart. We deduced that a straightforward analogue is not trivial and studied the advantages and flaws of the Cartesian, spherical and stereographic coordinates in terms of feasibility of the aforesaid mapping. We discovered that the most suitable set of coordinates in which a very similar analogue of Botero's formalism can be deduced is the stereographic projection, due to its nature and the obtained results. However, the spherical analogue of the studied planar three-body problem was demonstrated to be not obvious nor simple. We leave the proper study of this map for future work.\\

Espa\~nol: En este documento se reproduce un formalismo elegante para estudiar el problema de tres cuerpos en el plano desarrollado por Botero et al. \cite{alonso}, con el objetivo de traducirlo al caso esf\'erico. Deducimos que un an\'alogo directo no es trivial al intentar esta traducci\'on en coordenadas cartesianas, esf\'ericas y estereogr\'aficas. En cada caso expusimos las ventajas y las falencias de cada sistema de coordenadas en terminos de feasibilidad de esta traducci\'on. Deducimos as\'i que si buscamos un mapeo de dicho formalismo que no cambie mucho su esencia, el mejor candidato es la proyecci\'on estereogr\'afica, dada su naturaleza y los resultados obtenidos. De todas maneras, como el problema demostr\'o no ser obvio ni simple, dejamos el estudio de la apropiada traducci\'on del formalismo como trabajo futuro.

}

\clearpage  % Abstract ended, start a new page

%% ----------------------------------------------------------------

\setstretch{1.3}  % Reset the line-spacing to 1.3 for body text (if it has changed)

% The Acknowledgements page, for thanking everyone
\acknowledgements{
\addtocontents{toc}{\vspace{1em}}  % Add a gap in the Contents, for aesthetics

I am grateful with my mother and brother who encouraged me to work hard and cheered me up in difficult moments. I am also very thankful with my advisor Alonso Botero for his patient guidance trough the development of this document and all the things that he taught me. For me, this was truly a very fruitful semester in terms of learning. I express my gratitude to my dad, without whom, my Physics career would not have been possible.\ldots

}
\clearpage  % End of the Acknowledgements
%% ----------------------------------------------------------------
%\lhead{\emph{Contents}}  % Set the left side page header to "Contents"
\tableofcontents  % Write out the Table of Contents




%% ----------------------------------------------------------------
\mainmatter	  % Begin normal, numeric (1,2,3...) page numbering
\pagestyle{fancy}  % Return the page headers back to the "fancy" style
% Include the chapters of the thesis, as separate files
% Just uncomment the lines as you write the chapters

\chapter{Introduction}
The $N$-body problem is a highly-known and studied issue in Physics. It consists on researching the trajectories that $N$ point masses would follow when interacting with external and internal forces with certain defined characteristics, given all the information of the initial conditions.\\

At first, the principal interest in the study of this problem was the exact prediction of the path of celestial bodies. However, as the problem was known more, it was understood that its study is of great importance not only for astrophysics but for the theoretical comprehension of classical mechanics. Great minds of physics and mathematics have worked in the restricted problem of three bodies, as Poincar\'e \cite{introPoincare} and Jacobi \cite{introJacobi}. It was this way that Poincar\'e, in an attempt of solving the three body problem, discovered that it is not integrable in general, and formulated the bases of what is known nowadays as chaos theory \cite{introPoincare}.\\

Independently of the formalism chosen to define the system, the N-body problem is reduced to the integration of the equations of motion for the $N$ particles. As this problem has been known to be non-integrable for the $N\geq 3$ cases, with the exception of few occurrences that involve forces with strange features as explicit dependence of the position an velocities \cite{strangeCases}, the study of realistic 3 body problems is a very interesting question in physics.\\

Regarding quantum mechanics, there are many puzzling aspects about its fundamentals, as the role that symmetries play in the quantization of states. In this case, the analysis of N-body problems may be useful to get some intuition about the meaning and cause of those perplexing facts, given the strong correlation between classical and quantum mechanics via the Hamiltonian formalism.

\section{Motivation}
One of the particular realistic cases of the three body problem that is known to be integrable, is the system of three particles on the plane with mass $m$ and charge $e$ under the influence of a large constant magnetic field perpendicular to the surface and forces whose potentials are invariant under rotations and translations in the plane \cite{alonso}.\\

This problem is integrable by virtue of the action of the big magnetic potential, which decouples the movement of  the particle into two degrees of freedom known as guiding centres and linear momenta. Further analysis of the movement of the particles can be performed in terms of a canonical transformation that encodes the shape of the generated triangle into a Bloch sphere.\\

The analysis carried out in \cite{alonso} can be extended to the quantum formalism via the canonical quantization rules \cite{Cq}, where the role of the $SU(2)$ symmetry becomes clear with the implementation of a Schwinger angular momentum associated with the Bloch sphere variables. The classical decoupling of the guiding centres degrees of freedom is associated with the fact that, in the quantum formalism, electrons are confined to the ground state of energy due to the Landau level gap modulated by the magnetic field, an effect that is at the basis of the Quantum Fractional Hall Effect (QFHE). The clear relation between both quantum and classical formalisms, together with the fact that the spherical QFHE shows similar behaviour than the planar case on the large magnetic field regimen, tempts us to intend a generalization of the procedure for the spherical geometry.  \\

To do so, as a first approach to the problem, we are going reproduce the calculations and analysis from \cite{alonso} for the classical and quantum system in great detail. Then, we are going to study some important aspects of the classical one-body problem on the sphere under the influence of a magnetic monopole to obtain some intuition about the analogies of the movement of the particle in this case. Then, an analogue formalism to study the motion of the three-body problem in the sphere is going to be intended, giving some insights as why the translation of the analysis from the plane to the sphere is not trivial.
 % Innroduction

\chapter{The three body problem in the plane}

In this chapter a classic approach of a somehow general case of the three body problem in 2 dimensions is going to be presented. This will give some necessary intuition to develop the analogous problem in the spherical geometry. To begin with, the problem is going to be described in great detail; then its integrability is going to be proven; and finally, a formalism to describe the movement of the particles is going to be presented.\\

\section{The definition of the problem}

The three body problem presented here is that of three particles of electrical charge $e$ and mass $m$ confined to a plane, under the influence of a strong magnetic field perpendicular to it, and forces whose potentials satisfy translational and rotational symmetries in the plane.\\

Given this information, the Hamiltonian associated with this system has the form:

\begin{equation}
H = \sum_{i=1}^{3} \frac{1}{2m} \norm{ \vec{p_i} - 
e\vec{ A } \left( \vec{q_i} \right)}^2
+ V \roundP{ \vec{q_1},\vec{q_2},\vec{q_3} }
+\frac{\omega_c^2}{2m}\sum_{i=1}^{3} \norm{\vec{q_i}}^2
\label{eq:ham2d}
\end{equation}

Where $\vec{q_i} = x_i \hat{\imath} + y_i \hat{\jmath}$, $\vec{p_i} = {p_x}_i\hat{\imath} + {p_y}_i\hat{\jmath}$ and $\vec{A}\roundP{\vec{q}}$ is the magnetic vector potential, which satisfies $\nabla \times \vec{A} = B\hat{k}$.\\

Besides, the potential $V \roundP{ \vec{q_1},\vec{q_2},\vec{q_3} }$ satisfies the symmetries:

\begin{equation}
V \roundP{ R\vec{q_1}+\vec{a},R\vec{q_2}+\vec{a},R\vec{q_3}+\vec{a}  }= V \roundP{ \vec{q_1},\vec{q_2},\vec{q_3} }
\label{eq:vsym}
\end{equation}

For any rotation $R$ and translation $\vec{a}$ in the plane.\\

\section{The canonical transformation of the guiding centres}

For the proof of integrability for this system, and for further analysis of the trajectories of the particles, let us perform the well known transformation of the guiding centres.\\

This transformation is defined by the following two equations:

\begin{equation}
\vec{\pi_i} = \vec{p_i} - e\vec{A}\roundP{\vec{q_i}}
\label{eq:ct2d1}
\end{equation}

\begin{equation}
\vec{R_i} = \vec{q_i} - \frac{\hat{k}\times\vec{\pi_i}}{eB}
\label{eq:ct2d2}
\end{equation}

The equation \eqref{eq:ct2d1} passes from the canonical momentum $\vec{p_i}$ to the linear momentum $\vec{\pi_i}$, which is much more intuitive and understandable; while the equation \eqref{eq:ct2d2} transforms the general position  $\vec{q_i}$ to the position of the instantaneous guiding centre $\vec{R_i}$.\\
% Sera carreta?

In a system without the interaction potentials, the electrically charged particles are known to perform the circular motion of the cyclotron with radii that depends on the initial linear momenta. In this case, the guiding centres would be constant in time as would be the linear momenta. However, with the introduction of an interacting potential, the momentum of each particle may vary making the guiding centre change too, which is why the instantaneous interpretation of the guiding centres is necessary.\\

Now, let us calculate the Poisson brackets for this new set of coordinates in a specific particle.

\begin{align*}
\poisson{\pi_1}{\pi_2} &= \frac{\partial \pi_1}{\partial q_{\alpha}}\frac{\partial \pi_2}{\partial p_{\alpha}} - \frac{\partial \pi_2}{\partial q_{\alpha}}\frac{\partial \pi_1}{\partial p_{\alpha}}\\
&= -e\delta_{\alpha 2}\frac{\partial A_1}{\partial q_{\alpha}} +e\delta_{\alpha 1}\frac{\partial A_2}{\partial q_{\alpha}}\\
&= -e\frac{\partial A_1}{\partial q_{2}}  + e\frac{\partial A_2}{\partial q_{1}}\\
&= e(\nabla \times \vec{A})_3 = eB
\end{align*}
\begin{align*}
\poisson{R_1}{R_2} &= \poisson{q_1}{q_2} + \poisson{q_1}{-\frac{\pi_1}{eB}} +\poisson{\frac{\pi_2}{eB}}{q_2} + \poisson{\frac{\pi_2}{eB}}{-\frac{\pi_1}{eB}}\\
&= \frac{1}{eB}\roundP{\cancelto{-1}{\poisson{p_1}{q_1}}- \cancelto{0}{e\poisson{A_1}{q_1}}+\cancelto{-1}{\poisson{p_2}{q_2}} \cancelto{0}{e\poisson{A_2}{q_2}}} +\frac{eB}{(eB)^2}\\
&= \frac{-2}{eB}+ \frac{1}{eB} = -(eB)^{-1}
\end{align*}
\begin{align*}
\poisson{R_{1}}{\pi_{2}} &= \poisson{q_1}{\pi_2}+\cancelto{0}{\poisson{\frac{\pi_2}{eB}}{\pi_2}}\\
&= \cancelto{0}{\poisson{q_1}{p_2}}-e\cancelto{0}{\poisson{q_1}{A_2}}\\
&= \poisson{R_2}{\pi_1} = 0
\end{align*}

This Poisson brackets can be generalised to the transformation for the three particles. Taking $i,j = \poisson{1,2}{3}$ and $\alpha,\beta=\poisson{1}{2}$:

\begin{equation}
\poisson{\pi_{i,\alpha}}{\pi_{j,\beta}}=\roundP{eB}\delta_{ij}\epsilon_{\alpha \beta}  
\label{eq:pb1}
\end{equation}

\begin{equation}
\poisson{R_{i,\alpha}}{R_{j,\beta}}= -\roundP{eB}^{-1} \delta_{ij}\epsilon_{\alpha \beta}  
\label{eq:pb2}
\end{equation}

\begin{equation}
\poisson{R_{i,\alpha}}{\pi_{j,\beta}}=0 
\label{eq:pb3} 
\end{equation}

Equations \eqref{eq:pb1}-\eqref{eq:pb2} allow us to identify the proposed transformation as canonical. However, this is not the usual canonical transformation where the position coordinates and the momentum coordinates are canonical conjugates. In this special case, one component of the momentum is canonical conjugate with the other momentum coordinate, as equally happens for the position coordinates.\\

Now, with a huge magnetic field, if the potential of the interaction forces does not vary abruptly in space, we can use the approximation $\vec{R}_i \approx \vec{q}_i$ to average the potentials over the guiding centres, that is, we can replace $\vec{q}_i$ for $\vec{R}_i$ in  $V \roundP{ \vec{q_1},\vec{q_2},\vec{q_3} }$.\\

We can support the last approximation as follows: In the cyclotron problem, the radius of the circular motion described is proportional to the linear momentum and inversely proportional to the magnetic field. Then, in the presence of a big $B$, the radius of the cyclotron would shrink to a very small size. Regarding the case we are working with, the radii of the instantaneous cyclotron motion would be proportional to $\norm{(\hat{k} \times \vec{\pi_i})(eB)^{-1}}$ and its frequency to $\sqrt{B}$. As the potential $V$ does not vary abruptly in the radii scale, the averaging of this motion over the guiding centres means that this potential does not sense that circular motion. Moreover, given the big frequency of the cyclotrons and the scale of variance of the potential, the scale of time of the local circular motions is far smaller than that of the motion of the guiding centres. Therefore, we can ignore the instantaneous quality of the circular motion, and take it as constant in a scale of time small enough for the motion of the guiding centres. In this sense we say that the coordinates for the guiding centres decouple from that of the linear momenta of the particles.\\

Before replacing the new set of coordinates in the Hamiltonian, it is necessary to do a scale transformation to obtain the proper Poisson brackets for the formal definition of canonical transformation, that is:

\begin{align*}
\vec{\pi}_i & \rightarrow (eB)^{-1/2} \vec{ \pi}_i\\
\vec{R}_i & \rightarrow \sqrt{eB} \vec{R}_i
\end{align*}

With this consideration, the Hamiltonian of the system in the new set of rescaled coordinates is given by:

\begin{equation}
H = \sum_{i=1}^{3} \frac{eB}{2m} \norm{ \vec{\pi}_i}^2
+ V\roundP{ (eB)^{-1/2}\vec{R_1},(eB)^{-1/2}\vec{R_2}, (eB)^{-1/2}\vec{R_3} }
+\frac{\omega_c^2}{2meB}\sum_{i=1}^{3} \norm{\vec{R}_i}^2
\label{eq:newham}
\end{equation}

This Hamiltonian, given equation \eqref{eq:pb3} can be decomposed in a Hamiltonian that describes the movement of the guiding centres, and other that describes de movement of the linear momenta. In one hand, the Hamiltonian for the linear momenta is easily identified with the harmonic oscillator, whereas the one that characterises the movement of the guiding centres needs a deeper analysis.\\

\section{Integrability of the system}
As the Hamiltonian describing the trajectories of the linear momenta of the particles is that of an harmonic oscillator, this part of the problem is integrable and its solutions are widely known. The guiding centre Hamiltonian, in turn, needs to be analysed more deeply. For this purpose, let us take the following convention:\\


\begin{equation}
H_{gc} = \frac{{\omega_c^*}^2}{2m} \sum_{i=1}^{3} \norm{\bar{x}^2} + \norm{\bar{y}^2}
+ V^*\roundP{\bar{x},\bar{y}}
\label{eq:hamgc}
\end{equation}

Where $\bar{x} = \roundP{x_1,x_2,x_3}$ and $\bar{y} = \roundP{y_1,y_2,y_3}$, being $x_i,y_i$ the rescaled coordinates of the guiding centres of the particles. For simplicity, the potential $V$ and the constant $\omega_c$ have been rescaled to take into account the scale transform of the coordinates and maintain the original form of the Hamiltonian:

\begin{align*}
&\omega^* = \frac{\omega}{\sqrt{eB}}\\
&V^*\roundP{\bar{x},\bar{y}} = V\roundP{\frac{\bar{x}}{\sqrt{eB}},\frac{\bar{y}}{\sqrt{eB}}}
\end{align*}

Clearly, the new potential $V^*$ still has the symmetries expressed in the equation \eqref{eq:vsym}. Furthermore, in the new order for the scaled guiding centres coordinates, the Poisson brackets take the form:

\begin{equation}
\poisson{y_i}{x_j} = \delta_{ij}
\label{lastpb}
\end{equation}

Now that the guiding centres Hamiltonian has been expressed in terms of the proper canonical set of coordinates, the fastest way to prove the integrability of the system is via the Liouville-Arnol'd theorem \cite[Sect. 49]{arnold}. For this theorem, it is only necessary to find 2 more independent integrals in involution (besides the Hamiltonian).\\

To get this 2 integrals, let us exploit the symmetries of the guiding centres Hamiltonian. We then define the generators of translations and rotation in the plane, which are symmetries of the potential:

\begin{align*}
T_x &= \sum_{i=1}^{3} x_i \\
T_y &= \sum_{i=1}^{3} y_i 
\end{align*}


\begin{equation}
R_z = \frac{1}{2} \sum_{i=1}^{3} \roundP{x_i^2 + y_i^2}
\label{eq:genrotation}
\end{equation}

It is easily verifiable that these are indeed the symmetries generators. To see that, take the first order infinitesimal transformations of translation and rotation:

\begin{align*}
x_i  &\rightarrow x_i + \epsilon \\
y_i &\rightarrow y_i + \epsilon \\
\roundP{x_i,y_i} &\rightarrow \roundP{x_i +\epsilon y_i, y_i - \epsilon x_i}
\end{align*}

Now note that for the infinitesimal translations, the potential of the primed coordinates is related to the potential of the normal coordinates by a directional derivative, which can be identified with the Poisson bracket of the potential $V^*$ and each generator:

\begin{align*}
0={V^*\roundP{\bar{x}+\epsilon,\bar{y}} - V^*\roundP{\bar{x},\bar{y}}} &= \epsilon \sum_{i = 1}^3 \frac{\partial V^*\roundP{\bar{x},\bar{y}}}{\partial x_i}  = \epsilon \poisson{V^*}{T_x} = 0\\
0={V^*\roundP{\bar{x},\bar{y}+\epsilon} - V^*\roundP{\bar{x},\bar{y}}} &= \epsilon \sum_{i = 1}^3 \frac{\partial V^*\roundP{\bar{x},\bar{y}}}{\partial y_i} = \epsilon \poisson{T_y}{V^*} = 0
\end{align*}

For the infinitesimal rotation, the relation is analogous:

\begin{equation*}
0={V^*\roundP{\bar{x} +\epsilon \bar{y}, \bar{y} - \epsilon \bar{y}}-V^*\roundP{\bar{x},\bar{y}}}
= \frac{\partial V^*\roundP{\bar{x},\bar{y}}}{\partial x_i} \roundP{\epsilon y_i} -                                \frac{\partial V^*\roundP{\bar{x},\bar{y}}}{\partial y_i} \roundP{\epsilon x_i}                                        = \epsilon \poisson{V^*}{R_z}
\end{equation*}

Therefore, we conclude that the generators of translations and rotations in the plane commute with the potential $V^*$ due to its symmetries. Besides, the generator of rotations is multiple of the harmonic-like part of the guiding centres Hamiltonian which validates that $R_z$ is other integral in involution. The generators of translations are not integrals in involution, for they do not commute with the harmonic potential, however, we can calculate a quantity in terms of these generators, which already commute with the potential $V^*$, to make it commute with the remaining part of $H_{gc}$: 

\begin{equation}
L = T_x^2 + T_y^2
\end{equation}

This new quantity $L$ clearly commutes with the potential $V^*$ because the Poisson bracket is  a linear differential operator in one component and it obeys the Leibniz rule. Moreover, it also commutes with the rotation generator $R_z$:

\begin{align*}
\poisson{T_x^2+ T_y^2}{ R_z} &= \sum_{i,j,k} \poisson{x_ix_j + y_iy_j}{x_k^2 + y_k^2}\\
&= \sum_{i,j,k}  \poisson{x_ix_j}{y_k^2} + \poisson{y_iy_j}{x_k^2} = \sum_{i,j,k} y_k \poisson{x_ix_j}{y_k} + x_k \poisson{y_iy_j}{x_k}\\
&= \sum_{i,j,k} y_kx_i\delta_{jk}+y_kx_j\delta_{ik} - x_ky_i\delta_{jk}- x_ky_j\delta_{ik}\\
&= \sum_{i,j} y_jx_i + y_ix_j - x_jy_i - x_iy_j = 0
\end{align*}

As we found $L$ as the last integral in involution, we conclude, by the Liouville-Arnol'd theorem, that the subsystem of guiding centres is integrable by quadratures.\\

\section{Analysis of the motion}










 % Background Theory 

\chapter{Quantum analysis of the three body problem in the plane}
In this chapter we continue the analysis implemented in \cite{alonso}, explaining in detail the quantum treatment of the three body problem in the plane. To achieve this goal, the quantum problem will be defined at the same time as the quantum analogous of the classical reduction of problem is going to be briefly discussed; then, the remaining problem is going to be analysed; and finally, some intriguing and rare aspects of this problem are going to be clarified.\\

\section{The quantum three body problem on the plane}
The quantum treatment of the three body problem on the plane is not very different from the classical approach, given that we thoroughly worked the classical system out in the Hamiltonian formalism. In fact, all the transformations and reductions from the classical approach work, however, we have to be careful when treating topics referring equations of movement.  The only thing we have to do is to apply the principles of canonical quantization \cite{Canonical quantization}. To make the problem of parsing to the quantum formalism less complicated, let us carry the analysis in fundamental units $\hbar =1$. \\

Including the last considerations, the quantum Hamiltonian is going to be the same as the one expressed in \eqref{eq:ham2d}, and as canonical transformations still work in quantum mechanics, a similar reduction of the problem can be performed.\\

%In this case, the classical guiding centres motion decoupling from the linear momenta $\vec{\pi}_i$ will produce a separable quantum Hamiltonian in the same fashion. However, the decoupling cannot be applied to the states, at least not completely. What we can do is consider pure states that can be totally separable, bearing in mind that at the end the final general solution  may consist of a linear combination of those pure states which may not be separable.Once left this clear, we can continue with the reduction of the problem in the same fashion we did in the classical formalism.%\\

First, given the big magnetic field decoupling, the problem can be reduced to the analysis of the guiding centres motion given by the Hamiltonian \eqref{eq:newham}. The decoupled system of linear momenta is equally identified as a quantum harmonic oscillator, which is also very well known amongst the physical sciences community.\\

%Puzzling comment on the Heissenbergs uncertainty principle consequences of the canonical transformation.\\%

Therefore, we are left with the reduced Hamiltonian for the guiding centres \eqref{eq:newham} with the same definition of the canonical coordinates. The integrability analysis carried out in Chapter 2 where there were found two integrals of motion in involution with the Hamiltonian, may be interpreted as the finding of a complete set of commutative observables (CSCO) \cite{csco}. This means that the Hamiltonian of the guiding centres can be simultaneously diagonalised with the operators $L$ and $J$ that can be interpreted as orbital and total angular momentum respectively. However, as the problem can be refined to a better extent, this occurrence is currently of no interest for us.\\

Now, there are some remarkable aspects referring the canonical variables $(\bar{x},\bar{y})$ associated with \eqref{eq:newham}. With these one can form creation-annihilation operators given by:\\

\begin{align*}
a_j &= \frac{1}{\sqrt{2}}(x_j+iy_j)\\
a_j^\dagger &= \frac{1}{\sqrt{2}}(x_j-iy_j)\\
\left[a_j,a_k^\dagger\right] &= \delta_{kj}
\end{align*}

We note then that the variables called $z_i$ in the classical formalism will correspond to annihilation operators in the quantum formalism.\\

Following the reduction of the problem analogously to the classical approach, we can proceed to decouple the center of mass coordinates form the relative coordinates. The spinor $\Psi$ is known to follow relations of creation-annihilation operators (taking into account the canonical quantisation), however, the center of mass was not formalised as a canonical variable in the classical approach. To take this into account, we choose the constants to make it also an annihilation operator. If we encode the center of the triangle in $b$, the decoupling transformation can be expressed in the following way:\\

\begin{align*}
\begin{pmatrix} b \\ \Psi_1 \\ \Psi_2 \end{pmatrix} &= 
\begin{pmatrix}\frac{1}{\sqrt{3}} &\frac{1}{\sqrt{3}}&\frac{1}{\sqrt{3}}\\
				-\frac{1}{\sqrt{2}}&\frac{1}{\sqrt{2}}&0\\
				\frac{1}{\sqrt{6}}&\frac{1}{\sqrt{6}}&\frac{1}{\sqrt{6}}\end{pmatrix}
\end{align*}

Where the new variables satisfy the creation-annihilation operators relation:

\begin{align*}
\left[ \Psi_\alpha,\Psi_\beta^\dagger\right] &= \delta_{\alpha\beta}\\
\left[ b,b^\dagger\right] &= 1\\
\left[ b,\Psi_\alpha^\dagger\right] &=\left[ b,\Psi_\alpha\right] = 0\\
\end{align*}

From this transformation it is worth noticing:

\begin{align*}
b^\dagger b &= \frac{1}{3}\roundP{\sum_i a_i^\dagger}\roundP{\sum_j a_j} =  \frac{1}{6}\roundP{\sum_i x_i-iy_i}\roundP{\sum_j x_j+iy_j}\\
&= \frac{1}{6}\sum_i\sum_j(x_i-iy_i)(x_j-iy_j) = \frac{1}{6}\sum_i\sum_jx_ix_j+y_iy_j -i\cancelto{i\delta_{ij}}{[xi,y_j]}\\
&= L + \frac{1}{2}\\
\end{align*}
\small
\begin{align*}
\Psi^\dagger\Psi &= \Psi_1^\dagger\Psi_1 +\Psi_2^\dagger\Psi_2\\
&= \frac{1}{2} (a_2^\dagger-a_1^\dagger)(a_2-a_1) + \frac{1}{6}(a_2^\dagger+a_1^\dagger-2a_3^\dagger)(a_2+a_1-2a_3)\\
&= \frac{1}{2} (x_2-x_1 -i(y_2-y_1))(x_2-x_1 +i(y_2-y_1))\\
&+\frac{1}{6}(x_2+x_1-2x_3-i(y_2+y_1-2y_3))(x_2+x_1-2x_3 +i(y_2+y_1-2y_3))\\
&= \frac{1}{2} (x_2-x_1)^2+(y_2-y_1)^2-i\cancelto{2}{[x_2-x_1,y_2-y_1]} \\
&+ \frac{1}{6} (x_2+x_1-2x_3)^2+(y_2+y_1-2y_3)^2-i\cancelto{3}{[x_2+x_1-2x_3,y_2+y_1-2y_3]}\\
&= S+1
\end{align*}
\normalsize

This tells us that the eigenvalues from operator $b$ are semi-integers, while those of $\Psi$ are quantized integers.\\

So far we have reduced the problem to the study of the center of mass and relative coordinates spinor. In this terms, the Hamiltonian for these variables will be given by:\\

\begin{align*}
H_{gc} &= V\roundP{\Psi,\Psi^\dagger}+ \omega (S+L)\\
&= V\roundP{\Psi,\Psi^\dagger} + \omega{\Psi^\dagger\Psi + b^\dagger b -\frac{3}{2}}\\
&= V\roundP{\Psi,\Psi^\dagger} + \omega{\Psi_1^\dagger\Psi_1 +\Psi_2^\dagger\Psi_2+ b^\dagger b -\frac{3}{2}}
\end{align*}

Ignoring the effects of the center of mass $b$, what we have here is the Hamiltonian of two oscillators $(\Psi_1,\Psi_2)$ coupled by the potential $V$ which does not affect the center of the triangle. What we will see next is that in fact, the two oscillators from $\Psi$ induce a Schwinger angular momentum \cite{Schwinger}.\\

In quantum mechanics the symmetries of a system play a substantial role in the solution of the associated problem. In fact, the role of the symmetries in quantum mechanics is more important than in classical mechanics \cite{qfhebook}. In quantum mechanics they determine the algebra of operators, and consequently, the eigenvectors and eigenvalues that generate the states of the Hilbert space. In this case, we know that the potential $V$ follows the symmetries associated with the special Euclidean group in 2D ($SE(2)$), that, given the simplification of the problem and the canonical spinor variables, is transformed into $SU(2)$.\\

The role of this symmetry can be understood as the operators $\Psi_\alpha$ implement a Schwinger oscillator \cite{Schwinger} associated with this symmetry.\\

\section{The Schwinger oscillator and the angular momentum representation}
To understand better the role of the symmetry of $\Psi$ in the problem, let us explain the theory associated with the Schwinger oscillators. First, let us consider two decoupled harmonic oscillators with the same frequency $\omega$. The Hamiltonian of this system in terms of the annihilation-creation operators $(a_1,a_2)$ of each oscillator, will be given by:

\begin{equation*}
H_{SO} = \omega\roundP{a_1^\dagger a_1 + a_2^\dagger a_2 + 1}
\end{equation*}

As the oscillators are decoupled the commutation relations between the operators $a_i$ are given by:

\begin{align*}
\left[ a_i,a_j^\dagger\right] &= \delta_{ij}\\
\left[ a_i,a_j\right] &= 0\\
\end{align*}
 
Now we are going to show that this algebra of annihilation operators induces an angular momentum algebra with symmetry $SU(2)$. To do that, let us define an angular momentum in terms of the $\{a_i\}$ and see its commutation relations:

\begin{align*}
J_i &= \frac{1}{2}\sigma^i_{\alpha\beta}a^\dagger_\alpha a_\beta \\
\\
[J_i,J_j] &= \frac{1}{4}\left[ \sigma^i_{\alpha\beta}a^\dagger_\alpha a_\beta,\sigma^j_{\gamma\delta}a^\dagger_\gamma a_\delta \right] = \frac{1}{4}\sigma^i_{\alpha\beta}\sigma^j_{\gamma\delta}\left[ a^\dagger_\alpha a_\beta,a^\dagger_\gamma a_\delta \right]\\
&= \frac{1}{4}\sigma^i_{\alpha\beta}\sigma^j_{\gamma\delta}\roundP{a^\dagger_\alpha [a_\beta,a^\dagger_\gamma a_\delta]+  [a^\dagger_\alpha,a^\dagger_\gamma a_\delta]a_\beta}\\
&= \frac{1}{4}\sigma^i_{\alpha\beta}\sigma^j_{\gamma\delta}\roundP{a^\dagger_\alpha [a_\beta,a^\dagger_\gamma ]a_\delta +  a^\dagger_\gamma[a^\dagger_\alpha, a_\delta]a_\beta }\\
&= \frac{1}{4}\roundP{\sigma^i_{\alpha\beta}\sigma^j_{\beta\delta}a^\dagger_\alpha a_\delta - \sigma^i_{\alpha\beta}\sigma^j_{\gamma\alpha}a^\dagger_\gamma a_\beta }\\
&= \frac{1}{4}a^\dagger_\alpha a_\beta \left[ \sigma^i,\sigma_j \right]_{\alpha\beta}\\
&= i\epsilon_{ijk} \frac{1}{4} \sigma^k_{\alpha\beta}a^\dagger_\alpha a_\beta \\
\\
[J_i,J_j] &= i\epsilon_{ijk}J_k
\end{align*}

This representation of angular momentum is no different from the usual one. In fact, one can take annihilation-creation operators $J^\pm = J_1\pmiJ_2$ as well as the total angular momentum $J^2$ to deduce the rules for the quantum numbers $(j,m)$. Let us calculate $J^2$ to completely characterise this angular momentum algebra:





 % Experimental Setup

\chapter{The problem of a charged particle in the magnetic field of a monopole}

With the objective to obtain some intuition about the N-body problem restricted to a spherical geometry, we study in this chapter the symmetries and trajectories of a charged particle under the influence of the magnetic field of a monopole. To achieve this we present the deduction, via the Lagrangian formalism, of the so called Poincar\'e cone \cite{poincare} that characterises the trajectory of a particle in this situation. We then extrapolate the important symmetries used in the Lagrangian formalism to the Hamiltonian formalism to retrieve some important aspects of the classical counterparts of the known Haldane formalism \cite{haldane}.\\

We deduce in this chapter that the Poincar\'e cone, together with the restriction to the spherical surface, results in the particles describing uniform circular motion on the sphere, in analogy with the case of the plane.

\section{Definition of the Lagrangian}
Let $L$ be the Lagrangian of a charged particle of charge $-e$ and mass $m$, under the influence of a magnetic monopole of magnitude $g$. Then $L$ takes the form:

\begin{equation}
L\roundP{\vec{r},\dot{\vec{r}}} = \frac{m}{2}\norm{\dot{\vec{r}}}^2 - e\vec{A}_{\hat{u}}(\vec{r})\cdot\dot{\vec{r}},
\label{eq:lagrangian}
\end{equation}

where $\vec{A}(\vec{r})$ is the vector potential of the magnetic monopole with singularity along the direction defined by the unit vector $\hat{u}$. This singularity arises from our inability to reproduce the magnetic field of the monopole with the electromagnetic theory, and it has no physical meaning \cite{haldane}. In addition, the different vector potentials identified by different unit vectors $\hat{u}$ are related by gauge transformations which leave the trajectories invariant. This family of vector potentials is given by \cite{vectorPotentials}:

\begin{equation}
\vec{A}_{\hat{u}}(\vec{r}) = \frac{g}{r}\frac{\hat{u}\times\hat{r}}{1+\hat{u}\cdot\hat{r}}.
\label{eq:monopolepotential}
\end{equation}

\section{The symmetries and its conserved quantities}
The first thing to note in the previously defined Lagrangian is its time independence, which yields to the conservation of the Jacobi integral:

\begin{equation*}
 \frac{\partial L}{\partial\dot{\vec{r}}}\cdot\dot{\vec{r}} - L = m\norm{\dot{\vec{r}}}^2                    =m\norm{\dot{\vec{r_0}}}^2 ,
\end{equation*}

with $\dot{\vec{r_0}}$ the initial velocity of the particle. Here, the Jacobi integral clearly represents the kinetic energy of the particle, which is conserved because constant magnetic fields do no work.\\

Now, due to the simplicity of the problem, one can expect some other symmetries. The next symmetry presented here is not associated with the Lagrangian, but rather with the action invariance due to the associated transformation involving the time variable. If we define the action as in the equation \eqref{eq:action}, it can be seen that it may be invariant under a proper scale transform of position and time. It is not difficult to find this transform, and it is presented in equation \eqref{eq:scaletrans}.\\

\begin{equation}
S = \int_{t_1}^{t_2}L\roundP{\vec{r},\dot{\vec{r}}}dt = \int_{t_1}^{t_2}\roundP{\frac{m}{2}\norm{\dot{\vec{r}}}^2 - e\vec{A}_{\hat{u}}(\vec{r})\cdot\dot{\vec{r}}}dt,
\label{eq:action}
\end{equation}

\begin{equation}
\begin{aligned}
\vec{r}' &= e^{s}\vec{r}\\
t'&= e^{2s}t.
\end{aligned}
\label{eq:scaletrans}
\end{equation}

As a result of this symmetry, by the general form of Noether's theorem, there must be a conserved quantity implied. To calculate it we prefer the method stated in \cite[2.19 Noether's Thm]{scheck}; however, to apply this method, the Lagrangian must be parametrised to include the time $t$ as a generalised coordinate.\\

To achieve this, note that:

\begin{align*}
\dot{\vec{r}} = \frac{d\vec{r}}{dt} = \frac{\frac{d\vec{r}}{d\tau}}{\frac{dt}{d\tau}} \coloneqq \frac{\mathring{\vec{r}}}{\mathring{t}}.
\end{align*}

Then the action can be written in terms of the new parameter $\tau$:

\begin{equation*}
S = \int_{t_1}^{t_2}L\roundP{\vec{r},\dot{\vec{r}}}dt = \int_{\tau_1}^{\tau_2}\mathring{t}L\roundP{\vec{r},\mathring{\vec{r}}{\mathring{t}}^{-1}}d\tau                                                                    = \int_{\tau_1}^{\tau_2}L'\roundP{\vec{r},\mathring{\vec{r}},\mathring{t}}d\tau,
\label{eq:actiontau}
\end{equation*}

which, by analogy with equation \eqref{eq:action}, gives the new parametrised Lagrangian $L'$:

\begin{equation*}
L'\roundP{\vec{r},\mathring{\vec{r}},\mathring{t}} = \mathring{t}L\roundP{\vec{r},\mathring{\vec{r}}{\mathring{t}}^{-1}} = \frac{m}{2\mathring{t}}\norm{\mathring{\vec{r}}}^2 - e\vec{A}_{\hat{u}}(\vec{r})\cdot\mathring{\vec{r}}.
\end{equation*}

As we converted the symmetry of the action $S$ in a symmetry of the Lagrangian $L'$, the invariance given by  equation \eqref{eq:scaletrans} becomes clear. Now, using Noether's theorem \cite{scheck}, we obtain the conserved quantity according to the transformation \eqref{eq:scaletrans}: \\

\begin{align*}
G &= \left. \frac{\partial L'}{\partial \mathring{q_i}} \frac{\partial q'_i}{\partial s} \right|_{s=0} = -m\norm{\frac{\mathring{\vec{r}}}{\mathring{t}}}^2t + \roundP{\frac{m\mathring{\vec{r}}}{\mathring{t}} - e\vec{A}_{\hat{u}}}\cdot \vec{r}\\
\\
G &= m\dot{\vec{r_0}}\cdot\vec{r_0} = -m\norm{\dot{\vec{r}}}^2t + m\dot{\vec{r}}\cdot\vec{r} .
\end{align*}

Working the previous conserved quantity one can obtain an equation for the magnitude of the position in function of time:

\begin{align*}
\nonumber
 2\dot{\vec{r}}\cdot\vec{r} &= \frac{d\norm{r}^2}{dt} = 2\norm{\dot{\vec{r_0}}}^2t + 2\dot{\vec{r_0}}\cdot\vec{r_0}\\
 \\ 
 r^2 &= r_0^2 +2\dot{\vec{r_0}}\cdot\vec{r_0}t +\norm{\dot{\vec{r_0}}}^2t^2 ,
\end{align*}

\begin{equation}
r^2 = \norm{\vec{r_0}+\dot{\vec{r_0}}t}^2 .
\label{eq:radius}
\end{equation}

From equation \eqref{eq:radius} it is important to note that the only way the radius of the particle stays constant is that the initial velocity of the particle is zero, otherwise, the time term will always contribute to the change of that radius. Furthermore, as constant magnetic fields only affect perpendicular velocities, one expect equation \eqref{eq:radius} to be fulfilled not only in norm, but in direction when the initial velocity $\dot{\vec{r_0}}$ is parallel to the initial radius $\vec{r_0}$. We then retrieve the formula for uniform motion with no acceleration. To verify this suspicion, let us analyse other quantities associated with the symmetries of the problem.\\

Taking our attention to the usual symmetries on the original Lagrangian $L$ in equation \eqref{eq:lagrangian} , we note that once chosen a unit vector $\hat{u}$ for the vector potential  $\vec{A}_{\hat{u}}$, $L$ is invariant under rotations around said unit vector $\hat{u}$. To obtain the conserved quantity, take following infinitesimal transformation:

\begin{equation}
\vec{r}' = \vec{r} + s\roundP{\hat{u}\times\vec{r}}.
\label{eq:rotinv}
\end{equation}

As this transformation does not include the time $t$, we can perform the calculation of the conserved quantity over the original Lagrangian:

\begin{align*}
G_2 &= \left. \frac{\partial L}{\partial \dot{\vec{r}}}\cdot\frac{\partial \vec{r}}{\partial s} \right|_{s=0}\\
G_2 &= \roundP{m\dot{\vec{r}} - e\vec{A}_{\hat{u}}}\cdot \roundP{\hat{u}\times\vec{r}} \\
G_2 &= \roundP{m\vec{r}\times\dot{\vec{r}}}\cdot\hat{u}-eg\frac{\norm{\hat{u}\times\hat{r}}^2}{\roundP{1+\hat{r}\cdot\hat{u}}}\\
G_2 &= \roundP{m\vec{r}\times\dot{\vec{r}}}\cdot\hat{u} -eg\roundP{1-\hat{r}\cdot\hat{u}}\\
J_{\hat{u}} &\coloneqq \roundP{{m\vec{r}\times\dot{\vec{r}}}+ eg\hat{r}}\cdot\hat{u} = const.
\end{align*}

Now, the last argument is valid for any unit vector $\hat{u}$ chosen, and this yields the conservation of the  known Poincar\'e vector in equation \eqref{eq:poincarevec}.

\begin{equation}
\vec{J} = m\vec{r}\times\dot{\vec{r}}+ eg\hat{r}.
\label{eq:poincarevec}
\end{equation}

This last symmetry is very meaningful because it restricts the trajectory of the particle to a cone centred in the origin with central vector $\hat{J}$. To verify this, it is only necessary to see that the radial component of the Poincar\'e vector is constant for all points in the trajectory, which means that the angle between $\vec{J}$ and $\vec{r}(t)$ is a constant, in other words, the path of the particle is restricted to a cone:

\begin{align*}
\vec{J}\cdot\hat{r} = eg.
\end{align*}

Furthermore, we can deduce from the conservation of the Poincar\'e vector that the angular momentum $\mathbb{L}= m\vec{r}\times\dot{\vec{r}}$ of the particle is constant in magnitude and that it determines the aperture of the cone of restriction:

\begin{equation}
\begin{aligned}
\norm{\vec{J}}^2 &= \norm{\mathbb{L}}^2 +\roundP{gc}^2\\
\norm{\mathbb{L}} &= const\\
\cos{\theta} &= \frac{\vec{J}\cdot\hat{r}}{\norm{\vec{J}}} = \sqrt{\frac{\roundP{ge}^2}{\norm{\mathbb{L}}^2 +\roundP{ge}^2}}.
\end{aligned}
\label{eq:poincarecone}
\end{equation}

From this equations it can be seen that the angle of aperture of the Poincar\'e cone is zero when the angular momentum $\mathbb{L}$ cancel, which means that the particle performs rectilinear motion, as deduced before from the other symmetries of the problem.\\

It is important to note here that if we restrict the trajectories of the particle to be in a sphere, we cannot carry a scale transform, hence we would not obtain the radius trajectory described in equation \eqref{eq:radius}. However, a rotation is consistent with the norm conservation of the restriction to the sphere, and consequently, we can obtain the Poincar\'e's vector invariance. The cone confinement together with the restriction of the trajectories to a constant radius would result in the particle describing a circular trajectory on the sphere.\\ 

Moreover, in the sphere restriction, the position vector and the velocity vector must be perpendicular, therefore, as the norm of the position is the constant radius of the sphere, the conservation of the angular momentum would result in the conservation of the linear velocity, meaning that the circular motion of the particle is in fact uniform, in analogy with the planar problem.\\

Another important aspect that can be deduced from the set of equations \eqref{eq:poincarecone}, is that if we choose a magnetic monopole with big charge $g$ compared to the angular momentum (in proper units), the angle of the Poincar\'e cone would tend to zero. In the case of the particle restricted to the sphere, this would mean that the radius of the circular motion would also tend to zero, as happens with the already studied case of the particle in the plane. This gives us some clues as where to look for the analogue guiding center formalism in the case of the magnetic monopole.\\

\section{Important quantities in the Hamiltonian formalism}
From the analysis carried out before in the Lagrangian formalism, we can obtain some important quantities that are useful to describe the movement of particles in the presence of a magnetic monopole, some of which are associated with certain symmetries of the problem. In the Hamiltonian formalism, we would like to study specially the angular momentum $\mathbb{L}= m\vec{r}\times\dot{\vec{r}}$ and the Poincar\'e vector $\vec{J} = \roundP{m\vec{r}\times\dot{\vec{r}}}+ eg\hat{r}$. To do that, let us first calculate the Hamiltonian for the particle in the magnetic field of a monopole, this time restricting the radius $r$ of the sphere to a constant:

\begin{equation}
H\roundP{\vec{r},\vec{p}} = \left.\frac{1}{2m}\norm{\vec{p}+e\vec{A}_{\hat{u}}(\vec{r})}^2\right|_{S^2}.
\label{eq:hamiltonian}
\end{equation}

As the Hamiltonian is the Legendre transform of the Lagrangian, we obtain the generalised momentum $\vec{p}$  in terms of the velocity of the particle. Moreover, we can see that the Hamiltonian is just the kinetic energy of the particle:

\begin{equation*}
\begin{aligned}
\vec{p} &= \frac{\partial L}{\partial \dot{\vec{r}}} = m\dot{\vec{r}}-e\vec{A}_{\hat{u}} \coloneqq \vec{\pi}-e\vec{A}_{\hat{u}} \\
\\
H &= \left.\frac{1}{2m}\norm{\vec{\pi}}^2\right|_{S^2}.
\end{aligned}
\end{equation*}

From here we can propose the quantity $\mathbb{L}= \vec{r}\times\vec{\pi}$ as a canonical angular momentum. To verify this, it is useful to calculate first some Poisson brackets related to the linear momentum $\vec{\pi} = \vec{p} + e\vec{A}$, as it was done in the first chapter:


\begin{align*}
\poisson{\pi_i}{\pi_j} &= \frac{\partial \pi_i}{\partial r_l}\frac{\partial \pi_j}{\partial p_l} - \frac{\partial \pi_j}{\partial r_l}\frac{\partial \pi_i}{\partial p_l}\\
&= e\delta_{lj}\frac{\partial A_i}{\partial r_l} -e\delta_{li}\frac{\partial A_j}{\partial r_l}            = e\roundP{\delta_{lj}\delta_{mi} - \delta_{li}\delta_{mj}}\frac{\partial A_m}{\partial r_l} \\
&= -e\epsilon_{ijk}\epsilon_{lmk}\frac{\partial A_m}{\partial r_l}\\
&= -e\epsilon_{ijk}(\nabla \times \vec{A})_k = e\epsilon_{ijk}B_k = -\frac{eg}{r^3}\epsilon_{ijk}r_k,
\end{align*}

\begin{align*}
\poisson{\pi_i}{r_j} &= \frac{\partial \pi_i}{\partial r_l}\cancelto{0}{\frac{\partial r_j}{\partial p_l}} - \frac{\partial r_j}{\partial r_l}\frac{\partial \pi_i}{\partial p_l} =  -\delta_{ij}.
\end{align*}

Then the algebra becomes a little bit easier for $\mathbb{L}$

\begin{align*}
\poisson{L_i}{L_j} &= \poisson{\epsilon_{iab}r_a\pi_b}{\epsilon_{jcd}r_c\pi_d} = \epsilon_{iab}\epsilon_{jcd}\poisson{r_a\pi_b}{r_c\pi_d}\\
&= \epsilon_{iab}\epsilon_{jcd}\roundP{r_ar_c\poisson{\pi_b}{\pi_d}+\pi_br_c\poisson{r_a}{\pi_d}            +r_a\pi_d\poisson{\pi_b}{r_c} + \pi_b\pi_d\cancelto{0}{\poisson{r_a}{r_c}}}\\
&= \epsilon_{iab}\epsilon_{jcd}\roundP{-\frac{eg}{r^3}\epsilon_{bdk}r_ar_cr_k+\pi_br_c\delta_{ad}-\pi_dr_a\delta_{bc}}\\
&= -\frac{eg}{r^3}\epsilon_{iab}\roundP{\delta_{jk}\delta_{cb} - \delta_{jb}\delta_{ck}}r_ar_cr_k + \roundP{\delta_{bj}\delta_{ic} - \cancel{\delta_{bc}\delta_{ij}}}\pi_br_c - \roundP{\delta_{aj}\delta_{ib} - \cancel{\delta_{ia}\delta_{jb}}}\pi_dr_a\\
& = -\frac{eg}{r^3}\roundP{r_j\cancelto{0}{\epsilon_{iab}r_ar_b}-\epsilon_{aji}r_a{r_br_b}}             +\roundP{\delta_{ia}\delta_{jb}-\delta_{ib}\delta_{ja}}r_a\pi_b\\
& = \epsilon_{ijk}\roundP{\epsilon_{abk}r_a\pi_b-{eg}\hat{r}_k} = \epsilon_{ijk}\roundP{L_k-{eg}\hat{r}_k}.
\end{align*}

Now, we can observe that $\mathbb{L}$ does not follow the canonical relations for angular momentum, which is not surprising because the momentum $\vec{\pi}$ used in the definition of $\mathbb{L}$ is not the canonical momentum $\vec{p}$. We can fix this by taking a slight variation in $\mathbb{L}$:\\

\begin{equation}
\mathbb{J} \coloneqq \mathbb{L} + eg\hat{r}.
\label{eq:ham poincarevec}
\end{equation}

The Poisson bracket relations for the quantity $\mathbb{J}$ are given by:

\begin{align*}
\poisson{J_i}{J_j} &= \poisson{L_i + \frac{eg}{r}r_i}{L_j + \frac{eg}{r}r_j}\\
&= \poisson{L_i}{L_j} + \frac{eg}{r}\roundP{\poisson{L_i}{r_j} +\poisson{r_i}{L_j}} + \roundP{\frac{eg}{r}}^2\cancelto{0}{\poisson{r_i}{r_j}}\\
& = \epsilon_{ijk}\roundP{L_k-{eg}\hat{r}_k}+\frac{eg}{r}\roundP{-\delta_{lj}\frac{\partial \epsilon_{iab}r_a\roundP{p_b+eA_b}}{\partial p_l}+ \delta_{il}\frac{\partial\epsilon_{jcd}r_c\roundP{p_d+eA_d}}{\partial p_l}}\\
& = \epsilon_{ijk}\roundP{L_k-{eg}\hat{r}_k}+ \frac{eg}{r}\roundP{-\epsilon_{jia}r_a+\epsilon_{ijc}r_c}\\
&= \epsilon_{ijk}\roundP{L_k+{eg}\hat{r}_k} = \epsilon_{ijk}J_k.
\end{align*}

Then we deduce that $\mathbb{J}$ satisfies the canonical relations for angular momentum. It is not difficult to see that this is indeed the Poincar\'e vector in equation \eqref{eq:poincarevec} and that, as seen before, it determines the center of the circular motion performed by the particle:

\begin{align*}
\mathbb{J} &= \vec{r}\times\vec{\pi}+{eg}\hat{r} = m{\vec{r}}\times\dot{\vec{r}} + eg\hat{r} = \vec{J}.
\end{align*}

Moreover, we can calculate the Poisson brackets of this angular momentum with other important vectors of the system, namely $\vec{r}$ and $\vec{L}$:

\begin{align*}
\poisson{J_i}{r_j} &= \poisson{L_i+eg\hat{r}_i}{r_j} =  \epsilon_{iab}r_a\poisson{\pi_b}{r_j} \\
&= \epsilon_{ijk}r_k\\
\\
\poisson{J_i}{L_j} &= \poisson{J_i}{J_j-\frac{eg}{r}r_i} = \epsilon_{ijk}J_k-\frac{eg}{r}\poisson{L_i}{r_j} =\epsilon_{ijk}(J_k-eg\hat{r}_k) \\
&= \epsilon_{ijk}L_k.
\end{align*}

With this, we conclude that the angular momentum $\mathbb{J}$ is also the generator of rotations over the sphere.\\

Here it is important to clarify that this angular momentum is gauge-invariant and arises, as seen before in the Lagrangian formalism, from the rotational symmetry of the system that leaves invariant the equations of movement. This is indeed the difference between this angular momentum $\mathbb{J}$ and the canonical angular momentum $\vec{r}\times\vec{p}$.









 % Experiment 1

\chapter{Quantum analysis of the three body problem in the plane}

\section{The spinorial representation}

\section{The Schwinger oscillator and the angular momentum representation}
 % Experiment 2

%\chapter{Conclusions}

In this thesis we reproduced a very insightful approach developed by Botero and Leyvraz \cite{alonso}, to analyse the three-body problem in the plane. We showed that in the classical case, this problem is integrable due to the decoupling of the guiding centres and the linear momenta of the particles, in the regimen of big magnetic fields. Moreover, the canonical transformation of relative coordinates can be manipulated to codify the shape of the triangle in a Bloch sphere. With this analysis, the study of the orientation and shape of the triangle can be studied naturally in terms of the dynamical and geometrical angular velocity of the system, which are in fact related to the angular variables in the angle-action formalism.\\

In the quantum counterpart of the problem, a similar analysis was shown, where the symmetry of the system has a more noticeable role. The mentioned relative coordinate transformation implements a Schwinger quantum angular momentum $SU(2)$ connected to the Bloch sphere mapping of the triangle shapes. Moreover, the reduction of the degrees of freedom of the problem due to the big magnetic field can be compared to the quantum Hall effect on the plane, where the electron states are restricted to the ground state in this regimen.\\

With the study carried by Botero and Leyvraz \cite{alonso}, the correlation of the classic and quantum formalisms becomes more noticeable through the mapping of the quantum and classic symmetries $SU(2)$ of the relative triangular coordinates, and the connection between the classic decoupling of the guiding centres degrees of freedom and the quantum confinement of the particles to the ground state on the big magnetic field regimen. This, together with the fact that Haldane's formalism \cite{haldane} of the quantum base states for Hall effect on the sphere is very similar to that of the planar quantum Hall effect, leaded us to believe that a formalism similar to Botero's and Leyvraz's on the regime of big magnetic fields could be easily developed for the sphere.\\

Moreover, as we analysed the one-body problem on the sphere as an introduction to the three-body formalism, we discovered that in terms of trajectories, there were many similarities that indicated the finding of the wanted analogue analysis was actually possible.\\

When we tried to develop the mentioned formalism on the sphere, we found that although there are some analogues for some quantities and some coincidences on the trajectories, the structure of the algebra of the guiding center transformation on the sphere is very different from that of the plane, impeding further simplification of the problem trough the crucial relative coordinate transformation.\\

We studied the three-body problem on the sphere in Cartesian, spherical and stereographic coordinates, all of which have their advantages and flaws. We found that if we want a very similar analysis from that of the plane, the stereographic projection, due to its nature and its results, is the most suitable candidate. Otherwise, the mapping of the plane formalism to the sphere is not trivial, nor even similar.  This is supported by the fact that on the Haldane formalism, some spinor representation, introduced to describe the ground states, is deeply related to the stereographic projection.\\

Given all these hints, we believe that a proper development of a spherical formalism similar to the planar from Botero et al. \cite{alonso}, can be performed through the use of stereographic coordinates. However, as this mapping demonstrated to be not trivial, we leave this study for future work.\\

 % Results and Discussion

\chapter{Conclusions}

In this thesis we reproduced a very insightful approach developed by Botero and Leyvraz \cite{alonso}, to analyse the three-body problem in the plane. We showed that in the classical case, this problem is integrable due to the decoupling of the guiding centres and the linear momenta of the particles, in the regimen of big magnetic fields. Moreover, the canonical transformation of relative coordinates can be manipulated to codify the shape of the triangle in a Bloch sphere. With this analysis, the study of the orientation and shape of the triangle can be studied naturally in terms of the dynamical and geometrical angular velocity of the system, which are in fact related to the angular variables in the angle-action formalism.\\

In the quantum counterpart of the problem, a similar analysis was shown, where the symmetry of the system has a more noticeable role. The mentioned relative coordinate transformation implements a Schwinger quantum angular momentum $SU(2)$ connected to the Bloch sphere mapping of the triangle shapes. Moreover, the reduction of the degrees of freedom of the problem due to the big magnetic field can be compared to the quantum Hall effect on the plane, where the electron states are restricted to the ground state in this regimen.\\

With the study carried by Botero and Leyvraz \cite{alonso}, the correlation of the classic and quantum formalisms becomes more noticeable through the mapping of the quantum and classic symmetries $SU(2)$ of the relative triangular coordinates, and the connection between the classic decoupling of the guiding centres degrees of freedom and the quantum confinement of the particles to the ground state on the big magnetic field regimen. This, together with the fact that Haldane's formalism \cite{haldane} of the quantum base states for Hall effect on the sphere is very similar to that of the planar quantum Hall effect, leaded us to believe that a formalism similar to Botero's and Leyvraz's on the regime of big magnetic fields could be easily developed for the sphere.\\

Moreover, as we analysed the one-body problem on the sphere as an introduction to the three-body formalism, we discovered that in terms of trajectories, there were many similarities that indicated the finding of the wanted analogue analysis was actually possible.\\

When we tried to develop the mentioned formalism on the sphere, we found that although there are some analogues for some quantities and some coincidences on the trajectories, the structure of the algebra of the guiding center transformation on the sphere is very different from that of the plane, impeding further simplification of the problem trough the crucial relative coordinate transformation.\\

We studied the three-body problem on the sphere in Cartesian, spherical and stereographic coordinates, all of which have their advantages and flaws. We found that if we want a very similar analysis from that of the plane, the stereographic projection, due to its nature and its results, is the most suitable candidate. Otherwise, the mapping of the plane formalism to the sphere is not trivial, nor even similar.  This is supported by the fact that on the Haldane formalism, some spinor representation, introduced to describe the ground states, is deeply related to the stereographic projection.\\

Given all these hints, we believe that a proper development of a spherical formalism similar to the planar from Botero et al. \cite{alonso}, can be performed through the use of stereographic coordinates. However, as this mapping demonstrated to be not trivial, we leave this study for future work.\\

 % Conclusion

%% ----------------------------------------------------------------
% Now begin the Appendices, including them as separate files

\addtocontents{toc}{\vspace{2em}} % Add a gap in the Contents, for aesthetics

\appendix % Cue to tell LaTeX that the following 'chapters' are Appendices

%\input{Appendices/AppendixA}	% Appendix Title

%\input{Appendices/AppendixB} % Appendix Title

%\input{Appendices/AppendixC} % Appendix Title

\addtocontents{toc}{\vspace{2em}}  % Add a gap in the Contents, for aesthetics
\backmatter

%% ----------------------------------------------------------------
\bibliographystyle{unsrt}  % Use the "unsrtnat" BibTeX style for formatting the Bibliography
\bibliography{Bibliography}

\end{document}  % The End
%% ----------------------------------------------------------------