%% ----------------------------------------------------------------
%% Thesis.tex -- MAIN FILE (the one that you compile with LaTeX)
%% ---------------------------------------------------------------- 

% Set up the document
\documentclass[a4paper, 11pt, oneside]{Thesis}  % Use the "Thesis" style, based on the ECS Thesis style by Steve Gunn
\graphicspath{Figures/}  % Location of the graphics files (set up for graphics to be in PDF format)

% Include any extra LaTeX packages required
%\usepackage[square, numbers, comma, sort&compress]{natbib}  % Use the "Natbib" style for the references in the Bibliography
\usepackage{verbatim}  % Needed for the "comment" environment to make LaTeX comments
\usepackage{vector}  % Allows "\bvec{}" and "\buvec{}" for "blackboard" style bold vectors in maths
\hypersetup{urlcolor=black, colorlinks=false}  % Colours hyperlinks in blue, but this can be distracting if there are many links.

% Defines the command for the norm
\usepackage{cancel}
\usepackage{amsmath} 
\usepackage{mathtools}
\newcommand{\norm}[1]{\left\lVert #1 \right\rVert}
% Command for round parenthesis
\newcommand{\roundP}[1]{\left( #1 \right)}
% Command for poisson brackets
\newcommand{\poisson}[2]{\left\lbrace #1, #2 \right\rbrace}

%\pagestyle{fancy}
%\renewcommand{\chaptermark}[1]{\markboth{#1}{}}
%\fancyhf{}
%\fancyhead[RE]{\chaptername~\thechapter}
%\fancyhead[LO]{\leftmark}
%\fancyhead[LE,RO]{\thepage}
%% ----------------------------------------------------------------
\begin{document}
\frontmatter      % Begin Roman style (i, ii, iii, iv...) page numbering

% Set up the Title Page
\title  {The three-body problem in the spherical geometry}
\authors  {\texorpdfstring
            {\href{jd.prada1760@uniandes.edu.co}{Jesus David Prada Gonzalez}}
            {Jesus David Prada Gonzalez}
            }           
\addresses  {\groupname\\\deptname\\\univname}  % Do not change this here, instead these must be set in the "Thesis.cls" file, please look through it instead
\date       {\today}
\subject    {}
\keywords   {}

\maketitle
%% ----------------------------------------------------------------

\setstretch{1.3}  % It is better to have smaller font and larger line spacing than the other way round

% Define the page headers using the FancyHdr package and set up for one-sided printing
\fancyhead{}  % Clears all page headers and footers
\rhead{\thepage}  % Sets the right side header to show the page number
\lhead{}  % Clears the left side page header

\pagestyle{fancy}  % Finally, use the "fancy" page style to implement the FancyHdr headers

%% ----------------------------------------------------------------
% The Abstract Page
\addtotoc{Abstract}  % Add the "Abstract" page entry to the Contents
\abstract{
%\addtocontents{toc}{\vspace{1em}}  % Add a gap in the Contents, for aesthetics
In this document an insightful formalism for the study of the planar three-body problem by Botero and Leyvraz \cite{alonso} is reproduced with the objective to be mapped to its spherical counterpart. We deduced that a straightforward analogue is not trivial and studied the advantages and flaws of the Cartesian, spherical and stereographic coordinates in terms of feasibility of the aforesaid mapping. We discovered that the most suitable set of coordinates, with which a very similar analogue of Botero's formalism can be deduced, is the stereographic projection, due to its nature and the obtained results. However, the spherical analogue of the studied planar three-body problem was demonstrated to be not obvious nor simple. We leave the proper study of this map for future work.\\

}
\clearpage  % Abstract ended, start a new page


%% ----------------------------------------------------------------
% The Abstract Page
\addtotoc{Resumen}  % Add the "Abstract" page entry to the Contents
\resumen{
%\addtocontents{toc}{\vspace{1em}}  % Add a gap in the Contents, for aesthetics
En este documento se reproduce un formalismo para estudiar el problema de tres cuerpos en el plano, desarrollado por Botero y Leyvraz \cite{alonso}, con el objetivo de traducirlo al caso esf\'erico. Deducimos que un an\'alogo directo no es trivial al intentar esta traducci\'on en coordenadas cartesianas, esf\'ericas y estereogr\'aficas. En cada caso expusimos las ventajas y las falencias de cada sistema de coordenadas en t\'eminos de viabilidad de esta traducci\'on. Deducimos que si buscamos un mapeo de dicho formalismo, que no cambie mucho su esencia, el mejor sistema de coordenadas para estudiar el problema es la proyecci\'on estereogr\'afica, dada su naturaleza y los resultados obtenidos. De todas maneras, como el problema demostr\'o no ser obvio ni simple, dejamos el estudio de la apropiada traducci\'on del formalismo como trabajo futuro.

}
\clearpage  % Abstract ended, start a new page

%% ----------------------------------------------------------------

\setstretch{1.3}  % Reset the line-spacing to 1.3 for body text (if it has changed)

% The Acknowledgements page, for thanking everyone
\acknowledgements{
\addtocontents{toc}{\vspace{1em}}  % Add a gap in the Contents, for aesthetics

I am grateful with my mother and brother who encouraged me to work hard and cheered me up in difficult moments. I am also very thankful with my advisor Alonso Botero for his patient guidance trough the development of this document and all the things that he taught me. For me, this was truly a very fruitful semester in terms of learning. I express my gratitude to my dad, without whom my Physics career would not have been possible.\ldots

}
\clearpage  % End of the Acknowledgements
%% ----------------------------------------------------------------
%\lhead{\emph{Contents}}  % Set the left side page header to "Contents"
\tableofcontents  % Write out the Table of Contents




%% ----------------------------------------------------------------
\mainmatter	  % Begin normal, numeric (1,2,3...) page numbering
\pagestyle{fancy}  % Return the page headers back to the "fancy" style
% Include the chapters of the thesis, as separate files
% Just uncomment the lines as you write the chapters

\chapter{The three body problem in the plane}

In this chapter a classic approach of a somehow general case of the three body problem in 2 dimensions is going to be presented. This will give some necessary intuition to develop the analogous problem in the spherical geometry. To begin with, the problem is going to be described in great detail; then its integrability is going to be proven; and finally, a formalism to describe the movement of the particles is going to be presented.\\

\section{The definition of the problem}

The three body problem presented here is that of three particles of electrical charge $e$ and mass $m$ confined to a plane, under the influence of a strong magnetic field perpendicular to it, and forces whose potentials satisfy translational and rotational symmetries in the plane.\\

Given this information, the Hamitonian associated with this system has the form:

\begin{equation}
H = \sum_{i=1}^{3} \frac{1}{2m} \norm{ \vec{p_i} - 
e\vec{ A } \left( \vec{q_i} \right)}^2
+ V \roundP{ \vec{q_1},\vec{q_2},\vec{q_3} }
+\frac{\omega_c^2}{2m}\sum_{i=1}^{3} \norm{\vec{q_i}}^2
\label{eq:ham2d}
\end{equation}

Where $\vec{q_i} = x_i \hat{\imath} + y_i \hat{\jmath}$, $\vec{p_i} = {p_x}_i\hat{\imath} + {p_y}_i\hat{\jmath}$ and $\vec{A}\roundP{\vec{q}}$ is the magnetic vector potential, which satisfies $\nabla \times \vec{A} = B\hat{k}$.\\

Besides, the potential $V \roundP{ \vec{q_1},\vec{q_2},\vec{q_3} }$ satisfies the symmetries:

\begin{equation}
V \roundP{ R\vec{q_1}+\vec{a},R\vec{q_2}+\vec{a},R\vec{q_3}+\vec{a}  }= V \roundP{ \vec{q_1},\vec{q_2},\vec{q_3} }
\label{eq:vsym}
\end{equation}

For any rotation $R$ and translation $\vec{a}$ in the plane.\\

\section{The canonical transformation of the guiding centres}

For the proof of integrability for this system, and for further analysis of the trajectories of the particles, let us perform the well known transformation of the guiding centres.\\

This transformation is defined by the following two equations:

\begin{equation}
\vec{\pi_i} = \vec{p_i} - e\vec{A}\roundP{\vec{q_i}}
\label{eq:ct2d1}
\end{equation}

\begin{equation}
\vec{R_i} = \vec{q_i} - \frac{\hat{k}\times\vec{\pi_i}}{eB}
\label{eq:ct2d2}
\end{equation}

The equation \eqref{eq:ct2d1} passes from the canonical momentum $\vec{p_i}$ to the linear momentum $\vec{\pi_i}$, which is much more intuitive and understandable; while the equation \eqref{eq:ct2d2} transforms the general position  $\vec{q_i}$ to the position of the instantaneous guiding centre $\vec{R_i}$.\\
% Sera carreta?

In a system without the interaction potentials, the electrically charged particles are known to perform the circular motion of the cyclotron with radii that depends on the initial linear momenta. In this case, the guiding centres would be constant in time as would be the linear momenta. However, with the introduction of an interacting potential, the momentum of each particle may vary making the guiding centre change too, which is why the instantaneous interpretation of the guiding centres is necessary.\\

Now, let us calculate the Poisson brackets for this new set of coordinates in a specific particle.

\begin{align*}
\poisson{\pi_1}{\pi_2} &= \frac{\partial \pi_1}{\partial q_{\alpha}}\frac{\partial \pi_2}{\partial p_{\alpha}} - \frac{\partial \pi_2}{\partial q_{\alpha}}\frac{\partial \pi_1}{\partial p_{\alpha}}\\
&= -e\delta_{\alpha 2}\frac{\partial A_1}{\partial q_{\alpha}} +e\delta_{\alpha 1}\frac{\partial A_2}{\partial q_{\alpha}}\\
&= -e\frac{\partial A_1}{\partial q_{2}}  + e\frac{\partial A_2}{\partial q_{1}}\\
&= e(\nabla \times \vec{A})_3 = eB
\end{align*}
\begin{align*}
\poisson{R_1}{R_2} &= \poisson{q_1}{q_2} + \poisson{q_1}{-\frac{\pi_1}{eB}} +\poisson{\frac{\pi_2}{eB}}{q_2} + \poisson{\frac{\pi_2}{eB}}{-\frac{\pi_1}{eB}}\\
&= \frac{1}{eB}\roundP{\cancelto{-1}{\poisson{p_1}{q_1}}- \cancelto{0}{e\poisson{A_1}{q_1}}+\cancelto{-1}{\poisson{p_2}{q_2}} \cancelto{0}{e\poisson{A_2}{q_2}}} +\frac{eB}{(eB)^2}\\
&= \frac{-2}{eB}+ \frac{1}{eB} = -(eB)^{-1}
\end{align*}
\begin{align*}
\poisson{R_{1}}{\pi_{2}} &= \poisson{q_1}{\pi_2}+\cancelto{0}{\poisson{\frac{\pi_2}{eB}}{\pi_2}}\\
&= \cancelto{0}{\poisson{q_1}{p_2}}-e\cancelto{0}{\poisson{q_1}{A_2}}\\
&= \poisson{R_2}{\pi_1} = 0
\end{align*}

This Poisson brackets can be generalised to the transformation for the three particles. Taking $i,j = \poisson{1,2}{3}$ and $\alpha,\beta=\poisson{1}{2}$:

\begin{equation}
\poisson{\pi_{i,\alpha}}{\pi_{j,\beta}}=\roundP{eB}\delta_{ij}\epsilon_{\alpha \beta}  
\label{eq:pb1}
\end{equation}

\begin{equation}
\poisson{R_{i,\alpha}}{R_{j,\beta}}= -\roundP{eB}^{-1} \delta_{ij}\epsilon_{\alpha \beta}  
\label{eq:pb2}
\end{equation}

\begin{equation}
\poisson{R_{i,\alpha}}{\pi_{j,\beta}}=0 
\label{eq:pb3} 
\end{equation}

Equations \eqref{eq:pb1}-\eqref{eq:pb2} allow us to identify the proposed transformation as canonical. However, this is not the usual canonical transformation where the position coordinates and the momentum coordinates are canonical conjugates. In this special case, one component of the momentum is canonical conjugate with the other momentum coordinate, as equally happens for the position coordinates.\\

Now, with a huge magnetic field, if the potential of the interaction forces does not vary abruptly in space, we can use the approximation $\vec{R}_i \approx \vec{q}_i$ to average the potentials over the guiding centres, that is, we can replace $\vec{q}_i$ for $\vec{R}_i$ in  $V \roundP{ \vec{q_1},\vec{q_2},\vec{q_3} }$.\\

We can support the last approximation as follows: In the cyclotron problem, the radius of the circular motion described is proportional to the linear momentum and inversely proportional to the magnetic field. Then, in the presence of a big $B$, the radius of the cyclotron would shrink to a very small size. Regarding the case we are working with, the radii of the instantaneous cyclotron motion would be proportional to $\norm{(\hat{k} \times \vec{\pi_i})(eB)^{-1}}$ and its frequency to $\sqrt{B}$. As the potential $V$ does not vary abruptly in the radii scale, the averaging of this motion over the guiding centres means that this potential does not sense that circular motion. Moreover, given the big frequency of the cyclotrons and the scale of variance of the potential, the scale of time of the local circular motions is far smaller than that of the motion of the guiding centres. Therefore, we can ignore the instantaneous quality of the circular motion, and take it as constant in a scale of time small enough for the motion of the guiding centres. In this sense we say that the coordinates for the guiding centres decouple from that of the linear momenta of the particles.\\

Before replacing the new set of coordinates in the Hamiltonian, it is necessary to do a scale transformation to obtain the proper Poisson brackets for the formal definition of canonical transformation, that is:

\begin{align*}
\vec{\pi}_i & \rightarrow (eB)^{-1/2} \vec{ \pi}_i\\
\vec{R}_i & \rightarrow \sqrt{eB} \vec{R}_i
\end{align*}

With this consideration, the Hamiltonian of the system in the new set of rescaled coordinates is given by:

\begin{equation}
H = \sum_{i=1}^{3} \frac{eB}{2m} \norm{ \vec{\pi}_i}^2
+ V\roundP{ (eB)^{-1/2}\vec{R_1},(eB)^{-1/2}\vec{R_2}, (eB)^{-1/2}\vec{R_3} }
+\frac{\omega_c^2}{2meB}\sum_{i=1}^{3} \norm{\vec{R}_i}^2
\label{eq:newham}
\end{equation}

This Hamiltonian, given equation \eqref{eq:pb3} can be decomposed in a Hamiltonian that describes the movement of the guiding centres, and other that describes de movement of the linear momenta. In one hand, the Hamiltonian for the linear momenta is easily identified with the harmonic oscillator, whereas the one that characterises the movement of the guiding centres needs a deeper analysis.\\

\section{Integrability of the system}
As the Hamiltonian describing the trajectories of the linear momenta of the particles is that of an harmonic oscillator, this part of the problem is integrable and its solutions are widely known. The guiding centre Hamiltonian, in turn, needs to be analysed more deeply. For this purpose, let us take the following convention:\\


\begin{equation}
H_{gc} = \frac{{\omega_c^*}^2}{2m} \sum_{i=1}^{3} \norm{\bar{x}^2} + \norm{\bar{y}^2}
+ V^*\roundP{\bar{x},\bar{y}}
\label{eq:hamgc}
\end{equation}

Where $\bar{x} = \roundP{x_1,x_2,x_3}$ and $\bar{y} = \roundP{y_1,y_2,y_3}$, being $x_i,y_i$ the rescaled coordinates of the guiding centres of the particles. For simplicity, the potential $V$ and the constant $\omega_c$ have been rescaled to take into account the scale transform of the coordinates and maintain the original form of the Hamiltonian:

\begin{align*}
&\omega^* = \frac{\omega}{\sqrt{eB}}\\
&V^*\roundP{\bar{x},\bar{y}} = V\roundP{\frac{\bar{x}}{\sqrt{eB}},\frac{\bar{y}}{\sqrt{eB}}}
\end{align*}

Clearly, the new potential $V^*$ still has the symmetries expressed in the equation \eqref{eq:vsym}. Furthermore, in the new order for the scaled guiding centres coordinates, the Poisson brackets take the form:

\begin{equation}
\poisson{y_i}{x_j} = \delta_{ij}
\label{lastpb}
\end{equation}

Now that the guiding centres Hamiltonian has been expressed in terms of the proper canonical set of coordinates, the fastest way to prove the integrability of the system is via the Liouville-Arnol'd theorem \cite[Sect. 49]{arnold}. For this theorem, it is only necessary to find 2 more independent integrals in involution (besides the Hamiltonian).\\

To get this 2 integrals, let us exploit the symmetries of the guiding centres Hamiltonian. We then define the generators of translations and rotation in the plane, which are symmetries of the potential:

\begin{align*}
T_x &= \sum_{i=1}^{3} x_i \\
T_y &= \sum_{i=1}^{3} y_i 
\end{align*}


\begin{equation}
R_z = \frac{1}{2} \sum_{i=1}^{3} \roundP{x_i^2 + y_i^2}
\label{eq:genrotation}
\end{equation}

It is easily verifiable that these are indeed the symmetries generators. To see that, take the first order infinitesimal transformations of translation and rotation:

\begin{align*}
x_i  &\rightarrow x_i + \epsilon \\
y_i &\rightarrow y_i + \epsilon \\
\roundP{x_i,y_i} &\rightarrow \roundP{x_i +\epsilon y_i, y_i - \epsilon x_i}
\end{align*}

Now note that for the infinitesimal translations, the potential of the primed coordinates is related to the potential of the normal coordinates by a directional derivative, which can be identified with the Poisson bracket of the potential $V^*$ and each generator:

\begin{align*}
0={V^*\roundP{\bar{x}+\epsilon,\bar{y}} - V^*\roundP{\bar{x},\bar{y}}} &= \epsilon \sum_{i = 1}^3 \frac{\partial V^*\roundP{\bar{x},\bar{y}}}{\partial x_i}  = \epsilon \poisson{V^*}{T_x} = 0\\
0={V^*\roundP{\bar{x},\bar{y}+\epsilon} - V^*\roundP{\bar{x},\bar{y}}} &= \epsilon \sum_{i = 1}^3 \frac{\partial V^*\roundP{\bar{x},\bar{y}}}{\partial y_i} = \epsilon \poisson{T_y}{V^*} = 0
\end{align*}

For the infinitesimal rotation, the relation is analogous:

\begin{equation*}
0={V^*\roundP{\bar{x} +\epsilon \bar{y}, \bar{y} - \epsilon \bar{y}}-V^*\roundP{\bar{x},\bar{y}}}
= \frac{\partial V^*\roundP{\bar{x},\bar{y}}}{\partial x_i} \roundP{\epsilon y_i} -                                \frac{\partial V^*\roundP{\bar{x},\bar{y}}}{\partial y_i} \roundP{\epsilon x_i}                                        = \epsilon \poisson{V^*}{R_z}
\end{equation*}

Therefore, we conclude that the generators of translations and rotations in the plane commute with the potential $V^*$ due to its symmetries. Besides, the generator of rotations is multiple of the harmonic-like part of the guiding centres Hamiltonian which validates that $R_z$ is other integral in involution. The generators of translations are not integrals in involution, for they do not commute with the harmonic potential, however, we can calculate a quantity in terms of these generators, which already commute with the potential $V^*$, to make it commute with the remaining part of $H_{gc}$: 

\begin{equation}
L = T_x^2 + T_y^2
\end{equation}

This new quantity $L$ clearly commutes with the potential $V^*$ because the Poisson bracket is  a linear differential operator in one component and it obeys the Leibniz rule. Moreover, it also commutes with the rotation generator $R_z$:

\begin{align*}
\poisson{T_x^2+ T_y^2}{ R_z} &= \sum_{i,j,k} \poisson{x_ix_j + y_iy_j}{x_k^2 + y_k^2}\\
&= \sum_{i,j,k}  \poisson{x_ix_j}{y_k^2} + \poisson{y_iy_j}{x_k^2} = \sum_{i,j,k} y_k \poisson{x_ix_j}{y_k} + x_k \poisson{y_iy_j}{x_k}\\
&= \sum_{i,j,k} y_kx_i\delta_{jk}+y_kx_j\delta_{ik} - x_ky_i\delta_{jk}- x_ky_j\delta_{ik}\\
&= \sum_{i,j} y_jx_i + y_ix_j - x_jy_i - x_iy_j = 0
\end{align*}

As we found $L$ as the last integral in involution, we conclude, by the Liouville-Arnol'd theorem, that the subsystem of guiding centres is integrable by quadratures.\\

\section{Analysis of the motion}










 % Innroduction

\chapter{The three body problem in the plane}

In this chapter a classic approach of a somehow general case of the three body problem in 2 dimensions is going to be presented. This problem was developed by Alonso Botero et. al in \cite{alonso}. It will give some necessary intuition to develop the analogous problem in the spherical geometry. To begin with, the problem is going to be described in great detail; then its integrability is going to be proven; and finally, a formalism to describe the movement of the particles is going to be presented.\\

\section{The definition of the problem}

The three body problem presented here is that of three particles of electrical charge $e$ and mass $m$ confined to a plane, under the influence of a strong magnetic field perpendicular to it, and forces whose potentials satisfy translational and rotational symmetries in the plane.\\

Given this information, the Hamiltonian associated with this system has the form:

\begin{equation}
H = \sum_{i=1}^{3} \frac{1}{2m} \norm{ \vec{p_i} - 
e\vec{ A } \left( \vec{q_i} \right)}^2
+ V \roundP{ \vec{q_1},\vec{q_2},\vec{q_3} }
+\frac{\omega_c^2}{2m}\sum_{i=1}^{3} \norm{\vec{q_i}}^2
\label{eq:ham2d}
\end{equation}

Where $\vec{q_i} = x_i \hat{\imath} + y_i \hat{\jmath}$, $\vec{p_i} = {p_x}_i\hat{\imath} + {p_y}_i\hat{\jmath}$ and $\vec{A}\roundP{\vec{q}}$ is the magnetic vector potential, which satisfies $\nabla \times \vec{A} = B\hat{k}$.\\

Besides, the potential $V \roundP{ \vec{q_1},\vec{q_2},\vec{q_3} }$ satisfies the symmetries:

\begin{equation}
V \roundP{ R\vec{q_1}+\vec{a},R\vec{q_2}+\vec{a},R\vec{q_3}+\vec{a}  }= V \roundP{ \vec{q_1},\vec{q_2},\vec{q_3} }
\label{eq:vsym}
\end{equation}

For any rotation $R$ and translation $\vec{a}$ in the plane.\\

\section{The canonical transformation of the guiding centres}

For the proof of integrability for this system, and for further analysis of the trajectories of the particles, let us perform the well known transformation of the guiding centres.\\

This transformation is defined by the following two equations:

\begin{equation}
\vec{\pi_i} = \vec{p_i} - e\vec{A}\roundP{\vec{q_i}}
\label{eq:ct2d1}
\end{equation}

\begin{equation}
\vec{R_i} = \vec{q_i} - \frac{\hat{k}\times\vec{\pi_i}}{eB}
\label{eq:ct2d2}
\end{equation}

The equation \eqref{eq:ct2d1} passes from the canonical momentum $\vec{p_i}$ to the linear momentum $\vec{\pi_i}$, which is much more intuitive and understandable; while the equation \eqref{eq:ct2d2} transforms the general position  $\vec{q_i}$ to the position of the instantaneous guiding centre $\vec{R_i}$.\\
% Sera carreta?

In a system without the interaction potentials, the electrically charged particles are known to perform the circular motion of the cyclotron with radii that depends on the initial linear momenta. In this case, the guiding centres would be constant in time as would be the linear momenta. However, with the introduction of an interacting potential, the momentum of each particle may vary making the guiding centre change too, which is why the instantaneous interpretation of the guiding centres is necessary.\\

Now, let us calculate the Poisson brackets for this new set of coordinates in a specific particle.

\begin{align*}
\poisson{\pi_1}{\pi_2} &= \frac{\partial \pi_1}{\partial q_{\alpha}}\frac{\partial \pi_2}{\partial p_{\alpha}} - \frac{\partial \pi_2}{\partial q_{\alpha}}\frac{\partial \pi_1}{\partial p_{\alpha}}\\
&= -e\delta_{\alpha 2}\frac{\partial A_1}{\partial q_{\alpha}} +e\delta_{\alpha 1}\frac{\partial A_2}{\partial q_{\alpha}}\\
&= -e\frac{\partial A_1}{\partial q_{2}}  + e\frac{\partial A_2}{\partial q_{1}}\\
&= e(\nabla \times \vec{A})_3 = eB
\end{align*}
\begin{align*}
\poisson{R_1}{R_2} &= \poisson{q_1}{q_2} + \poisson{q_1}{-\frac{\pi_1}{eB}} +\poisson{\frac{\pi_2}{eB}}{q_2} + \poisson{\frac{\pi_2}{eB}}{-\frac{\pi_1}{eB}}\\
&= \frac{1}{eB}\roundP{\cancelto{-1}{\poisson{p_1}{q_1}}- \cancelto{0}{e\poisson{A_1}{q_1}}+\cancelto{-1}{\poisson{p_2}{q_2}} \cancelto{0}{e\poisson{A_2}{q_2}}} +\frac{eB}{(eB)^2}\\
&= \frac{-2}{eB}+ \frac{1}{eB} = -(eB)^{-1}
\end{align*}
\begin{align*}
\poisson{R_{1}}{\pi_{2}} &= \poisson{q_1}{\pi_2}+\cancelto{0}{\poisson{\frac{\pi_2}{eB}}{\pi_2}}\\
&= \cancelto{0}{\poisson{q_1}{p_2}}-e\cancelto{0}{\poisson{q_1}{A_2}}\\
&= \poisson{R_2}{\pi_1} = 0
\end{align*}

This Poisson brackets can be generalised to the transformation for the three particles. Taking $i,j = \poisson{1,2}{3}$ and $\alpha,\beta=\poisson{1}{2}$:

\begin{equation}
\poisson{\pi_{i,\alpha}}{\pi_{j,\beta}}=\roundP{eB}\delta_{ij}\epsilon_{\alpha \beta}  
\label{eq:pb1}
\end{equation}

\begin{equation}
\poisson{R_{i,\alpha}}{R_{j,\beta}}= -\roundP{eB}^{-1} \delta_{ij}\epsilon_{\alpha \beta}  
\label{eq:pb2}
\end{equation}

\begin{equation}
\poisson{R_{i,\alpha}}{\pi_{j,\beta}}=0 
\label{eq:pb3} 
\end{equation}

Equations \eqref{eq:pb1}-\eqref{eq:pb2} allow us to identify the proposed transformation as canonical. However, this is not the usual canonical transformation where the position coordinates and the momentum coordinates are canonical conjugates. In this special case, one component of the momentum is canonical conjugate with the other momentum coordinate, as equally happens for the position coordinates.\\

Now, with a huge magnetic field, if the potential of the interaction forces does not vary abruptly in space, we can use the approximation $\vec{R}_i \approx \vec{q}_i$ to average the potentials over the guiding centres, that is, we can replace $\vec{q}_i$ for $\vec{R}_i$ in  $V \roundP{ \vec{q_1},\vec{q_2},\vec{q_3} }$.\\

We can support the last approximation as follows: In the cyclotron problem, the radius of the circular motion described is proportional to the linear momentum and inversely proportional to the magnetic field. Then, in the presence of a big $B$, the radius of the cyclotron would shrink to a very small size. Regarding the case we are working with, the radii of the instantaneous cyclotron motion would be proportional to $\norm{(\hat{k} \times \vec{\pi_i})(eB)^{-1}}$ and its frequency to $\sqrt{B}$. As the potential $V$ does not vary abruptly in the radii scale, the averaging of this motion over the guiding centres means that this potential does not sense that circular motion. Moreover, given the big frequency of the cyclotrons and the scale of variance of the potential, the scale of time of the local circular motions is far smaller than that of the motion of the guiding centres. Therefore, we can ignore the instantaneous quality of the circular motion, and take it as constant in a scale of time small enough for the motion of the guiding centres. In this sense we say that the coordinates for the guiding centres decouple from that of the linear momenta of the particles.\\

Before replacing the new set of coordinates in the Hamiltonian, it is necessary to do a scale transformation to obtain the proper Poisson brackets for the formal definition of canonical transformation, that is:

\begin{align*}
\vec{\pi}_i & \rightarrow (eB)^{-1/2} \vec{ \pi}_i\\
\vec{R}_i & \rightarrow \sqrt{eB} \vec{R}_i
\end{align*}

With this consideration, the Hamiltonian of the system in the new set of rescaled coordinates is given by:

\begin{equation}
H = \sum_{i=1}^{3} \frac{eB}{2m} \norm{ \vec{\pi}_i}^2
+ V\roundP{ (eB)^{-1/2}\vec{R_1},(eB)^{-1/2}\vec{R_2}, (eB)^{-1/2}\vec{R_3} }
+\frac{\omega_c^2}{2meB}\sum_{i=1}^{3} \norm{\vec{R}_i}^2
\label{eq:newham}
\end{equation}

This Hamiltonian, given equation \eqref{eq:pb3} can be decomposed in a Hamiltonian that describes the movement of the guiding centres, and other that describes de movement of the linear momenta. In one hand, the Hamiltonian for the linear momenta is easily identified with the harmonic oscillator, whereas the one that characterises the movement of the guiding centres needs a deeper analysis.\\

\section{Integrability of the system}
As the Hamiltonian describing the trajectories of the linear momenta of the particles is that of an harmonic oscillator, this part of the problem is integrable and its solutions are widely known. The guiding centre Hamiltonian, in turn, needs to be analysed more deeply. For this purpose, let us take the following convention:\\


\begin{equation}
H_{gc} = \frac{{\omega_c^*}^2}{2m} \sum_{i=1}^{3} \norm{\bar{x}^2} + \norm{\bar{y}^2}
+ V^*\roundP{\bar{x},\bar{y}}
\label{eq:hamgc}
\end{equation}

Where $\bar{x} = \roundP{x_1,x_2,x_3}$ and $\bar{y} = \roundP{y_1,y_2,y_3}$, being $x_i,y_i$ the rescaled coordinates of the guiding centres of the particles. For simplicity, the potential $V$ and the constant $\omega_c$ have been rescaled to take into account the scale transform of the coordinates and maintain the original form of the Hamiltonian:

\begin{align*}
&\omega^* = \frac{\omega}{\sqrt{eB}}\\
&V^*\roundP{\bar{x},\bar{y}} = V\roundP{\frac{\bar{x}}{\sqrt{eB}},\frac{\bar{y}}{\sqrt{eB}}}
\end{align*}

Clearly, the new potential $V^*$ still has the symmetries expressed in the equation \eqref{eq:vsym}. Furthermore, in the new order for the scaled guiding centres coordinates, the Poisson brackets take the form:

\begin{equation}
\poisson{y_i}{x_j} = \delta_{ij}
\label{lastpb}
\end{equation}

Now that the guiding centres Hamiltonian has been expressed in terms of the proper canonical set of coordinates, the fastest way to prove the integrability of the system is via the Liouville-Arnol'd theorem \cite[Sect. 49]{arnold}. For this theorem, it is only necessary to find 2 more independent integrals in involution (besides the Hamiltonian).\\

To get this 2 integrals, let us exploit the symmetries of the guiding centres Hamiltonian. We then define the generators of translations and rotation in the plane, which are symmetries of the potential:

\begin{equation}
\begin{aligned}
T_x &= \sum_{i=1}^{3} x_i \\
T_y &= \sum_{i=1}^{3} y_i 
\end{aligned}
\label{eq:gentranslation}
\end{equation}


\begin{equation}
R_z = \frac{1}{2} \sum_{i=1}^{3} \roundP{x_i^2 + y_i^2}
\label{eq:genrotation}
\end{equation}

It is easily verifiable that these are indeed the symmetries generators. To see that, take the first order infinitesimal transformations of translation and rotation:

\begin{equation*}
x_i  \rightarrow x_i + \epsilon 
\end{equation*}

\begin{equation*}
y_i \rightarrow y_i + \epsilon 
\end{equation*}

\begin{equation*}
\roundP{x_i,y_i} \rightarrow \roundP{x_i +\epsilon y_i, y_i - \epsilon x_i}
\end{equation*}

Now note that for the infinitesimal translations, the potential of the primed coordinates is related to the potential of the normal coordinates by a directional derivative, which can be identified with the Poisson bracket of the potential $V^*$ and each generator:

\begin{align*}
0={V^*\roundP{\bar{x}+\epsilon,\bar{y}} - V^*\roundP{\bar{x},\bar{y}}} &= \epsilon \sum_{i = 1}^3 \frac{\partial V^*\roundP{\bar{x},\bar{y}}}{\partial x_i}  = \epsilon \poisson{V^*}{T_x} = 0\\
0={V^*\roundP{\bar{x},\bar{y}+\epsilon} - V^*\roundP{\bar{x},\bar{y}}} &= \epsilon \sum_{i = 1}^3 \frac{\partial V^*\roundP{\bar{x},\bar{y}}}{\partial y_i} = \epsilon \poisson{T_y}{V^*} = 0
\end{align*}

For the infinitesimal rotation, the relation is analogous:

\begin{equation*}
0={V^*\roundP{\bar{x} +\epsilon \bar{y}, \bar{y} - \epsilon \bar{y}}-V^*\roundP{\bar{x},\bar{y}}}
= \frac{\partial V^*\roundP{\bar{x},\bar{y}}}{\partial x_i} \roundP{\epsilon y_i} -                                \frac{\partial V^*\roundP{\bar{x},\bar{y}}}{\partial y_i} \roundP{\epsilon x_i}                                        = \epsilon \poisson{V^*}{R_z}
\end{equation*}

Therefore, we conclude that the generators of translations and rotations in the plane commute with the potential $V^*$ due to its symmetries. Besides, the generator of rotations is multiple of the harmonic-like part of the guiding centres Hamiltonian which validates that $R_z$ is other integral in involution. The generators of translations are not integrals in involution, for they do not commute with the harmonic potential, however, we can calculate a quantity in terms of these generators, which already commute with the potential $V^*$, to make it commute with the remaining part of $H_{gc}$: 

\begin{equation}
L \coloneqq \frac{1}{6}\roundP{T_x^2 + T_y^2}
\label{eq:gccircle}
\end{equation}

This new quantity $L$ clearly commutes with the potential $V^*$ because the Poisson bracket is  a linear differential operator in one component and it obeys the Leibniz rule. Moreover, it also commutes with the rotation generator $J \coloneqq R_z$:

\begin{align*}
\poisson{T_x^2+ T_y^2}{ J} &= \sum_{i,j,k} \poisson{x_ix_j + y_iy_j}{x_k^2 + y_k^2}\\
&= \sum_{i,j,k}  \poisson{x_ix_j}{y_k^2} + \poisson{y_iy_j}{x_k^2} = \sum_{i,j,k} y_k \poisson{x_ix_j}{y_k} + x_k \poisson{y_iy_j}{x_k}\\
&= \sum_{i,j,k} y_kx_i\delta_{jk}+y_kx_j\delta_{ik} - x_ky_i\delta_{jk}- x_ky_j\delta_{ik}\\
&= \sum_{i,j} y_jx_i + y_ix_j - x_jy_i - x_iy_j = 0
\end{align*}

As we found $L$ as the last integral in involution, we conclude, by the Liouville-Arnol'd theorem, that the subsystem of guiding centres is integrable by quadratures.\\

\section{Analysis of the motion}
Taking the motion integrals obtained in the previous section for the guiding center Hamiltonian \eqref{eq:hamgc} the motion of the particles can be broken down in that of the center of mass, and the relative trajectories of the particles. To see that, note that as $T_x^2 + T_y^2$ in \eqref{eq:gccircle} is an integral in involution with the Hamiltonian, it is a conserved quantity. The conservation of this term is clearly interpreted as a circular motion of the center of mass of the system $\roundP{\frac{T_x}{3},\frac{T_y}{3}}$.\\

The latter analysis can be clarified in terms of the decoupling of the guiding center trajectories from the general motion of the particles. In that sense one can deduce that the guiding centre movement is only affected by the interaction potential $V$ and the external harmonic potential, which can be confirmed by the lack of magnetic terms in the guiding centre Hamiltonian in equation \eqref{eq:newham}. Taking this into account, as the potential $V$ is a central potential that produces only internal forces, one can assure that the motion of the center of mass of the guiding centres is determined by the only term of external interaction in the Hamiltonian, producing, as consequence, the deduced circular motion of the center of mass.\\

The problem is then reduced to the analysis of the motion of the coordinates relative to the centre of mass. To carry on with this procedure, let us take the spinor of relative position defined as follows:

\begin{align}
\Psi_\alpha = \frac{1}{2\sqrt{3}}\begin{pmatrix}\sqrt{3}\roundP{z_{2}-z_{1}}\\
z_{2}+z_{1}-2z_{2}\end{pmatrix}
\end{align}

With $z_{i} = x_i + iy_i$. If we call define the center of mass as $T = \frac{1}{3}\roundP{T_x + iT_y} = \frac{1}{3}\roundP{z_1 +z_2+z_3}$ we can retrieve the general position of each particle as follows:
\small
\begin{align*}
z_1 &= T+\frac{1}{\sqrt{3}}\Psi_2 - \frac{1}{3}\Psi_1 \\
z_2 &= T+\frac{1}{\sqrt{3}}\Psi_2 + \frac{1}{3}\Psi_1 \\
z_3 &= T-\frac{2}{\sqrt{3}}\Psi_2
\end{align*}
\normalsize

The spinor components were chosen to satisfy the canonical commutation relations:

\small
\begin{align*}
\poisson{\Psi_\alpha}{\Psi_\alpha} &= 0\\ 
\poisson{\Psi_1}{\Psi_2} &= \poisson{\Psi_1^*}{\Psi_2^*}^* = \poisson{\frac{1}{2}\roundP{z_{2}-z_{1}}}{\frac{1}{2\sqrt{3}}\roundP{z_{2}+z_{1}-2z_{3}}} \\
&= i\poisson{\frac{1}{2}\roundP{x_{2}-x_{1}}}{\frac{1}{2\sqrt{3}}\roundP{y_{2}+y_{1}-2y_{3}}}\\
&+ i\poisson{\frac{1}{2}\roundP{y_{2}-y_{1}}}{\frac{1}{2\sqrt{3}}\roundP{x_{2}+x_{1}-2x_{3}}} = 0\\
\end{align*}
\small
\begin{align*}
\poisson{\Psi_1}{\Psi_2^*} &= \poisson{\Psi_1^*}{\Psi_2}^* = -\frac{i}{2\sqrt{3}}\poisson{x_{2}-x_{1}}{y_{2}+y_{1}-2y_{3}} = 0\\
\poisson{\Psi_1}{\Psi_1^*} &= \frac{-2i}{4}\poisson{x_{2}-x_{1}}{y_2-y_1} = -i\\
\poisson{\Psi_2}{\Psi_2^*} &= \frac{-i}{6}\poisson{x_{1}+x_{2}-2x_3}{y_1+y_2-2y_3} = -i
\end{align*}
\normalsize

Which can be resumed:

\begin{align*}
\poisson{\Psi_\alpha}{\Psi_\beta} &= \poisson{\Psi_\alpha^*}{\Psi_\beta^*} = 0\\
\poisson{\Psi_\alpha*}{\Psi_\beta} &= i\delta_{\alpha,\beta}
\end{align*}

This spinorial representation of the relative coordinates of the particles with respect to the center of mass is analogous to the transformation performed to analyse the two body problem, only adapted to take into account one more particle and one less dimension. This spinor, besides representing a canonical transformation of the guiding centres, has norm equal to the spin quantity $S = J-L$ which is also an integral of motion:

\begin{align*}
\Psi^\dagger\Psi = \frac{1}{4}\roundP{x_1^2 + x_2}
\end{align*} % Background Theory 

\chapter{The problem of a charged particle in the magnetic field of a monopole}

With the objective to obtain some intuition about the N-body problem restricted to a spherical geometry, we study in this chapter the symmetries and trajectories of a charged particle under the influence of the magnetic field of a monopole. To achieve this we present the deduction, via the Lagrangian formalism, of the so called Poincar\'e cone \cite{poincare} that characterises the trajectory of a particle in this situation. We then extrapolate the important symmetries used in the Lagrangian formalism to the Hamiltonian formalism to retrieve some important aspects of the classical counterparts of the known Haldane formalism \cite{haldane}.\\

\section{Definition of the Lagrangian}
Let $L$ be the Lagrangian of a charged particle of charge $-e$ and mass $m$ under the influence of a magnetic monopole of magnitude $g$. Then $L$ takes the form:

\begin{equation}
L\roundP{\vec{x},\dot{\vec{x}}} = \frac{m}{2}\norm{\dot{\vec{x}}}^2 - e\vec{A}_{\hat{u}}(\vec{x})\cdot\dot{\vec{x}}
\label{eq:lagrangian}
\end{equation}

Where $\vec{A}(\vec{x})$ is the vector potential of the magnetic monopole with singularity along the direction defined by the unit vector $\hat{u}$. It is know that there is no vector potential that is finite in $\mathbb{R}^3$ which reproduces the magnetic field of a monopole; in addition, the different vector potentials identified by different unit vectors $\hat{u}$ are related by a gauge transformation which leaves the trajectories invariant. This family of vector potentials are given by \cite{poincare}:

\begin{equation}
\vec{A}_{\hat{u}}(\vec{x}) = \frac{g}{r}\frac{\hat{u}\times\hat{r}}{1+\hat{u}\cdot\hat{r}}
\label{eq:monopolepotential}
\end{equation}

\section{The symmetries and its conserved quantities}

The first thing to note in the previously defined Lagrangian is its time independence, which yields to the conservation of the Jacobi integral:

\begin{equation*}
 \frac{\partial L}{\partial\dot{\vec{x}}}\cdot\dot{\vec{x}} - L = m\norm{\dot{\vec{x}}}^2                    =m\norm{\dot{\vec{x_0}}}^2 
\end{equation*}

With $\dot{\vec{x_0}}$ the initial velocity of the particle. Here the Jacobi integral clearly represents the kinetic energy of the particle, which is constant in time because of the linear dependence of the term associated with the magnetic interaction, that basically means that the magnetic fields do no work.\\

Now, due to the simplicity of the problem, one can suspect of many other symmetries. The next symmetry presented here is not associated with Lagrangian, but rather with the action invariance. If we define the action as in the equation \refeq{eq:action}, it can be seen that it may be invariant under a proper scale transform of position and time. It is not difficult to find this transform, and it is presented in equation \refeq{eq:scaletrans}.\\

\begin{equation}
S = \int_{t_1}^{t_2}L\roundP{\vec{x},\dot{\vec{x}}}dt = \int_{t_1}^{t_2}\roundP{\frac{m}{2}\norm{\dot{\vec{x}}}^2 - e\vec{A}_{\hat{u}}(\vec{x})\cdot\dot{\vec{x}}}dt 
\label{eq:action}
\end{equation}

\begin{equation}
\begin{aligned}
\vec{x}' &= e^{s}\vec{x}\\
t'&= e^{2s}t
\end{aligned}
\label{eq:scaletrans}
\end{equation}

As a result of this symmetry, by Noether's theorem there must be a conserved quantity implied. To calculate it we prefer the method stated in \cite[Thm (dontforget)]{scheck}; however, to apply this method, the Lagrangian must be parametrised to include the time $t$ as a generalised coordinate. To achieve this, note that:

\begin{align*}
\dot{\vec{x}} = \frac{d\vec{x}}{dt} = \frac{\frac{d\vec{x}}{d\tau}}{\frac{dt}{d\tau}} \coloneqq \frac{\mathring{\vec{x}}}{\mathring{t}}
\end{align*}

Then the action can be written in terms of the new parameter $\tau$:

\begin{equation*}
S = \int_{t_1}^{t_2}L\roundP{\vec{x},\dot{\vec{x}}}dt = \int_{\tau_1}^{\tau_2}\mathring{t}L\roundP{\vec{x},\mathring{\vec{x}}{\mathring{t}}^{-1}}d\tau                                                                    = \int_{\tau_1}^{\tau_2}L'\roundP{\vec{x},\mathring{\vec{x}},\mathring{t}}d\tau
\label{eq:action}
\end{equation*}

Which by analogy with equation \refeq{eq:action}, gives the new parametrised Lagrangian $L'$:

\begin{equation*}
L'\roundP{\vec{x},\mathring{\vec{x}},\mathring{t}} = \mathring{t}L\roundP{\vec{x},\mathring{\vec{x}}{\mathring{t}}^{-1}} = \roundP{\frac{m}{2\mathring{t}}\norm{\mathring{\vec{x}}}^2 - e\vec{A}_{\hat{u}}(\vec{x})\cdot\mathring{\vec{x}}}
\end{equation*}

As we converted the symmetry of the action $S$ in a symmetry of the Lagrangian $L'$, the invariance given by  equation\refeq{eq:scaletrans} becomes clear. Now, using \cite[Thm (dontforget)]{scheck}, we obtain the conserved quantity in accordance with the transformation \refeq{eq:scaletrans}: \\

\begin{align*}
\nonumber
G &= \frac{\partial L'}{\partial \mathring{q_i}} \frac{\partial q'_i}{\partial s} \rvert_{s=0}\\ \nonumber
\\ \nonumber
G &= -m\norm{\frac{\mathring{\vec{x}}}{\mathring{t}}}^2t + \roundP{\frac{m\mathring{\vec{x}}}{\mathring{t}} - e\vec{A}_{\hat{u}}}\cdot \vec{x}\\ \nonumber
\\ \nonumber
G &= m\dot{\vec{x_0}}\cdot\vec{x_0} = -m\norm{\dot{\vec{x_0}}}^2t + m\dot{\vec{x}}\cdot\vec{x} \\\nonumber
\end{align*}

Working the previous conserved quantity one can obtain an equation for the magnitude of the position in function of time:

\begin{align}
\nonumber
 2\dot{\vec{x}}\cdot\vec{x} &= \frac{d\norm{x}^2}{dt} = 2\norm{\dot{\vec{x_0}}}^2t + 2\dot{\vec{x_0}}\cdot\vec{x_0}\\ \nonumber
 \\ 
 x^2 &= x_0^2 +2\dot{\vec{x_0}}\cdot\vec{x_0}t +\norm{\dot{\vec{x_0}}}^2t^2  
 \label{eq:radius}
\end{align}

From equation \ref{eq:radius} it is important to note that the only way the radius of the particle stays constant is that the initial velocity of the particle is $0$, otherwise, the time-quadratic term will always contribute to the increase in the radius. This is in fact different from the analogous case of the cyclotron, where one can obtain closed trajectories with constant radius.\\

On the other hand, we note that once chosen a unit vector $\hat{u}$ for the vector potential  $\vec{A}_{\hat{u}}$, it is quite obvious that the original Lagrangian $L$ in equation \refeq{eq:lagrangian} is invariant under rotations around the unit vector $\hat{u}$, that is, under the following infinitesimal transformation:

\begin{equation}
\vec{x}' = \vec{x} + s\roundP{\hat{u}\times\vec{x}}
\label{eq:rotinv}
\end{equation}

Then, as this transformation does not include the time $t$, we can perform the calculation of the conserved quantity over the original Lagrangian:

\begin{align*}
G_2 &= \frac{\partial L}{\partial \dot{\vec{x}}}\cdot\frac{\partial \vec{x}}{\partial s} \rvert_{s=0}\\
G_2 &= \roundP{m\dot{\vec{x}} - e\vec{A}_{\hat{u}}}\cdot \roundP{\hat{u}\times\vec{x}} \\
G_2 &= \roundP{m\vec{x}\times\dot{\vec{x}}}\cdot\hat{u}-eg\frac{\norm{\hat{u}\times\hat{x}}^2}{\roundP{1+\hat{x}\cdot\hat{u}}}\\
G_2 &= \roundP{m\vec{x}\times\dot{\vec{x}}}\cdot\hat{u} -eg\roundP{1-\hat{x}\cdot\hat{u}}\\
J_{\hat{u}} &\coloneqq \roundP{\roundP{m\vec{x}\times\dot{\vec{x}}}+ eg\hat{x}}\cdot\hat{u} = constant
\end{align*}

Now, the last argument is valid for any unit vector $\hat{u}$ chosen and this yields the conservation of the  known as Poincar\'e vector in equation \refeq{eq:poincarevec}.

\begin{equation}
\vec{J} = \roundP{m\vec{x}\times\dot{\vec{x}}}+ eg\hat{x}
\label{eq:poincarevec}
\end{equation}

This last symmetry is very meaningful because it restricts the trajectory of the particle to a cone centred in the origin with central vector $\hat{J}$. To verify this, it is only necessary the radial component of the Poincar\'e vector is constant for all points in the trajectory, which means that the angle between $\vec{J}$ and $\vec{x}(t)$ is a constant and that the trajectory of the particle is restricted to a cone:

\begin{align*}
\vec{J}\cdot\hat{x} = eg
\end{align*}

Furthermore, we can deduce from the conservation of the Poincar\'e vector that the angular momentum of the particle is constant in magnitude and that it determines the aperture of the cone of restriction:





 % Experimental Setup

\chapter{The three body problem in the sphere}

\section{Definition of the problem}


\section{Integrability of the system}


\section{Analysis of the movement of the guiding centres}


 % Experiment 1

\chapter{The three body problem in the sphere}
Given the introduction about the one-body problem under the influence of a monopolar magnetic field, we are now able to generalize the three-body problem restricted to the sphere. As we did in chapter 2 and 3, we would like to search for a similar analysis for the three-body problem on the sphere on huge magnetic field regimen. As we will discuss later in this chapter, an analogous analysis is not trivial, however, we explain some important aspects of the problem and give some general guidelines that may, trough future work, carry to the so wanted analogous analysis.\\

This chapter is organized the following way: First, the problem is going to be formally defined. Then, a brief discussion of the integrability in spherical and Cartesian coordinates and some disadvantages of these coordinate systems are going to be presented. Finally, the stereographic projection, as an alternative coordinate system to analyse similarities is going to be introduced.\\

\section{Definition of the problem}
In this problem, we practically have three Hamiltonians \eqref{eq:hamiltonian} of free particles in a monopolar magnetic field, coupled by a potential $V$ with the three-dimensional rotational symmetries. To make things easier, we choose the same gauge for the vector potentials associated with each particle. In this terms, the three-body Hamiltonian will have the form:

\begin{align}
H &= \left. \frac{1}{2m}\roundP{\norm{\vec{\pi}_1}^2+\norm{\vec{\pi}_2}^2+\norm{\vec{\pi}_3}^2} + V\roundP{\vec{r}_1,\vec{r}_2,\vec{r}_3}    \right|_{S^2} \nonumber \\
&= \left. \frac{1}{2m}\roundP{\norm{\vec{p}_1+e\vec{A}_{\hat{u}}(\vec{r}_1)}^2+\norm{\vec{p}_2+e\vec{A}_{\hat{u}}(\vec{r}_2)}^2+\norm{\vec{p}_3+e\vec{A}_{\hat{u}}(\vec{r}_3)}^2} + V\roundP{\vec{r}_1,\vec{r}_2,\vec{r}_3}    \right|_{S^2}
\label{eq:3sphereHam}
\end{align}

Where $\vec{\pi}_i = \vec{p}_i+e\vec{A}_{\hat{u}}(\vec{r}_i)$ is the linear momentum of each particle,  $\vec{A}_{\hat{u}}(\vec{r})$ is defined in equation \eqref{eq:monopolepotential}, and the potential $V$ satisfies:

\begin{equation*}
 V\roundP{R\vec{r}_1,R\vec{r}_2,R\vec{r}_3} =  V\roundP{\vec{r}_1,\vec{r}_2,\vec{r}_3} 
\end{equation*}

For any three-dimensional rotation.\\

\section{The analogous canonical transformation of the guiding centres in Cartesian coordinates}
Since the canonical coordinates of each particle are independent, we can define the Poincar\'e vector for each particle in the same way as in equation \eqref{eq:ham poincarevec} without loosing its angular momentum Poisson relations.

\begin{align*}
\vec{J}_i &= \vec{r}_i\times\vec{\pi}_i + eg\hat{r}_i\\
&= \vec{r}_i\times\roundP{\vec{p}_i+e\vec{A}_{\hat{u}}(\vec{r}_i)} + eg\hat{r}_i\\
\poisson{J_i}{J_j} &= \epsilon_{ijk}J_k
\end{align*}

Now, as seen before, $\vec{J}_i$ is the generator of rotations, in this case, of the coordinate system of each particle. To obtain the generator of rotations for the general system, we simply add up the three Poincar\'e vectors producing $\mathbb{J}$, which will also follow the angular momentum relations.\\

It was previously discussed that the Poincar\'e vector points in the direction of the guiding center of a particle. We therefore are tempted to take this vector as the analogue of the canonical guiding center transformation performed in the plane:

\begin{align}
\vec{R}_i &= \frac{\vec{J}_i}{\frac{eg}{r}} = \vec{r}_i + \frac{\hat{r}_i\times\vec{\pi}_i}{\frac{eg}{r^2}} \nonumber\\
&= \vec{r}_i + \frac{\hat{r}_i\times\vec{\pi}_i}{eB}
\label{eq:gcSphere}
\end{align}

In fact, the previous transformation \eqref{eq:gcSphere} has the same form of that of the plane on equation \eqref{eq:ct2d2}. The problem is completely analogous: in stead of $\hat{k}$ indicating the direction of the constant magnetic field on the plane, we obtain $\hat{r}$ that is actually the direction of the magnetic field of the monopole. \\

This can get things complicated as $\vec{r}$ is not constant in time, but points to the position of the particle on the sphere at any time. This situation is reflected in the fact that this transformation is not canonical, but it follows the relations of angular momentum with the exemption of constants, namely $\frac{eg}{r}$.\\

Anyway, given the relation on equation \eqref{eq:gcSphere} in the big magnetic field regime we can also make the assumption that the guiding center is approximately equal to the position of the particle, simplifying the problem in a similar way. The next step on this problem would be to find the analogue to the linear momentum that characterises the radius of the guiding center and the position in the circle. As the guiding center variables follow the relations of angular momentum, it would be ideal to find this analogue quantity to follow the same relations.\\

However, the principal idea of doing the guiding center transformation is to obtain some separated canonical quantities with null Poisson brackets with the objective of separating the former Hamiltonian into two decoupled problems under a big magnetic field $B$. Then, while it is ideal to find the complementary quantity of the spherical guiding center to satisfy the angular momentum relations, we would be okay with it not necessarily being an angular momentum but unavoidably satisfying the null Poisson brackets with its guiding center counterpart. Besides, in analogy with the problem on the plane, we require the kinetic part of the Hamiltonian to be a function of these new complementary quantities or at least to be separable into some function of the guiding centres and other of the remaining coordinates.\\

One can try to find a vector which follows this condition. However, a little bit of thinking tells us that it can not be found at least in the Cartesian coordinate system. The reason is that, as it was demonstrated before, the guiding center transformation produces three generators of rotations of the particular coordinate system. In these terms, no matter what quantities we define, if they are vectors we automatically deduce that the Poisson bracket between this quantity and its respective guiding center coordinate is not null, but represents the infinitesimal rotation of this vector.\\

We can propose a transformation that encodes the counterpart of the guiding center coordinates in some scalar quantities so that the Poisson bracket with the guiding centres become null, and we can then perform the separation of the problem into two different decoupled Hamiltonians. The finding of these quantities is not easy and anyway would have not helped much in the development of a sphere formalism similar to that of the plane. The problem resides in the characterisation of the guiding center coordinates as the rotation generator in the Cartesian coordinate system, which impedes a proper separation of variables.\\

To see this clearly, let us assume that we found some proper non-vectorial quantities $\{\Lambda_i\}$ that commute with the quantities $\{\vec{R}_i\}$,and which generate the kinetic part of the Hamiltonian. We then can separate this Hamiltonian into the guiding centres Hamiltonian and other for $\{\Lambda_i\}$. The remaining Hamiltonian would be simply the potential $V$ averaged over the guiding centres and properly rescaled:

\begin{align*}
H_{gc} &= V(\vec{J_1},\vec{J_2},\vec{J_3})\\
\poisson{J_{\alpha,i}}{J_{\beta,j}} &= \delta_{\alpha\beta}\epsilon_{ijk} J_{\alpha,k}
\end{align*}

Where Greek indices label the particle and Latin indices the coordinate. The next step here would be to find some transformation that decouples the movement of the center of mass of guiding centres from the relative coordinates. However, the relative coordinates are vectorial quantities and would definitely not commute with the center of mass coordinates, nor between them. Moreover, the subtraction of angular momentum is not necessarily another angular momentum. \\

We would try the same argument as before of finding some scalar quantities to decouple the center of mas from the relative coordinates. Nonetheless, we would loose the coordinate-like transformation and would not be able to easily relate the quantities with the characteristics of the triangle.\\

The issue with Cartesian coordinates, besides the principal quantities being the generators of rotations, is that there is no simple way to introduce the restriction of the coordinates to the sphere. We will always be working with more coordinates than we need. Taking this into account, an obvious alternative to the solution of this problem is the use of spherical coordinates, where we would get rid of the radial coordinate and momentum, obtaining a problem properly described by two coordinates.\\ 

\section{The problem in spherical coordinates}
We redefine then the problem in spherical coordinates to verify if this is a suitable set of quantities to carry out a similar analysis than that of the plane. To do this let us start with the Lagrangian of a single particle in the magnetic field of a monopole restricted to the spherical surface. But first let us define the spherical coordinate transformation and some important relations \cite{sphericalcoordinates}:

\begin{align*}
x &= r\sin{\theta}\cos{\phi} & \hat{r} = \frac{\partial \vec{r}}{\partial r}\bigg/\norm{\frac{\partial \vec{r}}{\partial r}}\\
y &= r\sin{\theta}\sin{\phi} & \hat{\theta} = \frac{\partial \vec{r}}{\partial \theta}\bigg/\norm{\frac{\partial \vec{r}}{\partial \theta}} \\
z &= r\cos{\theta} & \hat{\phi} = \frac{\partial \vec{r}}{\partial \phi}\bigg/\norm{\frac{\partial \vec{r}}{\partial \phi}}\\
\hat{r}&\times \hat{\theta} = \hat{\phi}
\end{align*}

We begin with the definition of the  linear velocity of a single particle in terms of the angular coordinates $(\theta,\phi)$ and their unit vectors $(\hat{\theta},\hat{\phi})$, which is not difficult to deduce with some algebra:

\begin{equation}
\dot{\vec{r}} = r\dot{\theta}\hat{\theta} + r\sin{\theta}\dot{\phi}\hat{\phi}
\end{equation}

Now, transforming the vector potential with the convenient gauge along the $z$ axis, we obtain:

\begin{equation*}
\vec{A}_{\hat{k}} = \frac{eg}{r}\tan{\frac{\theta}{2}}\hat{\phi}
\end{equation*}

As these unit vectors are orthogonal we obtain the following Lagrangian:\\

\begin{equation*}
L = \frac{mr^2}{2}\roundP{\dot{\theta}^2 + \sin^2\theta \dot{\phi}^2} + \frac{eg}{r}\tan{\frac{\theta}{2}\dot{\phi}}
\end{equation*}

With this Lagrangian, the canonical conjugate momenta are determined:

\begin{align*}
p_\theta = mr^2\dot{\theta}
p_\phi = mr^2\sin^2\theta\dot{\phi} + \frac{eg}{r}\tan{\frac{\theta}{2}}
\end{align*}

Then the Hamiltonian is determined:

\begin{equation}
H = \frac{1}{2mr^2}\roundP{p_\theta^2 + \roundP{\frac{p_\phi - \frac{eg}{r}\tan{\frac{\theta}{2}}}{\sin\theta}}^2}
\end{equation}

We can then define a linear momentum vector in analogy with this Hamiltonian and with the linear velocity previously defined:

\begin{align*}
\vec{\pi} &= \frac{p_\theta}{r}\hat{\theta} + \frac{p_\phi - \frac{eg}{r}\tan{\frac{\theta}{2}}}{r\sin\theta}\hat{\phi}\\
\pi_\theta &= \frac{p_\theta}{r}\\
\pi_\phi &= \frac{p_\phi - \frac{eg}{r}\tan{\frac{\theta}{2}}}{r\sin\theta}
\end{align*}

The purpose of this more or less long introduction to the problem in spherical coordinates was to obtain the linear momenta formula. With this formula one can calculate the Poisson bracket between the two components $(\pi_\theta,\pi_\phi)$ and realize that, given the dependence of $\pi_\phi$ on both momenta and coordinates, this Poisson bracket will have a rather complicated expression that is not worth writing down. This tells us that the counterpart for the guiding center transformation on spherical coordinates wont function at least for the linear momentum. \\

One may argument that the correct analogous of the linear momentum in the plane are the conjugate angular momenta on the sphere in stead of the same linear momentum. While this argument is right and makes a lot of sense, we find that even with this consideration we obtain a set of variables that are not much different from before and posses the same problem:

\begin{align*}
\vec{L} &= p_\theta\hat{\theta} + \roundP{p_\phi - \frac{eg}{r}\tan{\frac{\theta}{2}}}\hat{\phi}\\
L_\theta &= p_\theta\\
L_\phi &= p_\phi - \frac{eg}{r}\tan{\frac{\theta}{2}}
\end{align*}

When we calculate the Poisson brackets between these two quantities, one obtain the following result:

\begin{align*}
\poisson{L_\theta}{L\phi} &= \cancelto{0}{\poisson{p_\theta}{p_\phi}}-\poisson{p_\theta}{\frac{eg}{r}\tan{\frac{\theta}{2}}}\\
&= \frac{eg}{r}\frac{\partial \tan{\frac{\theta}{2}} }{\partial \theta}\\
&=  \frac{eg}{r} \frac{1}{1+\cos{\theta}}
\end{align*}

While this expression is very close to something simple as happened in the case of the plane, we can not characterise this transformation as canonical. The reason is that this Poisson bracket can get as big as possible when $\theta \to \pi$. If we approximate this Poisson bracket to $\frac{eg}{r}$, which is valid when $\theta \to 0$, we would obtain the so wanted canonical transformation for the momentum. However, with this approximation, we are taking a region on the north pole of the sphere as if it was a plane, and the problem on the sphere its curvature.\\

On the other hand, regarding the transformation to the guiding center angular momentum, one can simply transform the momentum $\vec{J}$ to spherical coordinates. In fact, the same rotational symmetries analysed on chapter 4 have to produce the same conserved quantity.\\

\begin{align*}
\vec{J} &= \vec{r}\times\vec{\pi} + eg\hat{r}\\
&= r\pi_\theta \cancelto{\hat{\phi}}{(\hat{r}\times \theta)} + r\pi_\phi \cancelto{-\hat{\theta}}{(\hat{r}\times \phi)} + eg\hat{r}\\
&= -r\pi_\phi \hat{\theta} + r\pi_\theta \hat{\phi} + eg\hat{r}
\end{align*}

We then discover that the guiding center angular momentum has angular components that are the same than the linear momentum, besides a constant radial component. The Poisson bracket relations do not change a little bit in comparison to that of the linear momentum components. Besides not finding a nice expression for the algebra followed by the guiding center and the linear momenta, we find in the spherical coordinates a bigger problem: The linear momentum and the guiding center cannot be properly decoupled because they are practically the same.\\

These reasons leaded us to abandon the study of the problem in spherical coordinates.\\

\section{Integrability of the system}


\section{Analysis of the movement of the guiding centres}


 % Experiment 2

%\chapter{Conclusions}

In this thesis we reproduced a very interesting and elegant approach developed by Botero et al. \cite{alonso} to analyse the three body problem in the plane related to the Hall effect. We showed that in the classical case, this problem is integrable due to the decoupling of the guiding centres and the linear momenta in the regimen of big magnetic fields. Moreover, the canonical transformation of relative coordinates can be manipulated to codify the shape of the triangle in a Bloch sphere. With this analysis, the study of the orientation and shape of the triangle can be studied naturally in terms of the dynamical and geometrical angular velocity of the system, which are in fact the angular variables in the angle-action formalism.\\

In the quantum counterpart of the problem, a similar analysis was shown, where the symmetry of the system has a more noticeable role as this mentioned relative coordinates transformation implements a Schwinger quantum angular momentum $SU(2)$. Moreover, the reduction of the degrees of freedom of the problem due to de big magnetic field can be compared to the quantum Hall effect on the plane, where in this regime, the electron states are restricted to the base state.\\

The elegance of the study by Botero et al. \cite{alonso}, and its graceful correlation with the quantum counterpart, leaded us to look for a similar formalism to describe the three-body problem on the spherical surface. The fact that Haldanes \cite{haldane} formalism of the quantum base states for Hall effect on the sphere is very similar to that of the planar quantum Hall effect, leaded us to believe that a formalism similar to Botero et al. on the regime of big magnetic fields could be easily developed for the sphere. Moreover, as we analysed the one-body problem on the sphere as an introduction to the three-body formalism, we discovered that in terms of trajectories, there were many similarities that indicated that the finding of the wanted analogue analysis was actually possible.\\

When we tried to develop the mentioned formalism on the sphere, we found that although there are some analogues for some quantities and some coincidences on the trajectories, the structure of the algebra of guiding center transformation on the sphere is very different from the plane, impeding then further simplification of the problem trough the crucial relative coordinate transformation.\\

We studied the three-body problem on the sphere in Cartesian, spherical and stereographic coordinates, all of which have their advantages and flaws. We found that if we want a very similar analysis from that of the plane, the stereographic projection, due to its nature and its results, is the most suitable candidate. Otherwise, the mapping of the plane formalism to the sphere is not trivial nor even that similar.  This fact is supported by the fact that on the Haldane formalism, some spinor representation introduced to describe the base states is deeply related to the stereographic projection.\\

Given all these hints, we believe that a proper development of a spherical formalism similar to the planar from Botero et al. \cite{alonso}, can be performed through the use of stereographic coordinates. However, as this mapping demonstrated to be not trivial at all, we leave this study for further work.\\

 % Results and Discussion

\chapter{Conclusions}

In this thesis we reproduced a very interesting and elegant approach developed by Botero et al. \cite{alonso} to analyse the three body problem in the plane related to the Hall effect. We showed that in the classical case, this problem is integrable due to the decoupling of the guiding centres and the linear momenta in the regimen of big magnetic fields. Moreover, the canonical transformation of relative coordinates can be manipulated to codify the shape of the triangle in a Bloch sphere. With this analysis, the study of the orientation and shape of the triangle can be studied naturally in terms of the dynamical and geometrical angular velocity of the system, which are in fact the angular variables in the angle-action formalism.\\

In the quantum counterpart of the problem, a similar analysis was shown, where the symmetry of the system has a more noticeable role as this mentioned relative coordinates transformation implements a Schwinger quantum angular momentum $SU(2)$. Moreover, the reduction of the degrees of freedom of the problem due to de big magnetic field can be compared to the quantum Hall effect on the plane, where in this regime, the electron states are restricted to the base state.\\

The elegance of the study by Botero et al. \cite{alonso}, and its graceful correlation with the quantum counterpart, leaded us to look for a similar formalism to describe the three-body problem on the spherical surface. The fact that Haldanes \cite{haldane} formalism of the quantum base states for Hall effect on the sphere is very similar to that of the planar quantum Hall effect, leaded us to believe that a formalism similar to Botero et al. on the regime of big magnetic fields could be easily developed for the sphere. Moreover, as we analysed the one-body problem on the sphere as an introduction to the three-body formalism, we discovered that in terms of trajectories, there were many similarities that indicated that the finding of the wanted analogue analysis was actually possible.\\

When we tried to develop the mentioned formalism on the sphere, we found that although there are some analogues for some quantities and some coincidences on the trajectories, the structure of the algebra of guiding center transformation on the sphere is very different from the plane, impeding then further simplification of the problem trough the crucial relative coordinate transformation.\\

We studied the three-body problem on the sphere in Cartesian, spherical and stereographic coordinates, all of which have their advantages and flaws. We found that if we want a very similar analysis from that of the plane, the stereographic projection, due to its nature and its results, is the most suitable candidate. Otherwise, the mapping of the plane formalism to the sphere is not trivial nor even that similar.  This fact is supported by the fact that on the Haldane formalism, some spinor representation introduced to describe the base states is deeply related to the stereographic projection.\\

Given all these hints, we believe that a proper development of a spherical formalism similar to the planar from Botero et al. \cite{alonso}, can be performed through the use of stereographic coordinates. However, as this mapping demonstrated to be not trivial at all, we leave this study for further work.\\

 % Conclusion

%% ----------------------------------------------------------------
% Now begin the Appendices, including them as separate files

%\addtocontents{toc}{\vspace{2em}} % Add a gap in the Contents, for aesthetics

%\appendix % Cue to tell LaTeX that the following 'chapters' are Appendices

%\chapter{An Appendix}

Lorem ipsum dolor sit amet, consectetur adipiscing elit. Vivamus at pulvinar nisi. Phasellus hendrerit, diam placerat interdum iaculis, mauris justo cursus risus, in viverra purus eros at ligula. Ut metus justo, consequat a tristique posuere, laoreet nec nibh. Etiam et scelerisque mauris. Phasellus vel massa magna. Ut non neque id tortor pharetra bibendum vitae sit amet nisi. Duis nec quam quam, sed euismod justo. Pellentesque eu tellus vitae ante tempus malesuada. Nunc accumsan, quam in congue consequat, lectus lectus dapibus erat, id aliquet urna neque at massa. Nulla facilisi. Morbi ullamcorper eleifend posuere. Donec libero leo, faucibus nec bibendum at, mattis et urna. Proin consectetur, nunc ut imperdiet lobortis, magna neque tincidunt lectus, id iaculis nisi justo id nibh. Pellentesque vel sem in erat vulputate faucibus molestie ut lorem.

Quisque tristique urna in lorem laoreet at laoreet quam congue. Donec dolor turpis, blandit non imperdiet aliquet, blandit et felis. In lorem nisi, pretium sit amet vestibulum sed, tempus et sem. Proin non ante turpis. Nulla imperdiet fringilla convallis. Vivamus vel bibendum nisl. Pellentesque justo lectus, molestie vel luctus sed, lobortis in libero. Nulla facilisi. Aliquam erat volutpat. Suspendisse vitae nunc nunc. Sed aliquet est suscipit sapien rhoncus non adipiscing nibh consequat. Aliquam metus urna, faucibus eu vulputate non, luctus eu justo.

Donec urna leo, vulputate vitae porta eu, vehicula blandit libero. Phasellus eget massa et leo condimentum mollis. Nullam molestie, justo at pellentesque vulputate, sapien velit ornare diam, nec gravida lacus augue non diam. Integer mattis lacus id libero ultrices sit amet mollis neque molestie. Integer ut leo eget mi volutpat congue. Vivamus sodales, turpis id venenatis placerat, tellus purus adipiscing magna, eu aliquam nibh dolor id nibh. Pellentesque habitant morbi tristique senectus et netus et malesuada fames ac turpis egestas. Sed cursus convallis quam nec vehicula. Sed vulputate neque eget odio fringilla ac sodales urna feugiat.

Phasellus nisi quam, volutpat non ullamcorper eget, congue fringilla leo. Cras et erat et nibh placerat commodo id ornare est. Nulla facilisi. Aenean pulvinar scelerisque eros eget interdum. Nunc pulvinar magna ut felis varius in hendrerit dolor accumsan. Nunc pellentesque magna quis magna bibendum non laoreet erat tincidunt. Nulla facilisi.

Duis eget massa sem, gravida interdum ipsum. Nulla nunc nisl, hendrerit sit amet commodo vel, varius id tellus. Lorem ipsum dolor sit amet, consectetur adipiscing elit. Nunc ac dolor est. Suspendisse ultrices tincidunt metus eget accumsan. Nullam facilisis, justo vitae convallis sollicitudin, eros augue malesuada metus, nec sagittis diam nibh ut sapien. Duis blandit lectus vitae lorem aliquam nec euismod nisi volutpat. Vestibulum ornare dictum tortor, at faucibus justo tempor non. Nulla facilisi. Cras non massa nunc, eget euismod purus. Nunc metus ipsum, euismod a consectetur vel, hendrerit nec nunc.	% Appendix Title

%\input{Appendices/AppendixB} % Appendix Title

%\input{Appendices/AppendixC} % Appendix Title

%\addtocontents{toc}{\vspace{2em}}  % Add a gap in the Contents, for aesthetics
%\backmatter

%% ----------------------------------------------------------------
\bibliographystyle{unsrt}  % Use the "unsrtnat" BibTeX style for formatting the Bibliography
\bibliography{Bibliography}

\end{document}  % The End
%% ----------------------------------------------------------------