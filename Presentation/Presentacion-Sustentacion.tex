% Copyright 2004 by Till Tantau <tantau@users.sourceforge.net>.
%
% In principle, this file can be redistributed and/or modified under
% the terms of the GNU Public License, version 2.
%
% However, this file is supposed to be a template to be modified
% for your own needs. For this reason, if you use this file as a
% template and not specifically distribute it as part of a another
% package/program, I grant the extra permission to freely copy and
% modify this file as you see fit and even to delete this copyright
% notice. 

\documentclass[xcolor=dvipsnames]{beamer}
\usepackage{bibentry}
\setbeamertemplate{bibliography item}{\insertbiblabel}
\usepackage{cancel}
\usepackage{amsmath} 
\usepackage{mathtools}
\newcommand{\norm}[1]{\left\lVert #1 \right\rVert}
% Command for round parenthesis
\newcommand{\roundP}[1]{\left( #1 \right)}
% Command for poisson brackets
\newcommand{\poisson}[2]{\left\lbrace #1, #2 \right\rbrace}

\usepackage{varwidth}
\usepackage{lipsum}
\usepackage{color}
\usepackage{todonotes}
\definecolor{dgray}{gray}{0.30}
\definecolor{uyellow}{RGB}{253,241,0}

\usepackage[utf8]{inputenc}
\usepackage{graphicx}
\usepackage{epstopdf}
\usepackage[english]{babel}
\usepackage{hyperref}
\usepackage{datenumber}
\usepackage{todonotes}
\usepackage{mathtools}
\usepackage{amsmath}
\usepackage{amssymb}
\usepackage{amsthm}

\newtheorem*{thm}{Theorem}
\newcommand{\fmunu}{F^{\mu\nu}}
\newcommand{\E}{\vec{E}}
\newcommand{\B}{\vec{B}}
\newcommand{\rot}{\nabla\times}
\newcommand{\dive}{\nabla\cdot}
\newcommand{\tmunu}{T^{\mu\nu}}
\newcommand{\levichi}{\epsilon_{\mu\nu\sigma\gamma}}
\newcommand{\dete}{\textrm{det}}
\newcommand{\lna}{\textrm{ln}}
\newcommand{\Tr}{\textrm{Tr}}
\newcommand{\seno}{\textrm{sin}}

\makeatletter
\newcommand{\pushright}[1]{\ifmeasuring@#1\else\omit\hfill$\displaystyle#1$\fi\ignorespaces}
\newcommand{\pushleft}[1]{\ifmeasuring@#1\else\omit$\displaystyle#1$\hfill\fi\ignorespaces}
\makeatother



% There are many different themes available for Beamer. A comprehensive list with examples is given here:
% http://deic.uab.es/~iblanes/beamer_gallery/index_by_theme.html
%\usetheme{AnnArbor}
%\usetheme{Antibes}
%\usetheme{Bergen}
%\usetheme{Berkeley}
%\usetheme{Berlin}
%\usetheme{Boadilla}
%\usetheme{boxes}
%\usetheme{CambridgeUS}
%\usetheme{Copenhagen}
%\usetheme{Darmstadt}
%\usetheme{default}
%\usetheme{Frankfurt}
\usetheme{Goettingen}
%\usetheme{Hannover}
%\usetheme{Ilmenau}
%\usetheme{JuanLesPins}
%\usetheme{Luebeck}
%\usetheme{Madrid}
%\usetheme{Malmoe}
%\usetheme{Marburg}
%\usetheme{Montpellier}
%\usetheme{PaloAlto}
%\usetheme{Pittsburgh}
%\usetheme{Rochester}
%\usetheme{Singapore}
%\usetheme{Szeged}
%\usetheme{Warsaw}
%\usecolortheme{seagull}
%\usetheme{Malmoe} 
\setbeamercolor{frametitle}{fg=Black,bg=Blue!60}
\setbeamercolor{section in head/foot}{bg=Blue, fg=Black}
\setbeamercolor{author in head/foot}{bg=blue, fg=Black} 
\setbeamercolor{date in head/foot}{fg=blue} 
\setbeamercolor{institute in head/foot}{fg=Black}
\usecolortheme[named=Black]{structure}
\setbeamerfont{footnote}{size=\footnotesize}

\setbeamertemplate{}[page number] % To replace the footer line in all slides with a simple slide count uncomment this line
 
\title{\textbf{El problema de tres cuerpos en la superficie esférica}}

% A subtitle is optional and this may be deleted
%\subtitle{Optional Subtitle}

\author{Jesus Prada}
% - Give the names in the same order as the appear in the paper.
% - Use the \inst{?} command only if the authors have different affiliation.

\institute[{\color{Black} Universidad de los Andes}] % (optional, but mostly needed)
{
 \vspace{5mm} \normalsize Advisor: PhD Alonso Botero \\ \vspace{6mm} \small  Universidad de los Andes, Departamento de F\'isica
}
% - Use the \inst command only if there are several affiliations.
% - Keep it simple, no one is interested in your street address.

\tiny
\date{ \footnotesize Marzo 9, 2016}
% - Either use conference name or its abbreviation.
% - Not really informative to the audience, more for people (including yourself) who are reading the slides online
	

\subject{Quantum and classical N body Problem}
% This is only inserted into the PDF information catalog. Can be left out. 

% If you have a file called "university-logo-filename.xxx", where xxx  is a graphic format that can be processed by latex or pdflatex, resp., then you can add a logo as follows:
% \pgfdeclareimage[height=0.5cm]{university-logo}{university-logo-filename}
% \logo{\pgfuseimage{university-logo}}

% Delete this, if you do not want the table of contents to pop up at the beginning of each subsection:
%\AtBeginSubsection[] { \begin{frame}<beamer>{Outline} \tableofcontents[currentsection,currentsubsection] \end{frame}}

% Let's get started
\begin{document}

\begin{frame}
  \titlepage
\end{frame}
%--------------------------------------------
\begin{frame}{Outline}
 \tableofcontents
  % You might wish to add the option [pausesections]
\end{frame}
%------------------------------------------------
\section{Introducción}
\subsection{Introducción}
\begin{frame}
\begin{itemize}
\item El problema de N cuerpos trata de estudiar las trayectorias que seguir\'ian N partículas interactuando con fuerzas externas e internas, teniendo en cuenta la informaci\'on de las condiciones iniciales.\\~\\ 
\item Es de inter\'es para la comprensi\'on de la mec\'anica cl\'asica. Fue as\'i como Poincar\'e plante\'o las bases de la teor\'ia del caos. \cite{poincare}
\end{itemize}
\end{frame}
%--------------------------------------------------------------------------------------
\subsection{Motivación}
\begin{frame}
\begin{itemize}
\item Para $N\geq 3$ los sistemas son en general no integrables. Varios casos son integrables pero involucran Fuerzas poco realistas \cite{strangeCases}.\\~\\ 
\item Alonso Botero et al. demostraron en \cite{alonso} la integrabilidad de un caso de tres partículas cargadas en el plano.\\~\\ 
\item El caso incluye potenciales centrales y puede ser tomado como modelo de electrones para estudiar el efecto Hall clásico y cuántico.\\~\\ 
\item La motivación principal es extender el análisis en \cite{alonso} al análogo caso de las esfera.\\~\\
\end{itemize}
\end{frame}

%---------------------------------------------------------------------------------------
%-------------------------------------------------------------------------
\section{El problema de 3 cuerpos en el plano}
\subsection{Integrabilidad del problema clásico}
\begin{frame}
El Hamiltoniano del problema propuesto está dado por:
\begin{equation*}
H = \sum_{i=1}^{3} \frac{1}{2m} \norm{ \vec{p_i} - 
e\vec{ A } \left( \vec{q_i} \right)}^2
+ V \roundP{ \vec{q_1},\vec{q_2},\vec{q_3} }
+\frac{m\omega}{2}\sum_{i=1}^{3} \norm{\vec{q_i}}^2
\end{equation*}
\\~\\
Se propone la transformación canónica de los centros guía:
\begin{align*}
\vec{\pi_i} &= \vec{p_i} - e\vec{A}\roundP{\vec{q_i}}\\
\vec{R_i} &= \vec{q_i} - \frac{\hat{k}\times\vec{\pi_i}}{eB}
\end{align*}
Que es canónica dados los corchetes de Poisson:
\begin{align*}
\poisson{\pi_{i,\alpha}}{\pi_{j,\beta}}&=\roundP{eB}\delta_{ij}\epsilon_{\alpha \beta}\\  
\poisson{R_{i,\alpha}}{R_{j,\beta}}&= -\roundP{eB}^{-1} \delta_{ij}\epsilon_{\alpha \beta}\\  
\poisson{R_{i,\alpha}}{\pi_{j,\beta}}&=0
\end{align*}
\end{frame}
%---------------------------------------------------------------------------
%-------------------------------------------------------------------------
\begin{frame}
En unidades necesarias y bajo un gran $B$ el Hamiltoniano es separable y el problema se reduce a solucionar el Hamiltoniano de los centros guía con $R_{i,\alpha} = \{(x_1,x_2,x_3),(y_1,y_2,y_3)\}$:
\begin{equation*}
H_{gc} = \frac{{\omega}}{2} \sum_{i=1}^{3} \norm{\bar{x}^2} + \norm{\bar{y}^2}
+ V\roundP{\bar{x},\bar{y}}
\end{equation*}
El cual es integrable dadas las integrales en involución \cite{scheck}:
\begin{align*}
R_z = \frac{1}{2} \sum_{i=1}^{3} \roundP{x_i^2 + y_i^2}\\
L = T_x^2 + T_y^2 = \roundP{\sum_{i=1}^{3} x_i}^2 + \roundP{\sum_{i=1}^{3} y_i}^2\\
\end{align*}
\end{frame}
%---------------------------------------------------------------------------
%-------------------------------------------------------------------------
\subsection{La representación espinorial y la esfera de Bloch}
\begin{frame}
El centro de masa del sistema describirá movimiento circular uniforme. El problema es reducido a analizar el comportamiento de las coordenadas relativas al centro de masa descritas por el espinor:

\begin{align*}
\Psi = \frac{1}{2\sqrt{3}}\begin{pmatrix}\sqrt{3}\roundP{z_{2}-z_{1}}\\
z_{2}+z_{1}-2z_{2}\end{pmatrix},
\end{align*}
Donde $z = x+iy$. Los corchetes de Poisson son:
\begin{align*}
\poisson{\Psi_\alpha}{\Psi_\beta} &= \poisson{\Psi_\alpha^*}{\Psi_\beta^*} = 0\\
\poisson{\Psi_\alpha^*}{\Psi_\beta} &= i\delta_{\alpha,\beta}.
\end{align*}
\end{frame}
%---------------------------------------------------------------------------
%-------------------------------------------------------------------------
\begin{frame}
La una fase aplicada a $\Psi$ representa una rotación alrededor del centro de masa. Entonces $\Psi$ está asociado con el grupo de simetría $SU(2)$:

\begin{equation*}
\vec{\zeta} = \frac{1}{S} \Psi^{\dagger}\vec{\sigma}\Psi.
\end{equation*}

Con $S$ una constante de movimiento. Los valores esperados de $\vec{\sigma}$ codifican la forma del triangulo: 

\begin{equation*}
\begin{aligned}
\rho_k &= 2S\roundP{1+\vec{m}_k\cdot\vec{\zeta}}\\
\vec{m}_k &= \roundP{\sin{\frac{2\pi k}{3}},0,\cos{\frac{2\pi k}{3}}}, k \in \poisson{1,2}{3}\\
A &= \frac{\sqrt{3}S}{2} \zeta_2.
\end{aligned}
\end{equation*}

Donde $\rho_k$ es el lado del triángulo opuesto a la partícula $k$.

\end{frame}
%---------------------------------------------------------------------------
%---------------------------------------------------------------------------
%-------------------------------------------------------------------------
\subsection{Análisis del movimiento}
\begin{frame}
Podemos construir un Hamiltoniano más apropiado para estudiar el movimiento de las partículas a través de $\Psi$:

\begin{align*}
H_{\Psi} &= H_{gc} - \omega L =  V\roundP{\Psi,\Psi^*}+ \omega (J-L) = V\roundP{\Psi,\Psi^*}+ \omega S\\
&= V\roundP{\Psi_\alpha,\Psi_\alpha^*}+ \omega\Psi_\alpha\Psi^*_\alpha.
\end{align*}

Donde las ecuaciones de movimiento para $\Psi$ serán:

\begin{align*}
i\dot{\Psi} &= i\poisson{\Psi}{H_{\Psi}} = \frac{\partial V}{\partial \Psi^*} + \omega \Psi.\\
&= \roundP{\omega +\frac{\partial V}{\partial S}+ \frac{1}{S}\frac{\partial V}{\partial \zeta_j}\roundP{-\zeta_j \mathbb{I}  + \sigma^j }}\Psi.
\end{align*}

\end{frame}
%---------------------------------------------------------------------------
%-------------------------------------------------------------------------
\begin{frame}
En coordenadas de la esfera de Bloch:
\begin{align*}
\dot{\vec{\zeta}} &= \frac{2}{S}\roundP{\nabla_{\vec{\zeta}}V}\times\vec{\zeta}. 
\end{align*}

Pero se necesita información de la fase asociada a $\Psi$ que codifica la rotación del triángulo. Defínase una fase entre dos estados $\Psi$ infinitesimalmente cercanos:

\begin{align*}
e^{-id\chi}\Psi &\approx \Psi -id\chi\Psi = \Psi + d\Psi\\
d\chi &= i\frac{\Psi^\dagger d\Psi}{\Psi^\dagger\Psi}.
\end{align*}

De esta manera obtenemos una velocidad angular dinámica asociada, dada por:

\begin{align*}
\omega_r^{(dyn)} &= \omega + \frac{\partial V}{\partial S}.
\end{align*}

\end{frame}
%---------------------------------------------------------------------------
%-------------------------------------------------------------------------
\begin{frame}
Si parametrizamos el espinor $\Psi$ de la siguiente manera, podemos obtener información de la rotación del triángulo:
\small
\begin{equation*}
\Psi = \sqrt{S}e^{-i\gamma} \begin{pmatrix}\cos{\frac{\theta}{2}}\\
e^{-i\phi}\sin{\frac{\theta}{2}}\end{pmatrix}.
\end{equation*}

\begin{align*}
\omega_r^{(dyn)} &= i\frac{1}{S}\Psi^\dagger\dot{\Psi} \\
&= \dot{\gamma}+\dot{\phi}\sin^2{\frac{\theta}{2}},
\end{align*}
\normalsize
La velocidad de rotación sobre un periodo $T_s$ es:
\small
\begin{align*}
\omega_r &= \frac{\Delta \gamma}{T_s} = \frac{1}{T_s}\int_0^{T_s}\omega_r^{(dyn)}dt -\frac{1}{T_s}\oint\sin^2{\frac{\theta}{2}}d\phi\\
\\
\omega_r &= \left\langle \omega_r^{(dyn)} \right\rangle + \omega_r^{(geo)}\\
\end{align*}
\normalsize


\end{frame}

%---------------------------------------------------------------------------------------
%-------------------------------------------------------------------------
\subsection{El análogo cuántico}
\begin{frame}
El problema es escencialmente el mismo, teniendo en cuenta las reglas de la cuantización canónica \cite{Cq}. 
\begin{equation*}
H_{gc} = V\roundP{\Psi,\Psi^\dagger}+ \omega (S+L).
\end{equation*}
\begin{align*}
\begin{pmatrix} b \\ \Psi_1 \\ \Psi_2 \end{pmatrix} &= 
\begin{pmatrix}\frac{1}{\sqrt{3}} &\frac{1}{\sqrt{3}}&\frac{1}{\sqrt{3}}\\
				-\frac{1}{\sqrt{2}}&\frac{1}{\sqrt{2}}&0\\
				\frac{1}{\sqrt{6}}&\frac{1}{\sqrt{6}}&\frac{1}{\sqrt{6}}\end{pmatrix}
				\begin{pmatrix} a_1 \\ a_2 \\ a_3 \end{pmatrix},
\end{align*}

Donde $a_i= \frac{1}{\sqrt{2}}(x_i+iy_i)$ son operadores creación aniquilación. Se tienen RCC análogas:

\begin{align*}
\left[ \Psi_\alpha,\Psi_\beta^\dagger\right] &= \delta_{\alpha\beta}\\
\left[ b,b^\dagger\right] &= 1,
\end{align*}


\end{frame}
%---------------------------------------------------------------------------
%-------------------------------------------------------------------------

\begin{frame}
Además $bb^\dagger = L-\frac{1}{2}$ y $\Psi^\dagger\Psi = S-1$, indica que $L$ y $S$ son cuantizados por números enteros y semienteros respectivamente. \\~\\

La simetría SU(2) asociada a $\Psi$ se hace clara al notar que $\Psi$ implementa un momento angular de Schwinger \cite{Schwinger}:
\begin{align*}
\vec{F} &= \frac{1}{2}\Psi^\dagger \vec{\sigma}\Psi,
\end{align*}

Con $[F_i,F_j] = \epsilon_{ijk}F_k$ y $F^2 = \frac{S^2 -1}{4}$. Dado esto, el las energías propias estarán determinadas:

\begin{equation*}
E_{l,m,s} = V_{s,m} + \omega(s+l).
\end{equation*}
\end{frame}
%---------------------------------------------------------------------------
%-------------------------------------------------------------------------

\section{El problema de 3 cuerpos en la esfera}
\subsection{El problema de una partícula bajo el campo magnético monopolar}
\begin{frame}
El Lagrangiano y Hamiltoniano de la partícula están dados por:
\begin{align*}
L\roundP{\vec{x},\dot{\vec{x}}} = \frac{m}{2}\norm{\dot{\vec{x}}}^2 - e\vec{A}_{\hat{u}}(\vec{x})\cdot\dot{\vec{x}}\\
H\roundP{\vec{x},\vec{p}} = \left.\frac{1}{2m}\norm{\vec{p}+e\vec{A}_{\hat{u}}(\vec{x})}^2\right|_{S^2}
\end{align*}
Con \cite{vectorPotentials}:
\begin{equation*}
\vec{A}_{\hat{u}}(\vec{x}) = \frac{g}{r}\frac{\hat{u}\times\hat{r}}{1+\hat{u}\cdot\hat{r}}
\end{equation*}
\end{frame}
%---------------------------------------------------------------------------
%---------------------------------------------------------------------------
\begin{frame}
Dadas las simetrías del problema se deduce que la trayectoria está restringida a un cono:

\begin{align*}
\mathbb{J} &= \vec{x}\times\vec{\pi}+{eg}\hat{x} = m{\vec{x}}\times\dot{\vec{x}} + eg\hat{x}
\end{align*}
\begin{equation}
\begin{aligned}
\norm{\vec{J}}^2 &= \norm{\mathbb{L}}^2 +\roundP{ge}^2\\
\norm{\mathbb{L}} &= cte\\
\cos{\theta} &= \frac{\vec{J}\cdot\hat{x}}{\norm{\vec{J}}} = \sqrt{\frac{\roundP{ge}^2}{\norm{\mathbb{L}}^2 +\roundP{ge}^2}}
\end{aligned}
\label{eq:poincarecone}
\end{equation}
En el formalismo Hamiltoniano dicho $\mathbb{J}$ satisface las relaciones Poisson de un momento angular canónico \cite{haldane}.\\
\end{frame}
%---------------------------------------------------------------------------
%---------------------------------------------------------------------------
%---------------------------------------------------------------------------
\subsection{En coordenadas cartesianas}
\begin{frame}
El Hamiltoniano en coordenadas cartesianas está dado por:
\begin{align*}
H &= \left. \frac{1}{2m}\sum_{i=1}^2\norm{\vec{p}_i+e\vec{A}_{\hat{u}}(\vec{r}_i)}^2+ V\roundP{\vec{r}_1,\vec{r}_2,\vec{r}_3}    \right|_{S^2},
\end{align*}
Definimos el momento angular de los centros guia:
\begin{align*}
\vec{J}_i &= \vec{r}_i\times\vec{\pi}_i + eg\hat{r}_i\\
&= \vec{r}_i\times\roundP{\vec{p}_i+e\vec{A}_{\hat{u}}(\vec{r}_i)} + eg\hat{r}_i\\
\poisson{J_i}{J_j} &= \epsilon_{ijk}J_k,
\end{align*}
E intentamos realizar la transformación de los centros guía, la cual descubrimos no es canónica:
\begin{align*}
\vec{R}_i &= \frac{\vec{J}_i}{\frac{eg}{r}} = \vec{r}_i + \frac{\hat{r}_i\times\vec{\pi}_i}{\frac{eg}{r^2}} \nonumber\\
&= \vec{r}_i + \frac{\hat{r}_i\times\vec{\pi}_i}{eB}.
\end{align*}
\end{frame}

%---------------------------------------------------------------------------
%---------------------------------------------------------------------------
\subsection{En coordenadas esféricas}
\begin{frame}
El Hamiltoniano en coordenadas esféricas el hamiltoniano de una sola partícula está dado por:
\begin{equation}
H = \frac{1}{2mr^2}\roundP{p_\theta^2 + \roundP{\frac{p_\phi - \frac{eg}{r}\tan{\frac{\theta}{2}}}{\sin\theta}}^2}.
\end{equation}
El momento lineal de la partícula determina su centro guía:
\begin{align*}
\pi_\theta &= \frac{p_\theta}{r}\\
\pi_\phi &= \frac{p_\phi - \frac{eg}{r}\tan{\frac{\theta}{2}}}{r\sin\theta}.
\end{align*}
Los momentos angulares asociados a cada variable harían el papel de momentos lineales:

\begin{align*}
L_\theta &= p_\theta\\
L_\phi &= p_\phi - \frac{eg}{r}\tan{\frac{\theta}{2}}\\
\poisson{L_\theta}{L_\phi} &=  \frac{eg}{r} \frac{1}{1+\cos{\theta}}.
\end{align*}
\end{frame}

%---------------------------------------------------------------------------
%---------------------------------------------------------------------------
\subsection{En coordenadas estereográficas}
\begin{frame}
La proyección estereográfica proyecta la esfera de radio $r$ al plano de la siguiente manera:
\begin{align*}
X &= \frac{x}{1+z/r}\\
Y &= \frac{y}{1+z/r}.
\end{align*}
La expresión del Hamiltoniano en estas coordenadas no es trivial. Por esto, empezamos con el formalismo Lagrangiano:
\small
\begin{align*}
L &= \frac{m}{2}\roundP{\frac{\dot{X}^2+\dot{Y}^2}{\roundP{\frac{1}{2r^2}\roundP{r^2+X^2+Y^2}}^2}} + \frac{eg}{r^2}\frac{X\dot{Y}-Y\dot{X}}{\frac{1}{2r^2}\roundP{r^2+X^2+Y^2}} \nonumber \\
&= \frac{m}{2}\roundP{\roundP{\frac{\dot{X}}{K}}^2+\roundP{\frac{\dot{Y}}{K}}^2} + eB\roundP{X\roundP{\frac{\dot{Y}}{K}}-Y\roundP{\frac{\dot{X}}{K}}},
\end{align*}
\normalsize
Con $K=\frac{1}{2r^2}\roundP{r^2+X^2+Y^2}$ un factor de escala debido a que la transformación es conformal.

\end{frame}

%---------------------------------------------------------------------------
%---------------------------------------------------------------------------
\begin{frame}
Pasando al formalismo Hamiltoniano:
\begin{align*}
H &= \frac{1}{2m}\roundP{\roundP{KP_X-eBY}^2+\roundP{KP_Y+eBX}^2}.
\end{align*}
Y definimos los momentos lineales en el plano estereográfico:
\begin{align*}
\pi_X &= KP_X-eBY\\
\pi_Y &= KP_Y+eBX
\end{align*}
Cuyos corchetes de Poisson están dados por:
\begin{align*}
\poisson{\pi_X}{\pi_Y} &= -K\roundP{2B + \frac{1}{r^2}\roundP{P_YX-P_XY}}\\
 &= -B + \frac{m}{K}\roundP{Y\dot{X}-X\dot{Y}}.
\end{align*}
\small
A pesar de que se llega a un resultado en el que es necesario aproximar para llegar a la transformación canónica, las coincidencias de esta transformación con el caso planar nos llevan a pensar que son las coordenadas más apropiadas para continuar el trabajo.
\end{frame}

\section{References}
\begin{frame}[t, allowframebreaks]
\frametitle{Referencias}
\tiny
\bibliographystyle{unsrt}
\bibliography{Bibliography}
\end{frame}
\end{document}
